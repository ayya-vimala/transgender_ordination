\section{Gratitude}

When writing this paper I have had help and input from so many friends. It became clear to me how important this issue is for so many people and I feel very grateful that I have had the opportunity and time to dedicate to this worthwhile cause.\\

I wish to especially thank my Spiritual Advisor, Ven. Anandabodhi Bhikkhunī, who has helped me to understand my own spiritual journey as a queer female monastic in a conservative and patriarchal Buddhist {\em Saṅgha}. I also thank my good friends Ayya Yeshe Chödrön and Ven. Akāliko Bhikkhu, who have always stood by me, encouraged me to do this research and in fact are the impetus for it.\\

With regards to the practical work on this paper, I wish to express my heartfelt gratitude to Ven. Ajahn Brahmali Bhikkhu for his input, translations, discussions and feedback, to Brenna Artinger for their thorough research, to Dr. Hsiao-Lan Hu for research and feedback in regards to the Chinese texts and to Derek Sola for proofreading.\\

I feel grateful to Ven. Sujato Bhikkhu, Dr. Orna Almogi, Sebastian Nehrdich and Blake Walsh for their work on the websites \href{https://suttacentral.net/}{SuttaCentral.net} and \href{https://buddhanexus.net/}{BuddhaNexus.net}, which made research into this subject so much easier.\\

And lastly, I wish to thank all the people, and in particular the LGBTIQA+ community and other female monastics, who have supported me as a monastic all these years and whose stories have touched me deeply.
