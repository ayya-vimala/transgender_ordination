\section{Changing Liṅga}
\label{trans}

In this chapter I want to pay some special attention to a very interesting passage in the Buddhist Canon. The \textit{Theravāda Vinaya} \textit{Pā­rāji­ka} 1 describes the curious case where a monk changes \textit{liṅga} and is thereafter seen as a woman. She is then admitted into the \textit{Bhikkhunī} Order. The same is repeated for a nun who changes \textit{liṅga} and is from that moment on a \textit{Bhikkhu}.\footnote{Translation by Ajahn Brahmali, \textit{Pā­rāji­ka} 1, PTS Vol. 3, page 35.} 

\begin{quote}
\textit{Tena kho pana samayena aññatarassa bhikkhuno itthiliṅgaṁ pātubhūtaṁ hoti. Bhagavato etamatthaṁ ārocesuṁ. ``Anujānāmi, bhikkhave, taññeva upajjhaṁ tameva upasampadaṁ tāniyeva vassāni bhikkhunīhi saṅgamituṁ. Yā āpattiyo bhikkhūnaṁ bhikkhunīhi sādhāraṇā tā āpattiyo bhikkhunīnaṁ santike vuṭṭhātuṁ. Yā āpattiyo bhikkhūnaṁ bhikkhunīhi asādhāraṇā tāhi āpattīhi anāpattī''ti.}

\textit{Tena kho pana samayena aññatarissā bhikkhuniyā purisaliṅgaṁ pātubhūtaṁ hoti. Bhagavato etamatthaṁ ārocesuṁ. ``Anujānāmi, bhikkhave, taññeva upajjhaṁ tameva upasampadaṁ tāniyeva vassāni bhikkhūhi saṅgamituṁ. Yā āpattiyo bhikkhunīnaṁ bhikkhūhi sādhāraṇā tā āpattiyo bhikkhūnaṁ santike vuṭṭhātuṁ. Yā āpattiyo bhikkhunīnaṁ bhikkhūhi asādhāraṇā tāhi āpattīhi anāpattī''ti.}
\end{quote}

\begin{quote}
At one time the \textit{liṅga} of a woman appeared on a monk. They told the Master. He said: ``Monks, I allow that very discipleship, that very ordination, those years as a monk, to be transferred to the nuns. The monks' offenses that are in common with the nuns are to be dealt with in the presence of the nuns. For the monks' offenses that are not in common with the nuns, there's no offense.''

At one time the \textit{liṅga} of a man appeared on a nun. They told the Master. He said: ``Monks, I allow that very discipleship, that very ordination, those years as a nun, to be transferred to the monks. The nuns' offenses that are in common with the monks are to be dealt with in the presence of the monks. For the nuns' offenses that are not in common with the monks, there's no offense.''
\end{quote}

The appearance of this passage in \textit{Pārājika} 1 is a bit odd. This rule is to do with sexual intercourse and obviously a change of characteristics has nothing much to do with that. It is likely that this passage was added later. The same passage is found in the texts of several of the Chinese schools\footnote{This passage possibly appears in all of the Chinese schools but I have been unable to locate it.} but in a different section, namely below the passages on ordination. This seems more logical as there is a question implied here about ordination, namely if (s)he needs to re-ordain or needs a new preceptor. Again the Chinese characters are confusing here, mixing up the characters for `root' and `shape', which seem to be used as synonyms.

The Buddha seems to handle this rather curious matter in a very matter-of-fact way. The monastic in question is simply assigned to the other Order while keeping their years of seniority as well as their preceptor. It does not seem to be a problem at all. He simply responds in the compassionate way we would expect.

In regards to this passage in the \textit{Vinaya}, Carol Anderson\footnote{\cite{anderson2016a}.} argues that this actually refers to the possibility of biological sex change as well as a change of gender characteristics on the basis that both the canonical passage as well as the commentaries interpret the word \textit{liṅga} to refer to both the biological sex as well as gender-expression. The distinction between anatomical sex and gender (either social roles or personal identification based on an internal awareness) that we have today is not made in ancient India.

The section on the monk changing \textit{liṅga} is discussed in the \textit{Samantapāsādikā} and its Chinese equivalent (T24 1462 善見律毘婆沙). The most striking thing about the commentarial explanation is that the change in \textit{liṅga} happens overnight and might also revert back. In fact the monastic in question can revert back and forth several times. This is something that is attributed to \textit{kamma}.\footnote{\cite{heirman} page 430 notes that when asleep one loses control and this can lead to shameful situations. Therefore, sexual misconduct can happen during sleep like erotic dreams or the emission of semen. Another possible explanation could be due to sexual orientation. The commentaries mention that this happens when the monk is sleeping under the same roof as another monk (at least before they go to sleep) and the reverse case for a nun. If in such a case an erotic dream occurs that has to do with this other monk (/nun) i.e. same-sex attraction and the word \textit{liṅga} also includes what is described as \textit{veda} by the Jains, it is possible that what we have here is that this same-sex attraction is seen as a female characteristic.} A likely explanation of this passage is that we are dealing here with a highly academic stance with the aim of explaining something that was not well understood at the time the commentary was written. Carol Anderson argues that the commentarial passage can be seen as a teaching mechanism to illustrate that male characteristics are a result of good \textit{kamma} in past lives while female characteristics are a result of bad \textit{kamma}. This patriarchal stance is found in all Buddhist traditions so is not entirely unexpected. 

To conclude, we can merely say that this passage is important but also raises questions. Its position near the bottom of \textit{Pārājika} 1 and in the sections on ordination in the Chinese \textit{Vinayas} seem to point to a later inclusion, similar to other passages found in the \textit{Vinaya} that have to do with gender non-conforming individuals. 

The question remains as to what has changed exactly. As discussed in chapter \ref{linga} I believe that the term \textit{liṅga} refers to a combination of sexual characteristics and gender-expressions i.e. what a person sees when they decide if this is a man, a woman or neither. There is no indication that the Buddhists followed the Jains in including sexuality in this term and monastics have already given up gendered attire and shaven their heads. Therefore the change of \textit{liṅga} in this passage must refer to either physical characteristics like breasts, genitals, body hair, etc. as well as to gender-expressions like behavior.

In a time when Hormone Replacement Therapy and surgery were not available it does not seem to be likely that anybody just changed sex from one day to the next. One possible explanation can be found in the rare case where somebody is raised as a boy or girl but during puberty turns out to be the opposite when sex markers become more apparent. We know from the \textit{Vinaya} that children were ordained very young and before puberty. This could be an indication that what we now define as intersex was not seen as an obstacle to ordination. 

Although the monastic in question in this passage changes \textit{liṅga}, they also seem to be different from an \textit{ubhatob­yañ­janaka}. After all, they are allowed to stay in robes and their change of \textit{liṅga} is not treated as anything special.\footnote{Note that in this passage the change mentioned is a change in \textit{liṅga} while the \textit{ubhatob­yañ­janaka} is described as changing \textit{nimitta} or \textit{byañjana} and \textit{liṅga} is only mentioned in relation to rebirth, quoting a commentary to the \textit{Abhidhamma}. I have used these terms as synomyms throughout to describe sex/gender characteristics but it might be that there is a more subtle difference. In the Chinese \textit{Vinayas} the same character is used in both cases.} This is all the more evidence that the term \textit{ubhatob­yañ­janaka} does not mean what we know today as intersex, nor transgender. The \textit{ubhatob­yañ­janaka} is described as hyperlibidinous and being able to change characteristics at will for the purpose of sexual intercourse, while the monastic in this passage is obviously quite keen to stay celibate and practice as a monastic. The aspect of intention is important here as the monastic in this story is not in control of the change.