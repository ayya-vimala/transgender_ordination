\section{Changing Gender}

In this chapter I want to pay some special attention to a very interesting passage in the Buddhist canon. The Theravāda Vinaya {\em Pā­rāji­ka} 1 describes the curious case where a monk changes gender characteristics and is now seen as a woman. She is then admitted into the Bhikkhunī order. The same is repeated for a nun who changes sex/gender and is from that moment on a Bhikkhu\footnote{Translation by Ajahn Brahmali, {\em Pā­rāji­ka} 1, PTS Vol. 3, page 35.}. 

\begin{quote}
{\em Tena kho pana samayena aññatarassa bhikkhuno itthiliṅgaṃ pātubhūtaṃ hoti. Bhagavato etamatthaṃ ārocesuṃ. “Anujānāmi, bhikkhave, taññeva upajjhaṃ tameva upasampadaṃ tāniyeva vassāni bhikkhunīhi saṅgamituṃ. Yā āpattiyo bhikkhūnaṃ bhikkhunīhi sādhāraṇā tā āpattiyo bhikkhunīnaṃ santike vuṭṭhātuṃ. ”Yā āpattiyo bhikkhūnaṃ bhikkhunīhi asādhāraṇā tāhi āpattīhi anāpattī”ti.

Tena kho pana samayena aññatarissā bhikkhuniyā purisaliṅgaṃ pātubhūtaṃ hoti. Bhagavato etamatthaṃ ārocesuṃ. “Anujānāmi, bhikkhave, taññeva upajjhaṃ tameva upasampadaṃ tāniyeva vassāni bhikkhūhi saṅgamituṃ. Yā āpattiyo bhikkhunīnaṃ bhikkhūhi sādhāraṇā tā āpattiyo bhikkhūnaṃ santike vuṭṭhātuṃ. ”Yā āpattiyo bhikkhunīnaṃ bhikkhūhi asādhāraṇā tāhi āpattīhi anāpattī”ti.}
\end{quote}

\begin{quote}
At one time the characteristics of a woman appeared on a monk. They told the Master. He said: “Monks, I allow that very discipleship, that very ordination, those years as a monk, to be transferred to the nuns. The monks’ offenses that are in common with the nuns are to be dealt with in the presence of the nuns. For the monks’ offenses that are not in common with the nuns, there’s no offense.”

At one time the characteristics of a man appeared on a nun. They told the Master. He said: “Monks, I allow that very discipleship, that very ordination, those years as a nun, to be transferred to the monks. The nuns’ offenses that are in common with the monks are to be dealt with in the presence of the monks. For the nuns’ offenses that are not in common with the monks, there’s no offense.”
\end{quote}

The appearance of this passage in {\em Pārājika} 1 is a bit odd. This rule has to do with sexual intercourse and obviously a change of characteristics has nothing much to do with that. It is likely that this passage was added later. The same passage is found in several of the Chinese schools\footnote{This passage possibly appears in all of the Chinese schools but I have been unable to locate it.} but in a different section, namely below the passages on ordination. This seems more logical as there is a question implied here about ordination, namely if he/she needs to re-ordain or needs a new preceptor. Again the Chinese words are confusing here, mixing up the words for 'root' and 'shape', which seem to be used as synonyms.

The Buddha seems to handle this rather curious matter in a very matter-of-fact way. The monastic in question is simply assigned to the other order while keeping their years of seniority as well as their preceptor. It does not seem to be a problem at all. He simply responds in the compassionate way we would expect.

In regards to this passage in the Vinaya Carol Anderson\footnote{See \cite{anderson2016a}.} argues that this actually refers to the possibility of biological sex change as well as a change of gender on the basis that both the canonical passage as the commentaries interpret the word {\em liṅga} to refer to both the biological sex as well as gender characteristics. The distinction between anatomical sex and culturally constructed gender as we have today is not made in Classical India.

The section on the monk changing gender is discussed in the {\em Samantapāsādikā} and its Chinese equivalent (T24 1462 善見律毘婆沙). The most striking about the commentarial explanation is that the change in {\em liṅga} happens overnight and might also revert. In fact the monastic in question can revert back and forth several times. This is something that is attributed to kamma\footnote{\cite{heirman} page 430 notes that when asleep one looses control and this can lead to shameful situations. Therefore, sexual misconduct can happen during sleep like erotic dreams or the emission of semen. Another possible explanation could be to sexual orientation. The commentaries mention that this happens when the monk is sleeping under the same roof as another monk (at least before they go to sleep) and the reverse case for a nun. If in such a case an erotic dream occurs that has to do with this other monk (/nun) i.e. homosexual attraction and the word {\em liṅga} also includes what is described as {\em veda} by the Jains, it is possible that what we have here is that this homosexual attraction is seen as a female characteristic. This is however speculation on my part and there is no proof of such a position, but it remains curious that a change of sex would happen overnight; it is far more likely that a person would suddenly find out about their homosexual sexuality overnight.}. A likely explanation of this passage is that we are dealing here with a highly academic stance with the aim of explaining something that was not well understood at the time the commentary was written. But as Carol Anderson argues, the commentarial passage can be seen as a teaching mechanism to illustrate that male characteristics are a result of good kamma in past lives while female characteristics are a result of bad kamma. This patriarchal stance is found in all Buddhist traditions so is not entirely unexpected. 

To conclude, we can merely say that this passage is important but also raises questions. It's position near the bottom of {\em Pārājika} 1 and in the sections on ordination in the Chinese Vinaya seem to point to a later inclusion, similar to other passages found in the Vinaya that have to do with gender non-conform individuals. In a time when Hormone Replacement Therapy and surgery were not available it does not seem to be likely that anybody just changes gender from one day to the next. One possible explanation can be found in the rare case where somebody is raised as a boy or girl but during puberty turns out to be the opposite when sex markers become more apparent. We know from the Vinaya that children were ordained very young and before puberty. But in this case this would be an intersex person. This would be an indication that what we now define as intersex was not seen as an obstacle to ordination.

Although the monastic in question in this passage changes gender, they also seem to be something different from an {\em ubhatob­yañ­janaka}. After all, they are allowed to stay in robes and their change in sex/gender is not treated as anything special. This is all the more evidence that the {\em ubhatob­yañ­janaka} does not mean what we know today as intersex, nor transgender. The {\em ubhatob­yañ­janaka} is described as hyperlibidinous and being able to change sex/gender at will for the purpose of sexual intercourse, while the monastic in this passage is obviously quite keen to stay celibate and practice as a monastic. They are also not in control of the change. The aspect of intention is important here and in the case of intersex individuals it is clear they are not intentionally so.
