\section{Appendix 1: Gender non-conformity in Chinese Vinayas of the different schools}
\label{appendix1}

In this appendix, I will limit myself to describing the term that are relevant with regards to gender-nonconform people as they appear in the texts in the Chinese Vinayas of different schools.

% \subsection{Theravāda Vinaya}
% The Theravāda Vinaya\footnote{Khandhaka 1 Pabbajjā PTS vol 1 page 85–86} describes a {\em paṇḍaka} monk who is trying to have sex with monks and novices but is rebuked each time. He finally manages with the elephant and horse-keepers. The matter is brought to the Buddha who lays down a rule saying {\em paṇḍaka} cannot ordain and if they are already ordained they need to be expelled.

% Further down there is the following passage\footnote{Khandhaka 1 Pabbajjā PTS vol 1 page 89, translation by Ajahn Brahmali}:

% \begin{quote}
% At one time an {\em ubhatob­yañ­janaka} had gone forth as a monk. He had sex and made others have it. They told the Buddha and he said, “An {\em ubhatob­yañ­janaka} should not be given the full ordination. If it has been given, he should be expelled.”
% \end{quote}

% Neither the {\em paṇḍaka} and the {\em ubhatob­yañ­janaka} are further defined here but the word {\em ubhatob­yañ­janaka} is a compound between {\em ubhato} meaning 'in both ways, on both sides' and {\em byañjana} or {\em vyañjana} meaning 'sign or mark'.

% The rule against ordination of the {\em paṇḍaka} and the {\em ubhatob­yañ­janaka} clearly mention that full ordination of these two types of individuals, the {\em upasampadā} is not allowed. This really only makes sense if we understand {\em pabbajjā} here to be equivalent to {\em upasampadā}. In fact this equivalence between {\em pabbajjā} and {\em upasampadā} is what we find throughout the earliest Vinaya, and indeed the suttas \footnote{The {\em sāmaṇeras/īs} are barely mentioned in the suttas. Instead we find the figure of the {\em samaṇuddesa}, 'one designated as a {\em samaṇa}', who seems to have had a looser affiliation with the Sangha, that is, no proper ordination. The commentaries glosses them as {\em sāmaṇeras}, but this might be an oversimplification. More likely they were a kind of precursor to the more formal status of novice. It seems likely that such people merely put on robes, and then lived in with loose connection to a particular community of ascetics, in which case their sex would have been a non-issue. I would argue it is natural to see novices proper in the same way. But the {\em samaṇuddesa} remains obscure.}. In any case, the rules itself are clearly limited to {\em upasampadā} and novice ordination seems to be allowed.

% There are various other words mentioned in the ordination procedures for {\em Bhikkhunī} as described in {\em Bhikkhunikkhandhaka} that might be interesting in this context. These do not excluse from ordination\footnote{Khandhaka 20 Bhikkhunikkhandhaka PTS vol 2 page 271, translated by Ajahn Brahmali}: \\

% \begin{tabular}{ l l }
%  {\em itthipaṇḍaka} & female {\em paṇḍaka} \\
%  {\em animittā } & woman who lacks genitals \\
%  {\em nimittamattā } & woman with incomplete genitals \\ 
%  {\em vepurisikā } & woman who is manlike \\
% \end{tabular} \\

% The word {\em animittā} literally means 'signless' and appears a number of times in the canon (excluding commentaries) but mostly in a different meaning, namely as in {\em animitto (ceto)samādhi}, which is translated by Bhikkhu Sujato as 'signless immersion', a term used in the context of meditation. In the context of not having genitals, it only appears in the canon in the {\em Bhikkhunikkhandhaka} and as a form of abuse for women in the {\em Bhikkhu Saṃ­ghā­di­sesa­ 3}, never on it's own but always in the same sequence of words of which the above are a few.

% An interesting paragraph can be found in ­{\em Pā­rāji­ka} 1\footnote{PTS vol. 3 page 35}. Here a monk changes gender ("... the characteristics of a woman appeared ...") and is subsequently admitted as a {\em Bhikkhunī}. The same passage repeats itself for a nun who subsequently becomes a {\em Bhikkhu} The word for characteristic used here is {\em liṅga}. It even mentions that the years of seniority are transferred.


\subsection{Mahāsaṅghika Vinaya}
The Mahāsaṅghika Vinaya Bhikkhu Pakiṇṇaka describes that monks feel groping at night and after catching the culprit, a monk, he admits being a 非男非女 i.e. neither male, nor female\footnote{T22 1425 摩訶僧祇律 0417c14–0418a10}. They report to the Buddha, who tells them there are six types of un-males (不能男 者有六種) (lit. those we are not capable of producing seed/impotent). The Buddha lays down a rule that none of these should be ordained and those already ordained should be expelled.

\begin{enumerate}
\item those born impotent (生). 
\item those who are born from a concubine (捺破)\footnote{This is the only place in the canon where this is mentioned but X44 0744 0432c13 四分律名義標釋 0432c09–0433a01 mentions that there are 5 types of 黃門 (lit. yellow gate), which is translated as 'eunuch' elsewhere and 6 types of 種不能男 (i.e. seed incapable men), the 6th type being those born from a concubine}.
\item a castrated impotent man (割却), who is castrated as a punishment by the King's minister (割却 男根 lit. cut faculty of masculinity).
\item a transformed impotent man who is aroused by the touch of others but cannot ejaculate (因他)\footnote{This is a very free translation based on other texts where this type is mentioned}.
\item a jealous impotent man who is a voyeur and becomes aroused when watching others have sex (妬).
\item a 'half-moon' impotent man (半月生者) (description of what this is exactly is unclear).
\end{enumerate}

The term 非男非女 (neither male nor female) is only used by the {\em paṇḍaka} to describe himself in the this Vinaya. This could be a literal translation of the term {\em napuṃsaka} as in Vedic India this is an umbrella term of which the {\em paṇḍaka} is a subsection. The hijra of India also refer to themselves with this term.

The term 二根 (i.e. 2 roots/faculties) is mentioned in passing as a question for Bhikkhu ordination but without further explanation\footnote{T22 1425 摩訶僧祇律 0413c02}. Also the term 黃門 (translated as 'eunuch'\footnote{In the remainder of this chapter I will use the translation 'eunuch' (in quotemarks) as the official translation of 黃門 according to the dictionary. I will come back to this later as I refute this translation as too narrow and probably erroneous.}) is also mentioned here without further explanation.

Other words we find in the {\em Bhikkhunī} ordination procedure are those who have no breasts (無乳) or just one breast (一乳) or those who are barren/sterile (石女 lit. a woman made of stone). In this procedure it is mentioned that the candidate can proceed if she does not suffer from these conditions\footnote{T22 1425 摩訶僧祇律 0472b05–0472b10}. We also find the question if she is not ’two-paths’ (二道)\footnote{The term 二道 seems to be used as a synomym of 二根 i.e. two faculties. Another term used in {\em Pārājika} 2 T22 1425 摩訶僧祇律 0244a24 is 二形 ('two shapes'). There seems to be some confusion between three terms: 二根 ('two roots'), 二道 ('two paths') and 二形 ('two shapes') that are sometimes used as synonyms in different places. 二道 is at least in the Dharmaguptaka Vinaya mainly used to denote a person who has caused a schism. This is confusing because the word 道小 ('small path') in the Dharmaguptaka Vinaya is translated by \cite{bodhi} as meaning 'underdeveloped female genetalia'}

\subsection{Dharmaguptaka Vinaya}
In the Dharmaguptaka Vinaya {\em Pabbajja Khandhaka} the story is similar to that in the Theravāda Vinaya. A 'eunuch' (黃門) is ordained and then tries to have sex with monks and novices but is rebuked. He ends up having sex with cowherds and shepherds. The story is brought to the Buddha who lays down the rule that all 'eunuchs' have to be expelled and cannot ordain. He identifies five types of 'eunuchs'\footnote{translation by \cite{bodhi}. T22 1428 四分律 0812b23–0812c10}: 

\begin{enumerate}
\item those born as 'eunuch' (生黃門). 
\item a castrated 'eunuch' (犍黃門)\footnote{lit. a bullock-'eunuch'}.
\item a jealous 'eunuch' (妬黃門), who is aroused at the sight of others having sex.
\item a transformed 'eunuch' (變黃門). Transformed means while committing a sexual act with another, he loses masculine function, and thereby becomes a paṇḍaka.
\item a 'half-moon' 'eunuch' (半月黃門), having male function for half a month, and being impotent for the other half of the month\footnote{The word 不能男 (i.e. incapable/impotent) is used here just like in the Mahāsaṅghika and Sarvāstivāda Vinayas}.
\end{enumerate}

The regular list of persons not to be ordained is given, using the word 二形 ('two shapes'), translated by \cite{bodhi} as 'hermaphrodite', while in other places in the Vinaya it uses 二根. 

After this list the Dharmaguptaka Vinaya adds here the story of a monk and nun resp. who change gender as is mentioned in the Theravāda {\em Pārājika} 1. The Buddha concludes that they can simply go to the other order and do not need to be expelled\footnote{T22 1428 四分律 0813b15–0813b23. The commentary X55 0884: 表無表章栖翫記 0230c22–0231a09 explains that there is no need for re-ordination in this case.}. Again, the word used here for gender characteristics is 形 (i.e. form or shape). The next paragraphs list the case of a monk and nun resp. who changed gender to become 男女二形 i.e. both male and female. The Buddha mentions that they have to be expelled but does not say that ordination is not possible for those who are already 男女二形 before. However we can conclude this by inference.

The Dharmaguptaka Vinaya proceeds to list details of monks who have been castrated through various causes\footnote{T22 1428 四分律 0813b25–0813c04}. Obviously these are not seen as falling under the same category as the above mentioned 'eunuch'. Most of these, except for the one who self-castrates, can stay in robes; when castration happens through accident or even when it happens through karmic causes, the monk in question can remain, if he causes the castration intentionally himself he is expelled. Here the phrase is 截其 男根 (lit. cut off the male root).

While in the Mahāsaṅghika Vinaya the castration (i.e. cutting off of the male faculty 男根) is seen as an impotent man and thus not fit for ordination, here this only matters when the action is voluntary and not accidental.

In the {\em Bhikkhunī} ordination procedure we find the two-faculties (二根) person as well and in the same sequence we find the word 道小, which is translated by \cite{bodhi} as 'underdeveloped genitalia'\footnote{T22 1428 四分律 0924c20} but the literally spells ('small path'). Unlike in the Theravāda Vinaya, this condition would lead to disqualification for ordination. Further down a separate clause is added for those who have no breasts (無乳) or just one breast (一乳), who are equally barred from ordination\footnote{T22 1428 四分律 0926c20–0926c21}.


\subsection{Mahīśāsaka Vinaya}
The story in the Mahīśāsaka Vinaya {\em Pabbajjā Khandhaka}\footnote{T22 1421 彌沙塞部和醯五分律 0117c29–0118a05} is similar to the Theravāda Vinaya. A {\em paṇḍaka} (黃門) is ordained and proceeds to try and have sex with various monks, novices and others. As a result that he is expelled together with others like him. Just like in the Theravāda Vinaya, there is no mention here of several types of {\em paṇḍaka}. At the end of the expulsion spoken by the Buddha, it is simply mentioned that the same holds true for 'two roots/faculties' (二根) without further explanation of what this is.

The story of the monk who became a woman and was allowed to live with the nuns thereafter is also mentioned here and also the opposite case of a nun who became a man. The next paragraph is dedicated to a monk who, due to his great lust, self-castrated and as a result is expelled\footnote{T22 1421 彌沙塞部和醯五分律 0119a11–0119a28. Unlike in the Dharmaguptaka Vinaya, the character for 根 (root or faculty) is used here for the monk/nun who change gender while the word 形 (shape or form) is used for the monk who castrates himself}. 

In the {\em Bhikkhunī} ordination procedure we another list in the questions asked during the ritual\footnote{T22 1421 彌沙塞部和醯五分律 0187c21–0187c29}. It asks if a woman is barren/sterile (石女), it also asks if she is not a {\em paṇḍaka} (黃門) and if the female genitals (faculties) are developed (女根具足). Here it is not specifically mentioned that somebody is barred from ordination if the answer is affermative.

\subsection{Sarvāstivāda Vinaya}
The story in the Sarvāstivāda Vinaya {\em Pabbajjā Khandhaka}\footnote{T23 1435 0153b18–0153c17} also tells of a monk who groped other monks at night which gave problems and started rumours. Again, the Buddha identifies five types 種不能男 (impotent males). All these are not allowed to ordain and are expelled if already ordained.

\begin{enumerate}
\item those born impotent (生). (here possibly defined as a bastard)
\item a 'half-moon' impotent man (半月), who is impotent for half of the month.
\item a jealous impotent man (妬), who likes to see others engage in sex.
\item an 'essential'(?) impotent man (精), who causes others to have sex?
\item a ill impotent man who became impotent through illness (?) (病).
\end{enumerate}

In another part of the Vinaya this term 二根 (two roots/faculties) is used next to the term 黃門 ('eunuch') but not in relation to ordination. {\em Pārājika} 1 (just like the {\em Pārājika} 1 of all the schools) mentions the existence of 4 kinds of offenders, men, women, 黃門 ('eunuch') and 二根 (2 roots/faculties). The same two words are used elsewhere in the Sarvāstivāda Vinaya while the word 種不能男 (impotent) is only used in the list for those who cannot ordain.

The {\em Bhikkhunī Khandhaka} goes more into detail about those who cannot ordain. The 二根 (two roots/faculties) is mentioned here\footnote{T23 1435 0294a23–0294a28}. 

A similar list of questions is asked of female ordination candidates for ordination as with the other schools. Amongst these are the question if the candidate has underdeveloped genitalia (女根小) (lit. small female root), has no breasts (無乳) or just one breast (一乳) and if she is sterile (是不能產). It seems however that regardless of the answer, the candidate is not barred from ordination\footnote{T23 1435 0332b11–0332b22}.


% \subsection{Commentaries}
% Although probably more material can be found in the Chinese commentarial texts, I have limited myself here to what I deem to be the most important aspects that throw some additional light on the possible meanings of the terms in questions.


With regards to the five types of 黃門, the Chinese commentarial texts merely add that these cannot ordain because they have difficulty keeping the precepts\footnote{T85 2792 毘尼心 0667b25–0667b26}.

% For the {\em ubhatob­yañ­janaka} we find the following in the {\em Samantapāsādikā}\footnote{{\em Samantapādādikā}, vol. 3, para. 116 translation by Ajahn Brahmali}:

% \begin{quote}
% Because of kamma giving rise to female characteristics and kamma giving rise to male characteristics, there is for them the characteristics of both. With the male characteristic they act to transgress through sexual intercourse with women. Having encouraged another, they cause action in their own female characteristic. 

% They are twofold: the female {\em ubhatob­yañ­janaka} and the male {\em ubhatob­yañ­janaka}. In regard to this, the female characteristic of the female {\em ubhatob­yañ­janaka} is apparent, but the male characteristic is hidden. The male characteristic of the male {\em ubhatob­yañ­janaka} is apparent, but the female characteristic is hidden. 

% While the female {\em ubhatob­yañ­janaka} is acting with manliness among women, the female characteristic is hidden, whereas the male characteristic is apparent. 
% When the male {\em ubhatob­yañ­janaka} enters the state of a woman for the sake of men, the male characteristic is hidden, whereas the female characteristic is apparent. 
% The female {\em ubhatob­yañ­janaka} becomes pregnant and causes others to become pregnant. The male {\em ubhatob­yañ­janaka} does not become pregnant, but causes others to become pregnant. This is the difference between them.
% \end{quote}

% The Chinese equivalent of the Pali {\em Samantapāsādikā} can be found in T24 1462: 善見律毘婆沙\footnote{T24 1462 善見律毘婆沙 0792c03–0792c06. 5th century CE}:
% \begin{quote}
% There are three kinds of two-facultied people (二根): those who can impregnate and conceive; those who can impregnate but not conceive; and those who cannot impregnate but who can conceive. These three types of people are not allowed to become monks and take the full precepts; if they have already taken the full precepts, they should be expelled.
% \end{quote}

% Other Chinese commentaries have variations of the same passage\footnote{See f.i. Shinsan X44 0744 四分律名義標釋 0450b01–0450b04}:
% \begin{quote}
% It is said that a person has two roots/faculties (二根): male and female. There are three kinds: The first is able to self-reproduce. He can impregnate and conceive. The second can impregnate others but cannot conceive himself. The third type cannot impregnate but he can conceive when impregnated by another. 
% \end{quote}

% The Theravāda commentary, both in regards to the {\em paṇḍaka} and the {\em ubhatob­yañ­janaka} differs from the Vinaya in making a distinction between {\em pabbajjā} (novice ordination) and {\em upasampadā} (full ordination) and does not allow either for ordination.

% The three words  {\em animittā}, {\em nimittamattā} and {\em vepurisikā} do not appear in any of the earlier commentarial texts but appear again in the {\em Tikā Vajirabuddhi / Cūḷavaggavaṇṇanā} without further explanation.

% The section on the monk changing gender is discussed in the {\em Samantapāsādikā} and its Chinese equivalent (T24 1462 善見律毘婆沙). Here the change of {\em liṅga} is described as appearing suddenly in the middle of the night; one goes to bed as a man and wakes up as a woman (or visa versa). The commentary also attributes such a change to good or bad kamma. Carol Anderson provides a full English translation of this commentarial passage\footnote{See \cite{anderson2016} page 237–242}. This passage also describes that a change back is possible and that it is possible to change back and forth several times. This passage is especially striking for its empathic tone and the recognition that such a change might be rather overwhelming for the unsuspecting monastic and the need to provide this person with the care they need.

