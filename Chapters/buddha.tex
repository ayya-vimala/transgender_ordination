\section{What would the Buddha say?}


Yathā etāsu jātīsu,
Liṅgaṃ jātimayaṃ puthu;
Evaṃ natthi manussesu,
Liṅgaṃ jātimayaṃ puthu.

Na kesehi na sīsena,
Na kaṇṇehi na akkhibhi;
Na mukhena na nāsāya,
Na oṭṭhehi bhamūhi vā.

Na gīvāya na aṃsehi,
Na udarena na piṭṭhiyā;
Na soṇiyā na urasā,
Na sambādhe na methune .

Na hatthehi na pādehi,
Nāṅgulīhi nakhehi vā;
Na jaṅghāhi na ūrūhi,
Na vaṇṇena sarena vā;
Liṅgaṃ jātimayaṃ neva,
Yathā aññāsu jātisu.

Paccattañca sarīresu ,
Manussesvetaṃ na vijjati;
Vokārañca manussesu,
Samaññāya pavuccati.




While in those births are differences,
each having their own distinctive marks,
among humanity such differences
of species—no such marks are found.

Neither in hair, nor in the head,
not in the ears or eyes,
neither found in mouth or nose,
not in lips or brows.

Neither in neck, nor shoulders found,
not in belly or the back,
neither in buttocks nor the breast,
not in groin or sexual parts.

Neither in hands nor in the feet,
not in fingers or the nails,
neither in knees nor in the thighs,
not in their “colour”, not in sound,
here is no distinctive mark
as in the many other sorts of birth.

In human bodies as they are,
such differences cannot be found:
the only human differences
are those in names alone.

vāseṭṭha sutta
PTS verse 607-611
snp 3.9 or mn98