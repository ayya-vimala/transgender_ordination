\section{Introduction}
Transgender and intersex people and at times other LGBTIQA+ have been excluded from ordination as a Buddhist monastic in the Theravāda tradition. This exclusion is the result of what I will show is an erroneous reading of several Pali terms--{\em paṇḍaka} and {\em ubhatob­yañ­janaka}--in the monastic disciplinary code (Vinaya Piṭaka) of the Theravāda school. Rendering the terms {\em paṇḍaka} and {\em ubhatob­yañ­janaka} into English, previous lexicographers of the Pali language have used vocabulary rooted in the (Christian) understanding of the early 20\textsuperscript{th} Century, like `eunuch' and `hermaphrodite'.\footnote{The Pali Text Society's Pali English Dictionary and Cone's Concise Pali English Dictionary. For dictionary entries I refer to \href{https://suttacentral.net/}{SuttaCentral.net}.} It has previously been noted that it is problematic to transpose modern terms in the understanding and translation of other (religious) movements.\footnote{\cite{maes} page 2, \cite{dudas} page 45, \cite{artinger} pages 1–4} In dealing with the concepts of {\em paṇḍaka} and {\em ubhatob­yañ­janaka} the terms `eunuch' and `hermaphrodite', but also terms like `transgender' and `intersex' are inappropriate as they wrongly suggest that the lived understanding of the relationship between sex, gender and sexuality in Ancient India was the same as it is for us in the West today. The fact that certain groups of people are unable to obtain monastic ordination based on terms that are so little understood creates a barrier for all LGBTIQA+ people who come to Buddhism seeking refuge from suffering.

When studying the Buddhist scriptures, especially where there are groups of people who are marginalized, it is important to understand where and under which circumstances these concepts and interpretations have originated.\footnote{Brenna \cite{artinger} pages 3–4 points to the use of Michael Foucault's methodology of `geneology': {\em ``Thus, this methodology is primarily concerned with the ways in which ideas are crafted through the shaping of cultural and political influence, and one's ability to trace such lineages of formation. Through this process, one is able to see not only how or why ideas are created, but also how specific terms are negotiated through paradigms of power to bear certain connotations and interpretations.''}}

In the beginning of the Buddhist Order there were no rules for the conduct of monastics. The Vinaya was laid down later and grew as more rules were established. These were implemented only when monks started to misbehave and guidelines became necessary.\footnote{See Bhikkhu \cite{sujato2009} pages 8–10 for a more details on the context of the Vinaya.} The Vinaya as we have it today was formed over a long period of time and has been highly redacted over the centuries, regulating many and diverse aspects of monastic life. It is not an original Buddhist text that was passed down unchanged since the time of the Buddha. The oldest parts of the Vinaya consist of the rules ({\em pāṭimokkha}) and procedures ({\em kammavācā}), possibly together with some other materials. The different Vinayas in existence today are the products of the various schools of Buddhism that emerged much later.\footnote{After the Buddha passed away we see a gradual emergence of schools in the Aśokan and post-Aśokan periods. See Bhikkhu \cite{sujato2012} for a detailed study on the emergence of the Buddhist schools.} The Second Council is of preeminent importance in the development of the Vinayas as this is the only major event in Buddhist history that revolves entirely around a Vinaya dispute. Bhikkhu Sujato\footnote{See \cite{sujato2009} pages 141–142 and 215–216.} suggests that the bulk of the Vinaya texts were added well after the Buddha's death, in contrast to the Suttas: 

\begin{quote}
But the Vinayas were, it seems, composed following the Second Council; and in particular the {\em Khandhakas}, with their massive narrative arc, were put together in order to authenticate the acts of the Second Council.
\end{quote}

This is important for our current topic because the rules regarding the exclusion from ordination of the {\em paṇḍaka} and {\em ubhatob­yañ­janaka} are found in the {\em Khandhakas} and the words do not appear in the early Suttas, nor in the oldest parts of the Vinaya. The question remains as to why these rules were added. The Buddhist community also evolved in constant negotiation with its wider religious environment and needs to be understood as a dialogue with its various `religious others'. Claire Maes\footnote{See \cite{maes} and \cite{maes2016}.} has clearly demonstrated that this process was central to the formation of the Vinaya as an ongoing dynamic to create a distinctive Buddhist identity. The Second Buddhist Council became an important event in the history of Buddhism to determine its identity vis-à-vis other religious Orders after the Buddha passed away.

Many scholars have pointed out the many similarities between the principal practices, precepts and structures of Buddhists, Jains and Brahman communities and they seem to have had a detailed knowledge of each other's practices and organization.\footnote{See \cite{maes2016} page 9 footnotes 26–28.} The interaction and debates with these `religious others' led the Buddhist {\em Saṅgha} to implement specific rules in order to be in conformity with certain well-established monastic customs on the one hand and to (re)define their identity as a clearly separate Order on the other. In this paper I will argue that the concepts {\em paṇḍaka} and {\em ubhatob­yañ­janaka} have entered the Buddhist Vinaya after the Buddha passed away in the context of a much wider religious discussion that took place regarding the position of women within religious life that has also reduced the opportunities for women to ordain as Buddhist monastics.\footnote{The legality of the ordination of women ({\em Bhikkhunīs}) in the Theravāda and Tibetan lineages of Buddhism has been a hotly debated topic for many years. Thanks to the efforts and research of many monastics and academics, the first full Theravāda ordination was held in Perth in October 2009. Although still not widely recognized in several traditional Theravāda countries, recognition is growing and the number of Bhikkhunīs is slowly increasing. See a.o. Bhikkhu \cite{sujato2009} and Bhikkhu \cite{analayo2013} for research in this field.}

I will also show that these terms have their roots in Vedic mythology and provide a fresh insight into the ancient Asian paradigms for gender identities. Here we find the living proof of evolving ideas on sex, sexuality and gender that are very different from our Western concepts. And here we find that these terms are intimately bound up with the deeply ambivalent attitude towards women and women's sexuality in ancient India.

In this paper I will first trace the emergence of these--and other gender-specific--terms in Vedic, Brahmanic and Jain scriptures and their changes over the centuries. I will then discuss the occurrences of these terms in the Pali and Chinese Vinayas and compare these with the understanding of the contemporary `religious others' to come to a deeper understanding of what the terms {\em paṇḍaka} and {\em ubhatob­yañ­janaka} really meant at the time these passages were written and the reasons why these are said to be barred from ordination. Finally I will show that neither these terms, nor any other regulations in the Vinaya, can be used as a justification to exclude candidates from ordination based on their sex, sexuality or gender.
