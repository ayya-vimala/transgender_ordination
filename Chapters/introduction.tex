\section{Introduction}
The legality of Bhikkhunī ordination in the Theravāda and Tibetan lineages of Buddhism has been a hotly debated topic for many years. Thanks to the efforts and research of many monastics and academics, the first full Theravada ordination was held in Perth in October 2009. Although still not widely recognized in several traditional Theravāda countries, recognition is growing and the number of Bhikkhunīs is slowly increasing. In this article I will not go into detail about the discussion with regards to the ordination of women in Buddhist circles after the passing away of the Buddha because other authors have already done excellent research on this\footnote{See \cite{sujato2009} and \cite{analayo2013}}, but this discussion also pertains to the subject matter at hand.

Next to women, there are other groups of people that have been marginalized and excluded from ordination, namely transgenders and intersex people and at times other queer people. This exclusion is the result of what I will show is an erroneous reading of several terms in the Theravada Vinaya: {\em paṇḍaka} and {\em ubhatob­yañ­janaka}. Rendering the terms {\em paṇḍaka} and {\em ubhatob­yañ­janaka} into English, previous lexicographers of the Pali language have used vocabulary rooted in the Christian understanding of the early 20th century, like 'eunuch' and 'hermaphrodite'\footnote{The Pali Text Society's Pali English Dictionary and Cone's Concise Pali English Dictionary. For dictionary entries I refer to \href{https://suttacentral.net/}{SuttaCentral.net}}. It has previously been noted that it is problematic to transpose Christian modern terms in the understanding and translation of other religious movements\footnote{\cite{maes} page 2, \cite{dudas} page 45}. In dealing with the concepts of {\em paṇḍaka} and {\em ubhatob­yañ­janaka} the terms 'eunuch' and 'hermaphrodite' are inappropriate as they wrongly suggest that the lived understanding of sex, gender and sexuality in Ancient India is the same as it for us in the West today. The fact that certain groups of people are unable to obtain monastic ordination based on terms that are so little understood creates a barrier to all queer people who come to Buddhism seeking refuge from suffering.

When studying the Buddhist scriptures, especially where there are groups of people who are marginalized, it is important to understand where and under which circumstances these concepts and interpretations have originated. The Buddhist community evolved in constant negotiation with its wider religious environment and needs to be understood as a dialogue with its various 'religious others', the processes of which and how it influenced the formation of the Vinaya have been clearly demonstrated by Claire Maes\footnote{See \cite{maes} and \cite{maes2016}}. This process was central to the formation of the Vinaya as an ongoing dynamic to create a Buddhist identity notion. Many scholars have pointed out the many similarities between the principal ascetic practices, precepts and structures of Buddhists, Jains and Brahmanic communities and they seem to have had a detailed knowledge of each other's practices and organization\footnote{See \cite{maes2016} page 9 footnotes 26–28}. The interaction and debates with these 'religious others' lead the Buddhist Sangha to implement specific rules in order to be in conformity with certain well-established ascetic customs on the one hand and to (re)define their identity as a clarly separte order on the other. In this paper I will argue that the concepts {\em paṇḍaka} and {\em ubhatob­yañ­janaka} have entered the Buddhist Vinaya after the Buddha passed away in the context of a much wider religious discussion that took place regarding the position of women within religious life. 

I will also show that these terms have their roots in Vedic mythology\footnote{Note that in this work I have deviated from some of the earlier points I made in \cite{vimala} with regards to the Vedic concept of the third sex. I have now rejected certain sources on the basis that I found them unreliable upon closer inspection and I hope to have rectified this with more thorough research.} and provide a fresh insight into the Asian paradigms for non-binary gender identities. Here we find the living proof of evolving ideas on gender that are very different from our western concepts of trans-sexuality, intersex, etc. And here we find that these terms are intimately bound up with the deeply ambivalent attitude towards women and women's sexuality in ancient India.

Thirdly, I will show that the editors of the Vinayas of the various Buddhist schools struggled themselves with the understanding of these terms based on their own culture and lexicon.

In this paper I will first trace the emergence of these--and other gender-specific--terms in Vedic, Brahmanic and Jain scriptures and their changes over the centuries. I will then discuss the occurances of these terms in the Pali and Chinese Vinayas and compare this with the understanding of the contemporary 'religious others' to come to an understanding of what the terms {\em paṇḍaka} and {\em ubhatob­yañ­janaka} really meant at the time these passages were written and the reasons why these are said to be barred from ordination. Finally I will show that neither these terms, nor any other regulations in the Vinaya, can be used as a justification to barr candidates from ordination based on their sex, sexuality or gender.\\\\

Ven. Vimala Bhikkhunī\\
Tilorien Monastery\\
December 2020\\


