\section{Introduction}

This section section just some examples of various types of Latex things that can be used.

\begin{enumerate}
 \item Adding more texts
 \item Adding cross-language parallels
 \item Further development of the website UI
\end{enumerate}

\newpage

\rotatebox{90}{\includegraphics[width=\textheight,height=\textwidth,keepaspectratio]{timeline.pdf}}

\includegraphics[width=\linewidth]{textview1.png}
\captionof{figure}{The website displaying T1600 (辯中邊論) as the main text in text view mode.}
\label{textview1}

\medskip
Figure \ref{textview1}

\footnote{this solution, rather by coincidence, is similar to the solution presented in (\cite{sturgeon2017}), where different shades of red were used to indicate layers of quotations. However, unlike (\cite{sturgeon2017}) in this study the differences in wordings between two passages are not visually highlighted. This is due to the fact that the data also contains examples of less verbatim quo-
tations, where such a mechanical highlighting strategy becomes difficult to apply.}



\begin{tabular}{ c c c }
 \textbf{Description} & \textbf{Monthly costs (USD)} & \textbf{Annual costs (USD)} \\ 
 Part-time developer & 3.000 & 36.000 \\  
 \href{https://www.linode.com/pricing}{VPS Linode 4G plan} & 20 & 240 \\
  \href{https://www.linode.com/blockstorage}{500 GB storage} & 50 & 600 \\
  3x Nvidia RTX 2080 TI GPU & - & 3600 \\
  TPU training Google cloud (opt) & - & 1500 \\
 \textbf{Total costs} & & \textbf{41.940}
\end{tabular}



-----------------------------------------------------

\subsection{Parallels research}
Since the 19th century it has been recognized by scholars that many Buddhist texts have counterparts (parallels) in other collections of the various schools, often preserved in different languages. The first documentation of these parallels was published by Nanjio Bunyiu (Nanjō Bun'yū, 南條文雄) in his {\em A Catalogue of the Chinese Translation of the Buddhist Tripiṭaka} of 1883. Nanjio listed 24 {\em Dīgha Nikāya} discourses as parallels to the {\em Dīrghāgama} in Chinese. A year later, Samuel \cite{beal} published translations of the Pāli {\em pātimokkha} and the Chinese Dharmaguptaka {\em prātimokṣa} and concluded that these are virtually identical.

This marked the start of the work on comparative studies by many great scholars. A work that has been ongoing until the present day. This work not only had a great influence on how we understand the Buddhist scriptures in the academic field, but also on the practical applications of the Buddha's teachings and thus on the daily lives of many people.

Parallels research can contribute to a.o. dating of scriptures, author-identification of texts, finding the origins and tracing the development of certain key concepts and many mor tasks of the vast field of textual Buddhist Studies.

\subsection{Practical applications}
In his Comparative Study of the {\em Majjhima Nikāya}, Bhikkhu \cite{analayo2011} has shown that all significant aspects of early Buddhist doctrine are the same across all extant textual transmission of the Suttas of the Pāli {\em Majjhima Nikāya}.

His work has had a great impact on the Buddhist world. Through it it became possible to distinguish the Early Buddhist Texts from later Buddhist literature, and therewith deepen our understanding of Buddhist practice (see {\em The Authenticity of Early Buddhist Texts} by \cite{sujatobrahmali}). 

It also had a great impact on the position of women within Buddhist communities. As in many parts of the Buddhist world, the full ordination of women and LGBTIQ was not possible before. His work on comparative research has made an immense contribution to facilitate the Bhikkhunī ordination in the Theravada tradition. (see \cite{analayo2018},\cite{analayo2018-2},\cite{analayo2013})

Also the influence of Jain and Vedic sources on the Buddhist scriptures is a very young field of study within the field of Buddhist Studies, and proper research on parallels between texts preserved in these different languges can have far-reaching influences on how Buddhism is practiced even today (\cite{maes}, \cite{vimala}). 

\subsection{Digital Humanities}
For the research of Buddhist textual material, citations and similar passages are of major importance. Finding these parallels is however not trivial. Comparative studies have sofar relied solely on manual comparison of Buddhist texts for the identification of parallels. For this it was necessary for scholars in the past not only to be familiar with the various languages in which these texts are written, but also to have a precise knowledge of the content and vocabular of a larg number of texts. Considering the huge amount of Buddhist textual material of the various schools in existence, it is close to impossible for a human being to detect all possible parallel passages. Digital approaches however by nature seem to be the appropiate way to tackle the task of finding similar and closely related passages reliably within a large number of digitalized texts.\\

Important parallels are not always literally identical chunks of texts, but smaller things such as word order, used vocabulary, spelling or grammatical structures can vary. Recently, neural networks have become the tool of choice for the detection of such approximate parallels, since they can be trained to extract the semantic features of tokens and do not need to rely solely on chunks of identical strings as was he case with earlier algorithms for parallel detection. 
It therefore rather recently has become possible to compare large volumes of data with a much higher degree of accuracy than any human would be able to do by manually comparing texts.

Donald \cite{sturgeon2017} points out in his article {\em 'Unsupervised identification of text reuse in early Chinese literature'}: 

\begin{quote}
A further observation emerging from this study is that a key advantage of digital systems over traditional printed forms of research lies in the possibility of allowing scholars to work with all data of a particular kind, rather than a useful and important subset selected by experts. With the sheer volume of parallels spread throughout the early Chinese corpus, even the most diligent of conventional studies risks omitting information that may prove relevant to the particular research questions that someone else may be interested in.
\end{quote}

Relevant attempts to calculate parallel passages for Buddhist and related material started in 2010 (\cite{prasad2010}) and are ongoing since. 

In 2016, Dr. Orna Almogi (University of Hamburg) a.o. (\cite{almogi}) organized a hackathon involving 17 scholars, scientists and students to develop and compare algorithms for finding parallels within the Tibetan classical corpus of texts.

Various algorithms and methods have been tried and tested subsequently (\cite{klein2014}). The BuddhaNexus is currently using algorithms that are based on the experience gained in these previous studies  while also consantly testing out and developing further new approaches.  

\subsection{The Buddhanexus Project}
The Buddhanexus project was started by Sebastian Nehrdich and Dr. Orna Almogi to create a tool to calculate parallel passages using a neural network, especially with an focus on extracting inexact matches. A detailed description of the neural network and Sebastian's work is given in the section on Methodology. 

The output of this neural network is a vast amount of data that needs to be structured and filtered in order to be used in a meaningful way. Ven. Vimala Bhikkhunī, former frontend programmer at SuttaCentral.net, became involved in the project and started to develop a web interface to display the neural network's output data, initially by just displaying the data in convenient tables. 

This was further augmented by the work of Prof. Kiyonori Nagasaki, who created a sankey graph with the Tibetan data and eventually a full text display of each text including a 'heat-map' to display the sections with the highest to the lowest potential parallels was added. 

The site was further developed with the help and feedback of various professionals in the field of parallels study such as Prof. Michael Radich (University of Heidelberg).

In order to manage the large amounts of data, a former programmer from SuttaCentral.net and indologist was hired to build a backend database for use with the site and speed up the loading times. A more detailed description of the website is given in the section on Website UI.

With this, Buddhanexus is now turning into a very powerful tool for the study of Buddhist texts and for advanced parallels-research which has never been seen before. However, at present Buddhanexus is only able to find potential parallels within texts of the same root language. In the future, we want to train more powerful neural models which are able to deal with the tasks of finding potential parallels between Pāli, Sanskrit and Tibetan and later on also between these and Chinese. These can be further supported by using existing dictionaries and known, verified, parallels and translations as training material. This is further described in the section on Future Vision as well as a roadmap for the development in the next year.

Cross-language comparisons will involve far larger amounts of data and much higher requirements for the GPU hardware during training and inteference. Also, a more stable platform for running the website will become necessary, for which financial backup will need to be sought. All this is described in the section on Financial Planning.
