\section{Introduction}
The legality of Bhikkhunī ordination in the Theravāda and Tibetan lineages of Buddhism has been a hotly debated issue for many years. Thanks to the efforts and research of many monastics and academics, the first full Theravada ordination was held in Perth in October 2010. Although still not widely recognized in several traditional Theravāda countries, recognition is growing and the number of Bhikkhunīs is slowly increasing. In this article I will not go into detail about the discussion with regards to the ordination of women in Buddhist circles after the passing away of the Buddha because other authors have already done excellent research on this\footnote{See \cite{sujato2009} and \cite{analayo2013}}, but this discussion also pertains to the subject matter at hand.

Next to women, there are other groups of people that have been marginalized and excluded from ordination, groups which we will refer to here with the Pāli terms used in the Theravada Vinaya: {\em paṇḍaka} and {\em ubhatob­yañ­janaka}. There have been various translations and interpretations of these terms with the consequence that intersex people and transgenders and sometimes others have been refused ordination. But there is also much ambiguity as to what these terms really mean and how they ended up in the Vinaya section barring them from ordination.

When studying the Buddhist scriptures, especially where there are groups of people who seem to be marginalized, it is important to understand where and under which circumstances these concepts and interpretations have originated. The Early Buddhist Texts mainly focus on the teachings themselves, and much less on the socio-cultural environment in which these originated. In fact, they seem to deal with sex and gender as a given, that need no further discussion. So we have to look elsewhere for more information on this topic, like the pre-Buddhist Vedic culture\footnote{Note that in this work I have deviated from some of the earlier points I made in \cite{vimala} with regards to the Vedic concept of the third sex. I have now rejected certain sources on the basis that I found them unreliable upon closer inspection and I hope to have rectified this with more thorough research.} and the Brahmanic and Jain cultures at the time of the Buddha and thereafter. No study of these terms would be complete without an understanding of what these words would have meant to the people that lived in the times that these words were used and so we also delve into the rich tapestry of Indian  culture and society. Here we find the living proof of evolving ideas on gender that are very different from our western concepts of trans-sexuality, intersex, etc. And here we find that these terms are intimately bound up with the deeply ambivalent attitude towards women and women's sexuality in ancient India.

In this article I will trace the emergence of these terms in Vedic, Brahmanical and Jain scriptures and their changes over the centuries and compare that with what we know from Buddhist texts to come to a better understanding of what the terms {\em paṇḍaka} and {\em ubhatob­yañ­janaka} really meant at the time these passages were noted down and the reasons why these are said to be barred from ordination.

I hope that this article will pave the way for ordination of all people as Buddhist monastics, regardless of sex, sexuality and gender.\\\\
Ven. Vimala Bhikkhunī\\
Tilorien Monastery\\
December 2020\\