\section{Liṅga and the Dispute about Women's Ordination}
\label{linga}

As we have touched upon in the context of the developments in the Jain Order there was a lively discussion around the 3\textsuperscript{rd} Century BCE on what the characteristics of sex or gender (\textit{liṅga}) are. This controversy hinged on the identification of the signs to designate somebody as a woman, which logically also led to the examination of what is male, and `neither male nor female'. This was a much wider religious debate that involved both the Jain and Buddhist communities, and possibly other religious traditions as well. Bhikkhu Sujato\footnote{\cite{sujato2009} pages 241–242.} points out that these discussions were conducted by the monks, while the women themselves were notable by their absence. We have no indication that this was any different in the Jain Order. It is therefore not surprising that at the Second Buddhist Council, which was held at around the same time, rules were laid down with regards to the ordination of nuns, and most likely the `third sex' category, with its perceived hyperlibidinous-ness, was also touched upon. We also find for instance various passages with regards to women's roles in the Order that seem out of place in the Buddhist scriptures but which appear with near identical wording in the Jain texts\footnote{\cite{sujato2009} pages 54–55.} so it is clear that these two Orders influenced each other.

This discussion regarding \textit{liṅga} in the light of ordination into the Buddhist monastic Order is all the more remarkable because monks and nuns forego the usual markers of sex and gender when they put on robes and shave their heads. Giving up gendered attire is one of the distinguishing characteristics of monastic life. In the Jain Order this discussion made more sense.\footnote{\cite{maes2016} pages 11–17 points out that the Jains' nakedness was one of the primary distinctive marks that set them aside from Buddhist monastics. The wearing of a bowl and robe was an important part of the Buddhist identity and is also referred to with the term \textit{liṅga}.} The Jains were naked ascetics and therefore the physical marks of sex could not so easily be given up. The female monastics needed to wear a cloth to cover their bodies while the males could go naked. This was a very important point for the Jains because this difference also meant that as females could not let go of all earthly possessions they could also not reach enlightenment. This was one of the main points of dispute between what became the two sects in Jainism.

The original meaning of the term \textit{liṅga} before this dispute was `characteristic mark or sign',\footnote{\textit{Nirukta} 1.17.} but during this dispute the term was refined and the different Orders developed different opinions about the characteristics of `male' and `female'. Bee Scherer\footnote{\cite{scherer} page 68.} takes the term \textit{liṅga} as a reference to the `secondary sex organs' and other characteristics of gender difference, which also include behavioral differences so the term can be used to denote both biological sex and some aspects of the expression of gender-identity as we define it today. They base this conclusion on the work of Buddhaghosa,\footnote{Buddhaghosa lived in the 5\textsuperscript{th} Century CE. He was a commentator, translator and philosopher. He worked in the Great Monastery (Mahāvihāra) at Anurādhapura, Sri Lanka. He is also the main author of the \textit{Samantapāsādikā} commentary as well as many other works. He is generally recognized as the most important philosopher and commentator of the \textit{Theravāda} Canon. His best-known work is the \textit{Visuddhimagga} (`Path of Purification'), a comprehensive summary of older Sinhala commentaries on \textit{Theravāda} teachings and practices.} who listed the secondary characteristics of the male and female, which included beards and mustaches, motherly instincts, way of walking, etc. The fact that next to male and female characteristics we also have the term \textit{napuṁsakaliṅga} makes it indeed very unlikely that this would denote primary sex characteristics only. The term is also used in for instance \textit{bhikkhuliṅga}, \textit{gihiliṅga} and \textit{hatthiliṅga}, which are defined as the distinctive marks or character of a monk, householder and elephant respectively.

The Brahmanical \textit{Mahābhāṣya}\footnote{\textit{Mahābhāṣya} 4.1.3. ``What is it that people see when they decide, this is a woman, this is a man, this is neither woman nor man?''} makes it clear that the term \textit{liṅga} refers to the sex/gender of a person as perceived by those who behold them and that this can refer to sexual characteristics, dress, hair and behavior. This view is confirmed in the Buddhist \textit{Abhidharmakośa},\footnote{\textit{Abhidharmakośa} IV.14 c. using the word \textit{vyañjana} as a synonym for \textit{liṅga}.} and the \textit{Samantapāsādikā} commentary\footnote{See \cite{anderson2016} pages 237–240 for an English translation of the passage on change of \textit{liṅga} in a monastic. The \textit{Samantapāsādikā} is a translation of Sinhala commentaries into Pali by Bhikkhu Buddhaghosa and possibly others in the 5\textsuperscript{th} Century CE. It was based on the \textit{Mahāpaccariya} and the \textit{Kurundī Atthakathā}. See \cite{goonesekere} for details on \textit{Theravāda} commentaries.} and elsewhere. There is no indication that the Buddhists agreed with the Jains on the inclusion of underlying sexuality and sexual feelings in the definition of \textit{liṅga},\footnote{The Jains also defined the sexual desire for the male body as part of the female characteristics and vice versa.} nor that our modern definition of gender as a personal identification based on an internal awareness was part of it.

There are several other words that are used as synonyms for \textit{liṅga} in the Pali and Sanskrit texts. These are \textit{nimitta} and \textit{byañjana} (Skt. \textit{vyañjana}). Both of these words have a much more general meaning of `sign' or `mark' but are also used in relation to sex and gender in various places. In Appendix \ref{appendix2}, Section \ref{appendix2b}, I have charted the occurrences of these words throughout the Pali and Sanskrit texts. I have done the same for their Chinese equivalents `root/faculty' (根) and `shape' (形). In order to get an idea as to the frequency in which these words are used in relation to sex/gender characteristics as opposed to general characteristics I have used the prefixes \textit{itthi} (`female' Skt. \textit{strī}, Chn. 女) and \textit{purisa} (`male' Skt. \textit{puruṣa}, Chn. 男). This is not an exact method and we cannot draw definite conclusions from it but it gives an idea of the way in which these words are used in the texts.

In the Pali texts the words \textit{itthiliṅga} and \textit{purisaliṅga} are the only ones used in the \textit{Suttas} and \textit{Vinaya} to denote sex/gender characteristics. There is one occurance of the word \textit{purisabyañjana} in the \textit{Bhikkhunikkhandhaka}, but this is very likely a later addition.\footnote{See chapter \ref{itthipandaka}. The fact that this word is further only used in the later commentaries seems to confirm its lateness.}  The words \textit{itthinimitta} and \textit{purisanimitta} appear in the \textit{Abhidhamma} and all six words are used in the commentaries. It seems therefore likely that the use of these alternative words to denote sex/gender characteristics is a later development. Especially the word for `female characteristics' \textit{itthiliṅga} becomes very prominent over time. We also see that in the \textit{Abhidhamma} the word \textit{liṅga} is used exclusively to denote sex-characteristics. In the \textit{Vinaya} the word \textit{byañjana} is used mostly to denote sex/gender characteristics but also appears a few times in the meaning of `curry'. In the other collections both these words are used relatively little to denote sex/gender characteristics. 

In the Sanskrit texts we see a very different picture. Again, in the \textit{Vinayas} only the words \textit{strīliṅga} and \textit{puruṣaliṅga} are used to denote sex/gender characteristics, but for the other Buddhist texts the result is rather garbled which is especially shown in the chart of the relative number of words (see Appendix \ref{appendix2b}, Figure \ref{stri}). This is mostly due to the fact that the Sanskrit texts, unlike the Pali, are not shown in order of approximate lateness and the collections from which the data are drawn are not complete. In all the collections both words \textit{liṅga} and \textit{vyañjana} are used relatively little to denote sex/gender characteristics. The most interesting feature here is that the words denoting `female characteristics' are used far more than the words for `male characteristics' especially in the Vedic/Brahmanical texts. Just like in the Pali we see a growing prominence of the word \textit{strīliṅga}, which becomes most important in the later \textit{Śāstra} collections. This could point to an increasing interest over time as to the question of what the characteristics of a `woman' are in ancient Indian society and to a later pre-occupation with sex and gender that is not found in earlier times.

In the charts for the Chinese texts we see that the words for male and female `root/faculty' (根) are most prominent in the \textit{Abhidhamma}\footnote{T26–T29. For a description of the sections in the Taishō Canon see \href{https://en.wikipedia.org/wiki/Taishō_Tripiṭaka}{wikipedia.org}.} as well as the \textit{Vinaya}\footnote{T22–T24.} and slightly less in the later sub-commentaries.\footnote{T40–T44.} In the \textit{Tantra} collections\footnote{T18–T21.} however, the word for `female shape' (女形), is far more prominent than its male counterpart. In all collections both words `root/faculty' (根) and `shape' (形) are used relatively little to denote sex/gender characteristics and are used more in other meanings.

To summarize, it is unclear what the word \textit{liṅga} refers to exactly in relation to sex and gender. It is certain that it does not relate merely to primary sex characteristics and most likely refers to secondary characteristics that also include behavior, dress, hair, etc. The words \textit{byañjana} and \textit{nimitta} seem to be used mainly in the later texts in relation to sex/gender and I treat them in this paper as synomyms of \textit{liṅga} but there might be more subtle differences. At least in the case of the \textit{ubhatob­yañ­janaka} there seems to be more of an emphasis on primary sexual characteristics as I will discuss in chapter \ref{ubhato}.\\

In the next chapters I will discuss the terms \textit{paṇḍaka} and \textit{ubhatob­yañ­janaka} in the Buddhist texts as well as terms relating to sex and gender as they occur in the \textit{Bhikkhunikkhandhaka}. A listing of the occurrences of the most common terms pertaining to sex and gender in Chinese texts is given in Appendix \ref{appendix1}. 
