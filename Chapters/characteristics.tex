\section{Liṅga and Vyañjana}
\label{linga}

As we have briefly touched upon, there was a lively discussion, probably around the 3th century BCE, on what the characteristics of sex or gender are. The original meaning of the term {\em liṅga} is 'characteristic mark or sign'\footnote{Nirukta 1.17} but later changed its meaning to 'sexual characteristic'. The 'third sex' ({\em napuṃsaka}) basically became a class of people that did not neatly fit into the categories of 'male' and 'female'. 

This discussion in the light of ordination into the Buddhist monastic order is all the more remarkable because monks and nuns forego the usual markers of sex and gender when they put on robes and shave their heads. Giving up gendered attire is one of the distinguishing characteristics of monastic life. In the Jain order this discussion made more sense. The Jain were naked ascetics and therefore the physical marks of sex could not so easily be given up. The women needed to wear a cloth to cover their bodies while the men could go naked. This was a very important point for the Jain because this difference also meant that as women could not let go of all earthly possessions they could also not reach enlightenment. This was one of the main points of dispute beteen what became the two sects in Jainism.

Burkhard Scherer\footnote{\cite{scherer} page 68} takes the term {\em liṅga} as a reference to the 'secondary sex organs' or characteristics of sexual difference, which also include behavioral differences so the term can be used to denote both biological sex and gender-identity as we define it today. He bases this conclusion on the work of Buddhaghosa, who listed the secondary characteristics of the male and female, which included beards and moustaches, motherly instincts, way of walking, etc. This is also in line with the first Brahmanical view on this matter as well as what is described in the Buddhist {\em Abhidharmakośa}\footnote{Abhidharmakośa IV.14 c. using the word {\em vyañjana} as a synonym for {\em liṅga}}, the {\em Samantapāsādikā} commentary\footnote{See \cite{anderson2016} page 237-240 for an English translation of the passage on change of {\em liṅga} in a monastic} and elsewhere. There is no indication that the Buddhists agreed with the Jains on the inclusion of underlying sexuality and sexual feelings in the definition of {\em liṅga}\footnote{The Jain also defined the sexual desire for the male body as part of the female characteristics and visa versa.}.
