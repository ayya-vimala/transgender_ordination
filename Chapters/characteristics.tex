\section{Liṅga and the Dispute on Women's Ordination}
\label{linga}

As we have touched upon in the context of the developments in the Jain order there was a lively discussion around the 3th century BCE on what the characteristics of sex or gender are. This controversy hinged on the identification of the signs to designate somebody as a woman, which logically also led to the examination of what is male, and 'neither male nor female'. This was a much wider religious debate that involved both the Jain and Buddhist communities as well as possible other sects. Bhikkhu Sujato\footnote{See \cite{sujato2009} page 241–242.} points out that these discussions were held amongst the monks, while the women themselves were notable by their absence. We have no indication that this was any different in the Jain order. It is therefore not surprising that at the Second Buddhist Council, which was held at around the same time, rules were laid down with regards to the ordination of nuns, and most likely the 'third sex' category, with it's perceived hyperlibidinous-ness, was also touched upon. We also find for instance various passages with regards to women's role in the order that seem out of place in the Buddhist scriptures but which use near identical wording in the Jain texts\footnote{See \cite{sujato2009} page 54–55.} so it is clear that there was a distinct influence between the orders.

This discussion regarding {\em liṅga} in the light of ordination into the Buddhist monastic order is all the more remarkable because monks and nuns forego the usual markers of sex and gender when they put on robes and shave their heads. Giving up gendered attire is one of the distinguishing characteristics of monastic life. In the Jain order this discussion made more sense\footnote{\cite{maes2016} pages 11–17 points out that the Jain's nakedness was one of the primary distinctive marks that set them aside from Buddhist monastics. The wearing of a bowl and robe was an important part of the Buddhist identity.}. The Jain were naked ascetics and therefore the physical marks of sex could not so easily be given up. The women needed to wear a cloth to cover their bodies while the men could go naked. This was a very important point for the Jain because this difference also meant that as women could not let go of all earthly possessions they could also not reach enlightenment. This was one of the main points of dispute between what became the two sects in Jainism.

The original meaning of the term {\em liṅga} before this dispute is 'characteristic mark or sign'\footnote{Nirukta 1.17. Also used in the early Buddhist Suttas in this way, f.i. Majjhima Nikāya 98 Vāseṭṭhasutta.}, but during this dispute the term was refined and the different orders developed different opinions about the characteristics of 'male' and 'female'. In the Buddhist Abhidhamma the term is often referred to exclusively as denoting sexual characteristics. Burkhard Scherer\footnote{\cite{scherer} page 68.} takes the term {\em liṅga} as a reference to the 'secondary sex organs' or characteristics of sexual difference, which also include behavioral differences so the term can be used to denote both biological sex and gender-identity as we define it today. He bases this conclusion on the work of Buddhaghosa, who listed the secondary characteristics of the male and female, which included beards and moustaches, motherly instincts, way of walking, etc. This is also in line with the first Brahmanical view on this matter as well as what is described in the Buddhist {\em Abhidharmakośa}\footnote{Abhidharmakośa IV.14 c. using the word {\em vyañjana} as a synonym for {\em liṅga}.}, the {\em Samantapāsādikā} commentary\footnote{See \cite{anderson2016} page 237–240 for an English translation of the passage on change of {\em liṅga} in a monastic.} and elsewhere. There is no indication that the Buddhists agreed with the Jains on the inclusion of underlying sexuality and sexual feelings in the definition of {\em liṅga}\footnote{The Jain also defined the sexual desire for the male body as part of the female characteristics and visa versa.}.

This change of meaning in the word {\em liṅga} has caused a lot of confusion and is especially important in determining the characteristics of the {\em paṇḍaka} and {\em ubhatob­yañ­janaka} in the next chapters.\\
\\
In the next chapters I will discuss the various terms and their occurrences in the Buddhist canon. A listing of the occurrences of the most common terms pertaining to gender like {\em paṇḍaka} and {\em ubhatob­yañ­janaka} in Chinese texts is given in Appendix \ref{appendix1}. 
