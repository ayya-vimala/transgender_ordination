\section{Liṅga and Vyañjana}

As we have briefly touched upon, there was a lively discussion, probably around the 3th century BCE, on what the characteristics of sex or gender are. The original meaning of the term {\em liṅga} is 'characteristic mark or sign'\footnote{Nirukta 1.17} but later changed its meaning to 'sexual characteristic'. The 'third sex' ({\em napuṃsaka}) basically became a class of people that did not neatly fit into the categories of 'male' and 'female'. 

This discussion in the light of ordination into the Buddhist monastic order is all the more remarkable because monks and nuns forego the usual markers of sex and gender when they put on robes and shave their heads. Giving up gendered attire is one of the distinguishing characteristics of monastic life. In the Jain order this discussion made more sense. The Jain were naked ascetics and therefore the physical marks of sex could not so easily be given up. The women needed to wear a cloth to cover their bodies while the men could go naked. This was a very important point for the Jain because this difference also meant that as women could not let go of all earthly possessions, they could not reach enlightenment. This was one of the main points of dispute beteen what became the two sects in Jainism.

Burkhard Scherer\footnote{\cite{scherer}} take the term {\em liṅga} as a reference to the 'secondary sex organs' or characteristics of sexual difference, which also include behavioral differences so the term can be used to denote both biological sex and gender-identity as we define it today. He bases this conclusion on the work of Buddhaghosa, a later commentator, who listed the secondary characteristics of the male and female, which included beards and moutaches, motherly instincts, way of walking, etc. This is also in line with the first Brahmanical view on this matter as well as what is described in the Buddhist {\em Abhidharmakośa}\footnote{Abhidharmakośa IV.14 c. using the word {\em vyañjana} as a synonym for {\em liṅga}}, the {\em Samantapāsādikā} commentary\footnote{See \cite{anderson2016} page 237-240 for an English translation of the passage on change of {\em liṅga} in a monastic} and elsewhere. There is no indication that the Buddhists agreed with the Jains on the inclusion of underlying sexuality and sexual feelings in the definition of {\em liṅga}\footnote{The Jain also defined the sexual desire for the male body as part of the female characteristics and visa versa.}.

--------------------------------------------------------------------


Still to do:


Claire's
BS Vinaya studies
Anderson study
Scherer study
















It is unclear what exactly “characteristics of a (wo)man” are. The word this hinges on is liṅga, which means sign or characteristics. It can refer to physical characteristics but not necessarily. The words for “characteristics of a (wo)man” are itthiliṅgaṃ and purisaliṅgaṃ. These words appear in the canon only 5 times, in later texts like the Abhidhamma and the Milindapañha. It also appears in the early suttas once, namely in Digha Nikāya 27 which describes the evolution. In the latter case, it seems that liṅga indeed refers to biological sex.


Burkhard Scherer\footnote{\cite{scherer}} and others take the term liṅga as a reference to the ‘secondary sex organs’ or characteristics of sexual difference, which also include behavioral differences so the term can be used to denote both biological sex and gender-identity as we define it today. They base this conclusion on the work of Buddhaghosa, a later commentator. However, the notion of gender as we have today is no doubt different from that in the time of the Buddha. More research in this field and also the corresponding parallels in other schools is needed to get a better picture.

In any case, it seems that there is a lot of uncertainty about what liṅga actually refers to. There are different attempts to explain the term in later commentarial literature but these have very different views from each other. All this has an impact on the ordination procedure, whereby one is asked if one is a purisa(man) or itthi (woman). It would follow from this passsage in Pārājika 1 that in order to be a man or woman for the purpose of ordination, one should have the liṅga of a man or woman.