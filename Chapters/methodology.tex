\section{Methodology}

\subsection{SuttaCentral}
The main resource I used for both Pali and Chinese Vinaya (and sometimes Sutta) texts is \href{https://suttacentral.net/}{SuttaCentral.net}. This website provides the Vinayas of all the schools in their original language as well as translations of all the Pali texts, divided into schools and chapters. Especially the lookup tools provided for the Pali and Chinese texts are very valueble; by hovering over a certain text the dictionary entry of that word/character or combination of characters is shown. Clicking on this brings you to the full dictionary entry.

SuttaCentral is however limited in the texts that are provided and the commentarial texts are not given.

\subsection{BuddhaNexus}
\href{https://buddhanexus.net/}{BuddhaNexus.net} provides the most extensive database of texts in Pali, Chinese, Sanskrit and Tibetan and I used this site a lot to find parallels between texts, expecially references of the passages found in SuttaCentral to the Commentarial literature. I also used the BuddhaNexus database to do a global search for specific words, which gave me a wealth of information and references and made it easy to find commentarial explanations of texts in the Canon. BuddhaNexus uses the Taisho numbering for the Chinese texts and PTS numbering for Pali texts which makes cross-referencing easy.

Furthermore, I used the BuddhaNexus database to create the frequency graphs, albeit by writing my own Python program to search through all the texts.

\subsection{Indo-Tibetan Lexical Resource}
\href{https://www.itlr.net/}{ITLR.net} is a lexical project built around Sanskrit headwords. It is mainly aimed at Tibetan but often also provides Pali and Chinese entries and references. It is an ongoing project with frequent new additions. I have used this on occasion to find references and translations to certain specific words.