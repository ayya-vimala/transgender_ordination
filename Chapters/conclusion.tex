\section{Conclusion}
In this article I have tried to give an alternative reading of the terms {\em paṇḍaka} and {\em ubhatob­yañ­janaka} based on the evidence we find in Vedic, Jain and Chinese texts. There have been many different attempts by different authors to capture the meanings of these terms, based on a variety of reasonings. All are bound to fail because the meaning of these terms only makes sense in the time and place in which they belonged and for over 2000 years people have attempted to solve the inherent ambiguities in these terms, in commentarial texts and sub-commentaries, up to the present day. The truth is that we will never know for sure as the true and full meaning of these terms cannot be captured in single words or phrases and have vanished in the mists of time. 

The only thing we can say for certain is that the {\em paṇḍaka} and the {\em ubhatob­yañ­janaka} are seen as problematic because they are unable to keep their precept of celibacy. This is also confirmed by the Chinese commentaries\footnote{T85 2792 毘尼心 0667b25–0667c05} as well as indicated in the origin stories. The idea that they are a threat to celibate monks because the monks might be attracted to them is not supported by the origin stories and is just another projection used in male-dominated societies in which women are made responsible for the feelings and desires of men. 

The main, and only, criteria for not allowing ordination to certain individuals here is their difficulty in keeping the precepts. This is a fair reason for barring somebody from ordination. All criteria based on perceived or imagined sex and gender characteristics that might or might not be part of a {\em paṇḍaka} or {\em ubhatob­yañ­janaka} are not. Transgenders and intersex people are generally not hyperlibidinous and are just as able to keep the precepts as any man or woman. 

It is therefore unfair, even cruel, to deny ordination to otherwise eligible individuals on the basis of a very limited and a most likely erroneous understanding of these terms, even more so because we know with a fair amount of certainty that they were inserted into the Vinaya after the Buddha's passing away, most likely under influence of discussions with other sects and discussions that were held in a male patriarchal system where the fear of the feminine, and thus everything that is seen as 'not-male' is paramount. 
