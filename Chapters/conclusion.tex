\section{Conclusion}
- go from the premise of a parallel development of the Buddhist and the Jains, also borrowing ideas from each other.
- most likely meaning an effeminate male, like a transvestite, who can be mistaken as a woman and can therefore not ordain as a monk because other monks might develop lust towards this person. As transvestites were often associated with prostitution it would seem logical that such persons could pose a threat to celibate monks. 
- there is no prohibition against hte ordination of masculine females as there are different words used for them.

In the early Buddhist Pāli texts, we find remarkably little on the subject, but in later commentaries and Mahāyāna texts we find a number of recognized third sex types that are discussed in more detail, defined also based on their sexual behavior and not only on their external characteristics. As Buddhist monks did not go naked, and in fact identified themselves as different from Jains by the fact that they had a bowl and robe, this issue might have been less important than for the Jain.


(CLAIRE's ESSAY ON THIS)


The rule against ordination of both the {\em paṇḍaka } and the {\em ubhatob­yañ­janaka } are laid down in Khandaka 1 and are clearly against the full ordination of these two types of individuals, the {\em upasampadā}. Yet it is interesting that in both cases the person to whom the rule applies is said to have been given the {\em pabbajjā}. This really only makes sense if we understand {\em pabbajjā} here to be equivalent to {\em upasampadā}. In fact this equivalence between {\em pabbajjā} and {\em upasampadā} is what we find throughout the earliest Vinaya, and indeed the suttas \footnote{The {\em sāmaṇeras/īs} are barely mentioned in the suttas. Instead we find the figure of the {\em samaṇuddesa}, 'one designated as a {\em samaṇa}', who seems to have had a looser affiliation with the Sangha, that is, no proper ordination. The commentaries glosses them as {\em sāmaṇeras}, but this might be an oversimplification. More likely they were a kind of precursor to the more formal status of novice. It seems likely that such people merely put on robes, and then lived in with loose connection to a particular community of ascetics, in which case their sex would have been a non-issue. I would argue it is natural to see novices proper in the same way. But the {\em samaṇuddesa} remains obscure.}. In any case, the rules itself are clearly limited to {\em upasampadā}.

But the commentary also makes a distinction between {\em pabbajjā} and {\em upasampadā} and does not allow either for ordination.



There are further complications, such as getting agreement from all Sangha members to ordain such a person. It is possible not everyone would feel at ease with it, for a number of possible reasons.

In cases such as this, we should always err on the side of compassion. If there is no clear rule in the Vinaya, are we obliged to follow the commentarial interpretation? Of course there is always the danger that an ordination is not accepted in the wider Sangha, but with the Bhikkhuni ordinations we have seen that acceptence gradually grows. It is up to the individual communities, with agreement of all Sangha members, to ordain a person or not and to judge their suitability, even if the interpretation of other communities is different. 

-----------------------------------------------------------------
How about intersex and transgender?


In this article I have tried to give an alternative reading of the terms {\em paṇḍaka} and {\em ubhatob­yañ­janaka} based on the evidence we find in Vedic, Jain and Chinese texts. There have been many different attempts by different authors to capture the meanings of these terms, based on a variety of reasonings. All are bound to fail because the meaning of these terms only makes sense in the time and place in which they belonged and for over 2000 years people have attempted to solve this ambiquity, in commentarial texts and subcommentaries, up to the present day. The truth is that we will never know for sure as the true and full meaning of these terms cannot be captured in single words or phrases and have vanished in the mists of time. 

The only thing we can say for certain is that the {\em paṇḍaka} and the {\em ubhatob­yañ­janaka} are seen as problematic because they are unable to keep their precept of celibacy. This is also confirmed by the Chinese commentaries\footnote{T85 2792 毘尼心 0667b25–0667b26} as well as indicated in the origin stories. The idea that they are a threat to celibate monks because the monks might be attracted to them is not supported by the origin stories and is just another projection used in male-dominated societies in which women are made responsible for the feelings and desires of men. 

It is therefore unfair, even cruel, to deny ordination to otherwise eligible individuals on the basis of a very limited and a most likely erroneous understanding of these terms, even more so because we know with a fair amount of certainty that they were inserted into the Vinaya after the Buddha's passing away, most likely under influence of discussions with other sects and discussions that were held in a male patriarchal system where the fear of the feminine, and thus everything that is seen as 'not-male' is paramount. The main, and only, criteria for not allowing ordination to certain individuals here is their difficulty in keeping the precepts. This is a fair reason for barring somebody from ordination. All criteria based on perceived sex and gender characteristics that might or might not be part of a {\em paṇḍaka} or {\em ubhatob­yañ­janaka} are not.
