\section{Conclusion}
In this article I have shown that the terms {\em paṇḍaka} and {\em ubhatob­yañ­janaka} have very likely deep roots in Vedic mythology and in case of the {\em paṇḍaka} also the enactment of that mythology by real people. Over thousands of years people in different parts of the Buddhist world have been trying to find explanations and interpretations of these words based on their own culture and society while very little research has been done as to the actual meaning of these words at the time of the Buddha and shortly thereafter as well as the influence of other orders like the Jain. 

We saw that the Chinese scribes, who translated the Vinaya, could only make sense of these words using a concept they knew in that culture, namely their own imperial palace eunuchs from the Han Dynasty; a concept which is vastly different from what the term {\em paṇḍaka} is trying to convey. It would equally be a mistake for us to try and interpret these words in terms of 'transgender' or 'intersex', terms we are familiar with in our culture. The {\em paṇḍaka} belongs in a time and place where the fabric of reality and mythology are woven into each other in a way that is daunting for our western rational minds. For thousands of years various authors have attempted to solve the inherent ambiguities in these terms, in commentarial texts and sub-commentaries, up to the present day. The truth is that the full meaning of these terms cannot be captured in single words or phrases based on modern concepts and any interpretation of these terms will always be flawed.

The only thing we can say for certain is that the {\em paṇḍaka} and the {\em ubhatob­yañ­janaka} are seen as problematic because they are unable to keep their precept of celibacy. This is also confirmed by the Chinese commentaries\footnote{T85 2792 毘尼心 0667b25–0667c05} as well as indicated in the origin stories. The idea that they are a threat to celibate monks because the monks might be attracted to them is not supported by the origin stories and is just another projection used in male-dominated societies in which women are made responsible for the feelings and desires of men. I find such arguments rather patronizing. In my experience the monks who have voluntarily taken up the training in the Dhamma are very well aware of the responsibility this brings.

The main, and only, undisputed criterion for not allowing ordination to certain individuals here is their difficulty in keeping the precepts. This is a fair reason for barring somebody from ordination. All criteria based on perceived or imagined sex and gender characteristics that might or might not be part of a {\em paṇḍaka} or {\em ubhatob­yañ­janaka} are not. Transgenders and intersex people are generally not hyperlibidinous and are just as able to keep the precepts as any man or woman. 

It is therefore unfair, even cruel, to deny ordination to otherwise eligible individuals on the basis of a very limited and a most likely erroneous understanding of these terms, even more so because we know with a fair amount of certainty that they were inserted into the Vinaya after the Buddha's passing away, most likely under influence of discussions with other sects and discussions that were held in a male patriarchal system where the fear of the feminine, and thus everything that is seen as 'not-male' is paramount. 

I certainly do not wish to justify ignoring any of the rules in the Vinaya. But this is an instance where contemporary social conventions are simply not covered by any of the Vinaya rules. We never before had the medical knowledge about intersex or the ability to change sex with Hormone Replacement Therapy and surgery. In such a case we must not question how to make the Vinaya rules apply to the the convention, but whether such rules apply at all. And when such a rule application causes unnecessary suffering on the basis of very feeble arguments, I think it is unjust to do this. 

In speaking with other Sangha members, the question often arises as to which Sangha, Bhikkhus or Bhikkhunis, a transgender or intersex person should ordain into and as such also according to which ordination procedure. I think we should simply leave such questions to the individuals involved based on their gender-experience in consultation with the members of the community they wish to ordain into. The Vinaya has even given us an example on what to do in this case: the person can simply live in the Sangha according to their own gender-experience\footnote{PTS vol. 3 page 35}. As I have outlined in this article, in ancient India there was a lively debate with regards to the characteristics that make up a 'man' or a 'woman'; these are not so clear-cut and also not limited to primary and secondary sex characteristics. 

Article 1 of the UN Universal Declaration of Human Rights reads: "All human beings are born free and equal in dignity and rights". Denying ordination on the basis of sex or gender is against basic human rights and as Buddhists it is not only our duty to ensure the ethical standards that are expected of us in our society, but also to be the living examples of the Buddha's compassion for all beings.

