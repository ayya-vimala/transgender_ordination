\section{Conclusion}
I have started my analysis by pointing out that a translation of the terms {\em paṇḍaka} and {\em ubhatob­yañ­janaka} based on a modern understanding of concepts is lacking in clarity which has led to the exclusion of transgender and intersex people from ordination.

Not only is the meaning of these terms not well understood, they have most likely been included in the Vinaya during the Second Council after the Buddha's death as the result of a wider religious debate with regards to the position of women in the Buddhist and Jain Orders, which hinged on the identification of the signs to designate somebody as a woman, which logically also led to the examination of what is male, and `neither male nor female'.

In this paper I have shown that the terms {\em paṇḍaka} and {\em ubhatob­yañ­janaka} are very likely to have deep roots in Vedic mythology and also the enactment of that mythology by real people. Over thousands of years people in different parts of the Buddhist world have been trying to find explanations and interpretations of these words based on their own culture and society while very little research has been done as to the actual meaning of these words at the time of the Buddha and shortly thereafter as well as the influence of other Orders like the Jains. 

We saw that the Chinese scribes, who translated the Vinaya, could only make sense of these words using a concept they knew in that culture, namely their own imperial palace eunuchs from the Han Dynasty, a concept which is vastly different from what the term {\em paṇḍaka} is trying to convey. It would equally be a mistake for us to try and interpret these words in terms of `transgender' or `intersex', terms we are familiar with in our Western culture. The {\em paṇḍaka} belongs in a time and place where the fabric of reality and mythology are woven into each other in a way that is daunting for our Western rational minds. For thousands of years various authors have attempted to solve the inherent ambiguities in these terms, in commentarial texts and sub-commentaries, up to the present day. The truth is that the full meaning of these terms cannot be captured in single words or phrases based on modern concepts and any interpretation of these terms will always be flawed.

The only thing we can say for certain is that the {\em paṇḍaka} and the {\em ubhatob­yañ­janaka} are seen as problematic because they are unable to keep their precept of celibacy. This is also confirmed by the Chinese commentaries\footnote{T85 2792 毘尼心 0667b25–0667c05.} as well as indicated in the origin stories. The idea that they are a threat to celibate monks because the monks might be attracted to them is not supported by the origin stories. I find such arguments rather patronizing. In my experience the monks who have voluntarily taken up the training in the Dhamma are very well aware of the responsibility this brings.

The main, and only, undisputed criterion for not allowing ordination to certain individuals is their difficulty in keeping the precepts. This is a fair reason for barring somebody from ordination. All criteria based on perceived or imagined gender characteristics that might or might not be part of a {\em paṇḍaka} or {\em ubhatob­yañ­janaka} are not fair reasons. Transgender and intersex people are generally not hyperlibidinous and are just as able to keep the precepts as anyone else. 

It is therefore unfair, even cruel, to deny ordination to otherwise eligible individuals on the basis of a very limited and a most likely erroneous understanding of these terms, even more so because we know with a fair amount of certainty that they were inserted into the Vinaya after the Buddha's passing away, most likely influenced by discussions with other religious traditions which were held in a male patriarchal system where the fear of the feminine, and thus everything that is seen as `not-male' is paramount. The wholesome aspiration to ordain and practice in line with the Dhamma is something that needs to be encouraged, not disparaged. Ordination as a monastic is not a right to be acquired to become part of an elite group through an initiation ritual. It is to be welcomed if someone feels inspired to play a part in the propagation of the Dhamma and to help safeguard it for future generations. 

I certainly do not wish to justify ignoring any of the rules in the Vinaya. But this is an instance where contemporary social conventions are simply not covered by any of the Vinaya rules. We never before had the medical knowledge about intersex or the ability to change sex with Hormone Replacement Therapy and surgery. In such a case we must not question how to make the Vinaya rules apply to the convention, but whether such rules apply at all. And when such a rule application causes unnecessary suffering on the basis of very feeble arguments, I think it is unjust to do this. Regardless of how the Vinaya is interpreted, the doctrine of {\em anattā} (non-self), which is fundamental to all Buddhist schools, denies that there is an identity or lasting entity at the center of any being. So this makes sex and gender difference at the deepest level a superficial factor just like race, ethnicity, appearance or social status. Therefore to deny anybody ordination on the basis of this is in itself against the Dhamma.

In speaking with Buddhist monastics the question often arises as to which {\em Saṅgha}, Bhikkhus or Bhikkhunis, a transgender or intersex person should ordain into and as such also according to which ordination procedure. This question emerges from the distinct binary structure of the Buddhist institution that reflects a society vastly different from our Western one. This might have been appropriate at the time of the Buddha, but is not necessarily appropriate for our current time and place and we will have to rethink how we deal with these issues. I therefore think we should leave such questions to the individuals involved based on their gender-experience in consultation with the members of the community they wish to ordain into. The Vinaya has given us an example where the person could simply live in the {\em Saṅgha} according to their own gender-experience\footnote{PTS vol. 3 page 35. See also chapter \ref{trans}.} where they could practice in a way that was most appropriate to them in order to get the best possible opportunities to eradicate defilements and practice the teachings. As I have outlined in this article, in ancient India there was a lively debate with regards to the characteristics that make up a `man' or a `woman' and these are not so clear-cut and also not limited to primary gender characteristics. There are however many variables involved and in which community a person would fit best is not a question that is easy to answer and should be carefully considered on a case-by-case basis. 

The preservation of the religious institution and its public image is an important reason for the establishment of the Vinaya\footnote{Brenna \cite{artinger} cites Shayne Clarke, page 66.}. As Buddhism has spread across the world and across many socio-cultural environments, the challenge for the institution is to maintain its integrity while at the same time acknowledge the socio-cultural differences in the environments it operates in, the people who support it and to whom it aims to provide a refuge from suffering. Unfortunately we see all too often that people in Western cultures, already disillusioned with religion due to the scandals, misogyny and sexism in the Catholic Church, turn away from Buddhism, and therewith also from the Dhamma, because they are unable to reconcile the inclusive nature of the teachings with yet another rigid patriarchal institution that seems out of place in the modern Western world. If we are not careful in addressing these issues we will miss the ball entirely and in our aim to preserve the reputation of the {\em Saṅgha} we will lose it.

Article 1 of the UN Universal Declaration of Human Rights reads: ``All human beings are born free and equal in dignity and rights''. Denying ordination on the basis of sex or gender is against basic human rights and as Buddhists it is not only our duty to ensure the ethical standards that are expected of us in our society, but also to be the living examples of the Buddha's compassion for all beings.

\begin{quote}
As Buddhists who espouse the ideal of unconditional loving kindness and respect, judging people on their behavior instead of their birth, we should be well positioned to show leadership on the development of gender equality in the modern world and the consequent reduction of suffering for half the world’s population. Moreover, if Buddhism is to remain relevant and grow, we must address these issues head on. But how can we speak about gender equality when some of our own Theravada Buddhist organizations are gender biased? {\em Ajahn Brahmavaṃso}
\end{quote}
