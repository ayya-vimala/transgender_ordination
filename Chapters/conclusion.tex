\section{Conclusion}
- go from the premise of a parallel development of the Buddhist and the Jains, also borrowing ideas from each other.
- most likely meaning an effeminate male, like a transvestite, who can be mistaken as a woman and can therefore not ordain as a monk because other monks might develop lust towards this person. As transvestites were often associated with prostitution it would seem logical that such persons could pose a threat to celibate monks. 
- there is no prohibition against hte ordination of masculine females as there are different words used for them.
- pandaka is most likely the same thing, like hijra are a class of transvestite dansers and prostitutes who also castrate themselves as their initiation as a Hijra. Therefore they are also euneuchs.

So the most likely translation for both is Hijra


This distinction between actual sex and grammatical gender seems to be similar to contemporary debates between what sex and gender really are and how they relate to each other, which also leads to much confusion i.e. is somebody born with female sex characteristics also a woman? Or is somebody a woman when they look like one?


Do we allow the lives of people to be affected on the basis of just one ambiguous paragraph in the Vinaya, which has most likely been added at a later date, possibly during the Second Counsil. If we go from the premise that this word is indeed the synonym to napumsaka and basically means transvestite singers, dancers and prostitutes, it is logical that they would pose a threat to celibate monks and therefore not be allowed to ordain. We cannot 


Another route to interpreting the texts is, as always, comparative study. It may turn out that the prohibition against ubhatobyañjanakas is a relatively late development that happened after the time of the Buddha. If so, we would have good grounds for disregarding this prohibition.


The pandaka does not allow itself to be reduced to a mere word to make it acceptable and understandable for the rational mind. As \cite{nanda} argues: "where Western culture stenuously attempts to resolve sexual contradictions and ambiguities, by denial or segregation, Hinduism appears to allow opposites to confront each other without resolution." It is the divine representation of the feminine within the masculine. It is the human representation of the mythical tales which have deep psychological roots, namely the ambivalence that leads to the inner struggle between man's love of the feminine and his fear thereof. The pandaka is not something that we can compare to any concept we have in western societies. If we have to capture the pandaka in one word, it would be 'hijra'. The hijra is a man, impotent from birth, emasculated in an initiation ritual, part of a caste, a religious seeker enacting the feminine of Śiva by dressing and behaving in traditional women's gender roles, changing into it and feeling the feminine sexual desire for the masculine.

There are some rare cases of people who are raised from birth as girls that later became assigned as hijra after they failed to develop secondary female sexual characteristics (breast development and menarche) at puberty\footnote{\cite{nanda}}. Although there is very little evidence to go on, I believe that these respresent the itthipandaka we find in the Bhikkhuni Khandaka.
