\section{Introduction}
The legality of Bhikkhunī ordination in the Theravada and Tibetan lineages of Buddhism has been a hotly debated issue for many years. Thanks to the efforts and research of many monastics and academics, the first full Theravada ordination was held in Perth in October 2010 (\cite{sujato2009}, \cite{analayo2013}). Although still not widely recognized in several traditional Theravada countries, recognition is growing and the number of Bhikkhunīs is slowly increasing. 

There are however two groups of people that have been marginalized and excluded from ordination, groups which we will refer to here with the Pāli terms used in the Theravada Vinaya: {\em paṇḍaka} and {\em ubhatob­yañ­janaka}. There have been various translations and interpretations of these terms with the consequence that intersex people and transgenders have been barred from ordination. But there is also much ambiquity.

When studying the Buddhist scriptures, especially where there are groups of people who seem to be marginalized, it is important to understand where and under which circumstances these concepts and interpretations have originated. The Early Buddhist Texts mainly focus on the teachings themselves, and much less on the socio-cultural environment in which these originated. In fact, they seem to deal with sex and gender as a given, that need no further discussion. So we have to look elsewhere for more information on this topic, like the pre-Buddhist Vedic culture\footnote{Note that in this work I have deviated from some of the earlier points I made in \cite{vimala} with regards to the Vedic concept of the third sex. I have now rejected certain sources on the basis that I found them unreliable upon closer inspection and I hope to have rectified this with more thourough research.} and the Brahmanic and Jain cultures at the time of the Buddha and thereafter. 

The Jains were contemporaries to the Buddha and also shared the acceptance of a third sex, which led them to speculate what the nature of this third sex might be as compared to male and female. Just like in Buddhism, the Jain order had a strong interest in controlling the sexuality of it's monastics. Jain monastics live celibate and at the time of it's emergence, the monks were mostly naked ascetics. The prestige and power of the order depended to a large extend on public opinion and therefore on the purity of their behavior, as well as their external appearance. The "third sex" was therefore subject of a very lengthy debate within the order within a larger discussion on the nature of sexuality. The Jains left us a vast literature in Sanskrit and Prakrit in which these discussions and concepts are discussed in detail.