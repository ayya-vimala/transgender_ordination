\section{Buddhism and Star Wars}

I come from a progressive Western country and when I grew up I hardly noticed patriarchy. It was not that it was not there, but it was relatively mild. I could be who I wanted to be, study, work, just like any man. There were no boys- and girls-schools and no school uniforms. Nobody lifted an eyebrow if I showed up in a man's business suit at university, or as a goth. My family had members and friends from all sexes, genders, sexualities and races and I never had a lived understanding of what discrimination was. I was spoiled. 

I had no idea what I was getting myself into by ordaining as a Buddhist monastic. I throught I was going to practice the Buddhist teachings and meditate. Little did I know that I was going to battle with depression and feelings of self-loathing due to the patriarchal structure I was so unfamilair with.

Next to the beautiful teachings of the Buddha, we imported a patriarchal system that existed at that time. The Buddha had to work in a very patriarchal environment and in order to bring his teachings, he also had to make some consessions. Thich Nhat Hanh explains that the Buddha had to manage the prejudice of the people in that time. But after the Buddha's death the monks took over and the system became embedded in a patriarchal cocoon in the name of the Buddha. Because the Buddha, a man, had supposedly said so, the monks laid the groundwork for over 2000 years of patriarchy and strengthened it over time.

In the {\em Saṅgha}, we get thrust into a segregated system where women are treated as inferior, and men are treated as gods and this is done in the name of the Buddha. Nowhere in the early Buddhist teachings however do we find him justifying anything like this. He treated all people alike, regardless of gender, sex or caste. As I pointed out in this paper, the patriarchal attitude is something that is not the Buddha's teachings, but finds it's way back to Vedic times and right through to today. It is not something the Buddha endorsed, but something he somehow had to manage in the culture of that time. But his name is used to keep the patriarchal structure alive and strong.

We often focus on how patriarchy harms women and LGBTIQA+ people, but it is something that harms everyone, men included, but the ways in which their golden cage harms them is far more subtle. It is obvious to see how in Buddhism nuns get less support, have to go behind in line for alms (if they are not skipped by the lay devotees), how they are treated as a threat to celibate monks, and how these lead to feelings of inferiority and worthlessness. While their male counterparts get veneration and their interaction with women revolves around women looking up to them and bowing to them. 

The term ``Toxic masculinity'' refers to some of the negative norms and expectations that are placed on men and boys in our society, and in this case on monks once they enter the {\em Saṅgha}. Toxic masculinity is the way in which they are socialized to perform masculinity — through suppressed emotions and dominance. Toxic masculinity aspires to toughness but is, in fact, an ideology of living in fear: The fear of ever seeming soft, tender, weak, or somehow less than manly. This insecurity is perhaps the most stalwart defining feature of toxic masculinity. Insensitivity to emotions, pressure to act mean or competitive to fit in are some of the ways in which this insecurity manifests itself. The {\em Saṅgha} learns that expressing emotions is taboo, which causes long-term harm to their relationships with each other and with people of other genders. Women are met with anger or stonewalling when they dare to come too close. Women are scary because they lure monks away from celibacy so that fear of the feminine is projected onto women as the evil ogres, the temptress.

Research done in the last 60 years confirms that men who conform to masculinity have poor mental health. Because men have to uphold these toxic gender norms throughout the course of their life, men can end up repressing their feelings, which can harm them and those close to them.

Although it can be tough for men to identify how the patriarchy, and toxic masculinity, affects them, it's up to them to recognize the signs, understand where negative behaviours come from, and learn how to address them. Jordan Stephens says in The Guardian: ``Accepting the patriarchy from a place of false benefit will prevent you from ever truly loving yourself or understanding others. It's OK to feel sad. It's OK to cry. It's OK to have loved your mum and dad. It's OK to have missed them or wanted more affection. It's OK to take a moment when you're reminded of these truths. When you allow your brain to access these emotions, it knows exactly what to do. So nurture yourself. Talk honestly to the people around you, and welcome the notion of understanding them more than you have ever done before\footnote{https://www.theguardian.com/commentisfree/2017/oct/23/toxic-masculinity-men-privilege-emotions-rizzle-kicks}.''

Fighting patriarchy is more than allowing women to ordain, or to give them support, or to let them go on pindapata in order of ordination with the monks. These things are important, but they do not touch the underlying problem; the fear that has caused patriarchy in the first place.

On 1st December 1955, a black woman called Rosa sat on a bus in a place reserved for white people. This small act of civil disobedience started a revolution that led to the abolishment of segregation based on skin color in the United States. But today, 65 years later, there is still no equality and black people still have to fight for their rights in a predominantly white patriarchal society. This is caused by white man's fear of black people and therefore they need to be controlled at all cost. It is no different in the {\em Saṅgha} where the nuns, and their frightening emotionality, has to be controlled. 

The more progressive monks are very happy to accept nuns ordination, as long as they stay far away in their own monasteries and don't mix, and certainly not talk! A woman crying causes an immediate flight-reaction of fear. The Vinaya is used as a shield against the frightening emotionality that also monks possess, but aim to deny. A monk doesn't cry. Being just around other men feels safe, they also do not show their emotions and this way the can simply project their fears outwards onto nuns and women, who are ``not us'' and far away.

Fighting patriarchy in the {\em Saṅgha} is about supporting monks, teaching them to accept their emotions and to embrace them rather than suppress them. And then there is no more need to blame women for uncomfortable feelings but they are accepted as part of live.

Jordan Stephens says: Only by confronting our privilege and opening up our emotions will we live a more positive life.

--------------------------------------------------------------------------
It is no secret that the Jedi Order in the Star Wars saga is modelled after the Buddhist {\em Saṅgha}. The evil that they fight are our own defilements. In the original Star Wars movies as well as the prequels we are confronted with a Jedi Order that is predominantly male and even the females in it are hardly shown, nor have much to say; they remain in the background in much the same way that the Buddhist nuns are there, but hardly seen, nor have a voice. The story is the traditional hero-story.

In the much newer Clone Wars and Rebels we see far more female protagonists and it is not surprising that it is a female padawan who first sees the corruption of patriarchy in the Jedi Order and turns her back to it, only to become one of the most powerful Force users in the galaxy, even able to stand up to Darth Vader while telling him with pride ``I'm no Jedi''.

The newer movies, especially The Last Jedi breaks with that tradition and shows us a way forward that I feel the {\em Saṅgha} would do well to heed. The fight is against patriarchy itself, in all its aspects. The women, especially those using The Force (which can be translated to an empathic ability to connect with people), are the mature leaders while the men like to blow things up and are not in control of their anger, fear and obvious pain: the true face of toxic masculinity. They are afraid of power of the women, afraid of their honest emotionality that they also possess but repress so it manifests in rage and denial.

The hero from the first Star Wars series, Luke Skywalker, the Chosen One, is now portrayed by a scared and bitter man, who runs away to a remote island because he cannot face the pain of his past mistakes. He wants to destroy himself and whatever is left of the order with him. The female hero of this story, Rey, is empathic and kind but has a power that scares Luke more than anything: the power to face up to our emotions and accept them as they are, the power to reach out to people, even those that are our enemies. 

The film is a step in the right direction when it comes to gender diversity because it refuses to glorify its male heroes in simplistic ways that create unrealistic, harmful expectations for everyone involved. True gender diversity in the {\em Saṅgha} will come by recognizing that the system of patriarchy we live in benefits no one. 

The Last Jedi is a story that recognizes that we won’t “win” by fighting the things we hate, but by saving the things we love—and by being able to tell the difference between someone who is unable to accept any degree of accountability for their actions, and someone who has the capacity to recognize his mistakes and learn from them and to forgive others. It is a deeply empathetic story that explores the dangers of toxic masculinity, the competency of women, and the boxes we all must break out of to be free.

The Last Jedi gives the {\em Saṅgha} a pathway to change the patriarchal system into one that is better in supporting the Dhamma. I am not advocating to break Vinaya rules, but to look at them with compassion and find the real treasures within them, while refusing the patriarchal reading of them. After all, we know that a large part of the Vinaya was laid down by men with the explicit reason to solidify their power and make it last for a very long time.






https://www.mkzc.org/single-post/2009/10/09/an-alternative-to-patriarchy-women-e2-80-99s-search-for-non-christian-religious-routes


https://www.huffingtonpost.ca/2017/11/10/patriarchy-men-boys_a_23273251/?guccounter=1&guce_referrer=aHR0cHM6Ly93d3cuc3RhcnRwYWdlLmNvbS8&guce_referrer_sig=AQAAANzpnznAsBSG6CSxoANQK1yzTprbIyvosx4DcUsqXiIzXsr6u-BT-l1mNJGHLCllmA9mFrXNIZiRZLfU2J0FvBpkhoZ7i0ycyEqrPJTnu8bxR1Qh7ZnR-Mp4pLYDbeVx8SITrkLwI7NtOEsusG51vkji13_bLe_HnK0PYerEeSph

https://www.newstatesman.com/culture/film/2017/12/last-jedi-first-properly-feminist-star-wars

https://www.denofgeek.com/movies/toxic-masculinity-is-the-true-villain-of-star-wars-the-last-jedi/