\section{Vedic, Brahmanic and Jain scriptures}

\subsection{Emergence of the Third Gender}
We can trace the emergence of the concept of a third sex back to the late Vedic period (800–600 BCE). The term {\em napuṃsaka} was an umbrella term used to denote men who were impotent, effeminate or transvestite\footnote{\cite{zwilling}}. Literally the term means "not-a-male" i.e. men who did not conform to gender-role expectations. Such "un-males" constituted a distinct social group with institutionalized roles like singers, dancers and prostitutes. The modern day {\em hijras} are the contemporary representatives of these {\em napuṃsaka}\footnote{A good study on the {\em hijras} is provided by \cite{nanda}}.

The adoptation of the word {\em napuṃsaka} as a grammatical third gender\footnote{{\em Śatapatha Brāhmaṇa (ŚB) 10.5.1.2–3}} in the 6th century BCE seems to have prompted a significant shift in meaning. Because now the {\em napuṃsaka} was interpreted s meaning "neither male nor female". This resulted in the previously mentioned "un-males" to be regarded as persons with ambiguous sex\footnote{\cite{zwilling2000}}.

Just after the late Vedic period we see that a set of terms relating to the class of {\em napuṃsaka} has emerged like {\em klība} (sexually defective man\footnote{As pointed out by \cite{zwilling} the nature of the {\em klība} is suggested by the Bṛhadāraṇyaka Upaniṣad 6.1.12 and can be acquired due to the destruction of the penis as in ŚB 1.4.3.19}) and {\em paṇḍaka} ("impotent", or "sterile"\footnote{Atharva Veda (AV) 8.6.7, 11.16. The etymology of {\em paṇḍaka} is unknown but cf. baṇḍa at AV 7.65.3 is glossed by the commentator as {\em nirvīrya} ("impotent") \cite{zwilling}. Albrecht Wezler has suggested that paṇḍa and paṇḍaka be regarded as ultimately derived from {\em *apa}+ {\em āṇḍā}, thus: “one who has no testicles (anymore),” designating, if used as a substantive, a “eunuch,” etc. \cite{wezler}}). Both of these types were associated with transvestism and dancing, much like the modern day {\em hijras}.

Just like in Buddhism, the Jain order had a strong interest in controlling the sexuality of it's monastics. Jain monastics live celibate and at the time of it's emergence, the monks were mostly naked ascetics. The prestige and power of the order depended to a large extend on public opinion and therefore on the purity of their behavior, as well as their external appearance. The "third sex" was therefore subject of a very lengthy debate within the order. 

In addition to these practical considerations, there was a debate within the Jain community as to whether women can attain spiritual liberation because the monks felt it was improper for them to go naked. Eventually it was this dispute that led to the schism between the two major Jain orders\footnote{\cite{dudas}}\footnote{The two main sects of Jainism, the {\em Digambara} and the {\em Śvētāmbara} sect, likely started forming about the 3rd century BCE and the schism was complete by about 5th century CE.}. This controversy hinged on the identification of the signs to designate somebody as a woman, which logically also led to the examination of what is male, and neither male nor female. 

\subsection{A note on Grammar}
Most of our modern European languages stem from a language group called the indo-european languages, of which Vedic Sanskrit lies at the base. As we have seen above, there were only two recognized genders in Sanskrit until the 6th century BCE, when a third grammatical gender was introduced. This is significant because it forced people to assign gender to all objects including all living creatures and humans. Gender was seen as a property belonging to objects and objects are gendered by the presence or absence of certain defining characteristics or {\em liṅga}\footnote{The original meaning of {\em liṅga} is "characteristic mark or sign" (Nirukta 1.17) but later starts to mean "sexual characteristic"}. The third gender ({\em napuṃsaka}) was basically a class for things that were neither male nor female in nature.

This meant that there was an intimate connection between sex and grammatical gender that had far reaching consequences and caused much confusion\footnote{Other languages, like Uralic languages, do not have gender in language. Such genderless languages exclude many possibilities for reinforcement of gender-related stereotypes as they do not place objects (and thus people) in boxes.}. 

Note that the original meaning of {\em napuṃsaka} is "not a male", but the people this referred to were all males, just not conforming to gender role expectations. The word {\em napuṃsaka} itself retained it's masculine form in grammar.

\subsection{Characteristics of Gender}
The speculations and discussions that followed focussed around the characteristics necessary to identify a person as belonging to one of three groups. The {\em paṇḍaka}, {\em klība} and {\em keśavan} (long-haired male)\footnote{note that apparently long hair was seen as a sign of a woman} were recognized as males, but their gender role nonconformity assimilated them to females, so not "real males" and therefore still {\em napuṃsaka}. Yet their grammatical gender was still masculine.

This discussion was influenced by the Brahmanical views at that time concerning the essential markers for sex assignment ({\em liṅga}). By the third century BCE two views had developed to define gender.
\begin{enumerate}
 \item The first view went from the premise that gender was defined by what one perceived as a man, woman or neither based on the presence or absence of primary or secondary characteristics\footnote{Mahābhāṣya 4.1.3: Q: "What is it that people see when they decide, this is a woman, this is a man, this is neither woman nor man?", A: "That person who has breasts and long hair is a woman; that person who is hairy all over is a man; that person who is different from either when those characteristics are absent, is {\em napuṃsaka}."}.
 \item The second view is that gender assignment has to do with the ability to procreate or conceive. 
\end{enumerate}

Both the Brahmanical these views were rejected by the Jains as being inadequate to determine sex. \cite{dundas} describes how the Jains developed a system to define sex as a combination of sexual behavior, physical characteristics and also the underlying sexuality and feelings. The Jain came up with their own term {\em veda} to describe these characteristics\footnote{This move is rather remarkable because for the Brahmins {\em veda} meant their sacred knowledge and scriptures. But it is not unprecedented because the Jains often used existing words and gave them new meaning. In the Buddhist suttas we also find instances where the Buddhists use different terms for the same things as the Jains. Majjhima Nikāya 56 recounts a discussion between the Buddha and the Jain ascetic Tapassī in which the ascetic says: {\em “Na kho, āvuso gotama, āciṇṇaṃ nigaṇṭhassa nāṭaputtassa ‘kammaṃ, kamman’ti paññapetuṃ; ‘daṇḍaṃ, daṇḍan’ti kho, āvuso gotama, āciṇṇaṃ nigaṇṭhassa nāṭaputtassa paññapetun”ti.} “Reverend Gotama, Nigaṇṭha Nātaputta {\em (i.e. Mahāvīra)} doesn’t usually speak in terms of ‘deeds’ He usually speaks in terms of ‘rods’.” See also \cite{zwilling} note 34}.

This conception of sexuality most likely predates the schism between the two major Jain sects in the 5th century BC but was not part of the earliest Jain doctrine. This concept appears frequently in the later canonical Jain texts but is also mentioned once in the early Jain literature where male sexuality is explained as sexual desire for women and visa versa\footnote{See {\em Viyāha 2.5.1}}. The sexuality of the {\em napuṃsaka} is not clearly defined but is seen as a threat to the chastity of monks\footnote{See Ācārāṅga Sūtra (English translation \cite{jacobi}) p.220: monks are warned that a danger of drunkenness is seduction by a woman or a {\em klība}; p.285: sleeping places frequented by women or {\em paṇḍaka} are to be avoided}.

\cite{zwilling} mention:

\begin{quote}
From these passages we may infer that sexual desire for a man forms at least one aspect of third-sex sexuality. In a set of similes descriptive of the relative intensities of the sexualities of the three sexes, that of the third sex is viewed as most intense of all: a woman's {\em veda} is compared to a dung fire, a man's to a forest fire, but the third sex's is compared to a burning city. Thus third-sex persons are not only sexual persons, but hyperlibidinous ones at that.
\end{quote}

The word {\em napuṃsaka} has been subject to much debate within the Jain order, resulting over time in changes in meaning and use and definition of sub-categories. The word in the canonical texts seems to have referred only to males who were effeminate and transvestite, who are identified by the feminine way they dress, their behavior and sexual object choice. Because they looked female, their sexuality was also assumed as such. Because of this characterization the {\em napuṃsaka} can also be an object of lust for celibate monks. Part of the discussion was also fuelled by the nakedness of the Jain monks and therefore their physical male appearance as well as behavior. As celibate monks same-sex relations and the possibility of same-sex attractiveness were also an issue; the public perception, and the fear thereof, was of utmost importance for the livelihood of the Jain order. 

We also see a shift in the discussion over time about the abilities for a {\em napuṃsaka}, or at least some sub-categories thereof, to attain enlightenment or to ordain. The {\em Śvētāmbara} in their later Bhāgavatī Sūtra\footnote{Bhāgavatī Sūtra4.1–2} even define a fourth sex, namely the {\em puruṣanapuṃsaka} (male {\em napuṃsaka}, possibly a {\em napuṃsaka} who on the outside could "pass" as a regular male)\footnote{see \cite{zwilling} for more details.}. Lacking any of the outside characteristics of a {\em napuṃsaka}, the only characteristic left to define them as such must have been their sexuality (i.e. attraction to men).

The period of the commentarial literature redefined the sexuality of the {\em napuṃsaka} as being more bisexual in orientation. \cite{zwilling} believe that this new definition is not so much driven by actual observations of the behavior of {\em napuṃsaka} but rather by theoretical discussion. This bisexual orientation was not conceived of as a separate orientation, but as possessing the sexuality of \textbf{both} males and females together. This is a change from the canonical literature, where the sexuality of a {\em napuṃsaka} was characterized as female only.

The commentarial period continues to define the male and female {\em napuṃsaka} more clearly. The female {\em napuṃsaka} being the old category as defined in the canon of which the {\em klība} and {\em paṇḍaka} are sub-categories, the male {\em napuṃsaka} being the aforementioned {\em puruṣanapuṃsaka}. The female {\em napuṃsaka} seems to act as a female partner only (i.e. be acted upon), while the male {\em napuṃsaka} acts in both ways. So here male and female sexuality are no longer just defined as the sexual desire to have sex with a female and male resp. but also in terms of the role taken in intercourse as a penetrator or a receptor or both\footnote{Niśītha Sūtra 3507}.



\section{Jain Monastic Ordination}

In the formative years of the Jain order, the rules for ordination were still rather simple. Only the {\em klība}, the {\em paṇḍaka} and ill people were not allowed to ordain. Of the two Jain sects after the schism, the {\em Digambara} maintained nakedness and eligibility to ordain as a monk was quite straightforward; one had to be a man without genital defects and virile, except when he is overly libidinous. 

For the {\em Śvētāmbara}, who wore a cloth, the matter was far more complex and they devised an intricate system of ordainable categories, whereby the {\em napuṃsaka} was divided in 16 types, 10 of which were not allowed to be ordained, amongst which the original two categories. Most notably the aforementioned {\em puruṣanapuṃsaka} was allowed to ordain, presumably because these could not potentially evoke a monk's lust.




\section{Buddhist views on the third sex}
The Buddhist view as mentioned in the Abhidharma Kośa (IV.14c) (4–5th century CE) approximates that of the Brahmanical view that sex ({\em vyañjana}) is distinguised on the basis of primary and secondary sexual characteristics.

In the early Buddhist Pāli texts, we find remarkably little on the subject, but in later commentaries and Mahāyāna texts we find a number of recognized third sex types that are discussed in more detail, defined also based on their sexual behavior and not only on their external characteristics. As Buddhist monks did not go naked, and in fact identified themselves as different from Jains by the fact that they had a bowl and robe, this issue might have been less important than for the Jain.


(CLAIRE's ESSAY ON THIS)
(THE DISCUSSION BIT IN HERE)


From the statistical analysis in \ref{appendix2} we can see that the sanskrit texts follow the Pali in that no mention is made of {\em ubhayavyañjana} in the sutras. Just like in the pali, we only find the word in the later texts and Vinaya. What is more interesting is that the word is not found in the pre-Buddhist texts or later Brahmanical texts. This raises questions as to the origin of the word, of which not much seems to be known.

Considering that the literal meaning of the word {\em ubhayavyañjana} is "having the characteristics of both sexes"\footnote{\href{https://suttacentral.net/define/ubhatovya%C3%B1janaka}{The definition in the New Concise Pali English Dictionary} defines the term as "having the sexual characteristics of both sexes; hermaphrodite". But the literal meaning of the term says nothing about the characteristics needing to be sexual in origin; {\em ubhato} meaning "in both ways, on both sides" and {\em byañjana} or {\em vyañjana} means "sign or mark". As the term "hermaphrodite" is nowadays only used for species that can reproduce both as male and female, I discard this translation.}, we could postulate that the word is a synomym of {\em napuṃsaka} or at least a class of {\em napuṃsaka}. As we have also seen that the Buddhist texts seem to follow the Brahmanical idea of gender-characteristics or {\em liṅga}, this seems to underline this postulation.

Religious groups also formed their own vocabulary like for instance the the Jain did not use the Brahmanical standard term {\em liṅga} for gender-characteristics but introduced their own term {\em veda}. It is not unlikely that the Buddhists also formed their own vocabulary\footnote{In the suttas also we find instances where the Buddhists use different terms for the same things as the Jains. Majjhima Nikāya 56 recounts a discussion between the Buddha and the Jain ascetic Tapassī in which the ascetic says: {\em “Na kho, āvuso gotama, āciṇṇaṃ nigaṇṭhassa nāṭaputtassa ‘kammaṃ, kamman’ti paññapetuṃ; ‘daṇḍaṃ, daṇḍan’ti kho, āvuso gotama, āciṇṇaṃ nigaṇṭhassa nāṭaputtassa paññapetun”ti.} “Reverend Gotama, Nigaṇṭha Nātaputta {\em (i.e. Mahāvīra)} doesn’t usually speak in terms of ‘deeds’ He usually speaks in terms of ‘rods’.” }. and used the term {\em ubhayavyañjana} instead of {\em napuṃsaka}. In Buddhism we also find the term {\em napuṃsaka} in the commentarial and later texts only.


The {\em klība} is notable for it's absense in the Buddhist canon. Considering that this is defined as a man whose penis is destroyed, it could also have been taken as synomym of {\em paṇḍaka}, even though it is treated as different in the Sanskrit Vedas and Brahmanical texts.

linga also mainly appear in the commentarial texts in the pali.