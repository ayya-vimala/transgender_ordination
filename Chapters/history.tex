\section{The emergence of the Third Sex}

\subsection{Vedic period}
We can trace the emergence of the concept of a third sex back to the Vedic period (800-600 BCE). The term {\em napuṃsaka} was an umbrella term used to denote men who were impotent, effeminate or transvestite\footnote{\cite{zwilling}}. Literally the term means "not-a-male". Such "un-males" constituted a distinct social group with institutionalized roles like singers, dancers and prostitutes. The modern day {\em hijras} are the contemporary representatives of these {\em napuṃsaka}\footnote{A good study on the {\em hijras} is provided by \cite{nanda}}.

By the 6th century BCE, the word {\em napuṃsaka} was adopted as a gramatical third gender\footnote{{\em Śatapatha Brāhmaṇa (ŚB) 10.5.1.2-3}}, which resulted in these "un-males" to be regarded as persons with ambiguous sex\footnote{\cite{zwilling2000}}. 

\subsection{Jain period}
The concept of third sex was further refined in the Jain period, who discussed this to some extent and several other terms for referring to the members of this class were introduced. So next to the {\em napuṃsaka} we also see the emergence of the terms {\em klība} (sexually defective man\footnote{As pointed out by \cite{zwilling} the nature of the {\em klība} is suggested by the Bṛhadāraṇyaka Upaniṣad 6.1.12 and can be acquired due to the destruction of the penis as in ŚB 1.4.3.19}) and {\em paṇḍaka} ("impotent", "eunuch" or "sterile"\footnote{Atharva Veda (AV) 8.6.7, 11.16. The etymology of {\em paṇḍaka} is unknown but cf. baṇḍa at AV 7.65.3 is glossed by the commentator as {\em nirvīrya} ("impotent") \cite{zwilling}}).

\subsection{Common Era}
By the start of the common era, a term literally meaning "third sex" was introduced, possibly in the schools of traditional medicine, next to male and female ({\em tṛtīyāprakṛti}) and like the other two determined at conception by biological causes. \footnote{It is interesting to note that this third sex was seen as an equilibrium between male and female; it develops when the father's seed and the mother's blood are in equilibrium and the third sex also develops in the middle of the womb rather than on the right or left as with males or females. See footnote 16 in \cite{zwilling} for a detailed description.} In the 4th century CE the term had become synonymous to {\em napuṃsaka}.

In the fifth century CE the Jains used various terms like {\em tṛtīya} and {\em tṛairāśika} to refer to person of the third sex. 

\section{The Jain order and definition of sex}
Just like in Buddhism, the Jain order had a strong interest in controlling the sexuality of it's monastics. Jain monastics live celibate and at the time of it's emergence, the monks were mostly naked ascetics. The prestige and power of the order depended to a large extend on public opinion and therefore on the purity of their behavior, as well as their external appearance. The "third sex" was therefore subject of a very lengthy debate within the order. 

In addition to these practical considerations, there was a debate within the Jain community as to whether women can attain spiritual liberation because the monks felt it was improper for them to go naked. Eventually it was this dispute that led to the schism between the two major Jain orders\footnote{\cite{dudas}}\footnote{The two main sects of Jainism, the {\em Digambara} and the {\em Śvētāmbara} sect, likely started forming about the 3rd century BCE and the schism was complete by about 5th century CE.}. This controversy hinged on the identification of the signs to designate somebody as a man or woman or neither. 

What ensued was a lengthy discussion about the signs of (fe)male-ness. Several other terms were used to describe various forms of {\em napuṃsaka}, like the aforementioned {\em klība} and {\em keśavan} (long-haired male). Although they were still recognized as males, their gender role nonconformity assimilated them to females, so not "real males". 

This discussion was also influenced by the Brahmanical views at that time concerning the essential markers for sex assignment ({\em liṅga}). By the third century BCE two views had developed to define gender. The first view went from the premise that gender was defined by what one perceived as a man, woman or neither\footnote{Mahābhāṣya 4.1.3: Q: "What is it that people see when they decide, this is a woman, this is a man, this is neither woman nor man?", A: "That person who has breasts and long hair is a woman; that person who is hairy all over is a man; that person who is different from either when those characteristics are absent, is {\em napuṃsaka}."}. The second view is that gender assignment has to do with the ability to procreate or conceive. 

Both the Brahmanical these views were rejected by the Jains as being inadequate to determine sex. \cite{dundas} describes how the Jains developed a system to define sex as a combination of sexual behavior, physical characteristics and also the underlying sexuality motivation and feelings. This conception of sexuality most likely predates the schism between the two major Jain sects in the 5th century BC but was not part of the earliest Jain doctrine. This concept appears frequently in the later canonical Jain texts but is also mentioned once in the early Jain literature where male sexuality is explained as sexual desire for women and visa versa\footnote{See {\em Viyāha 2.5.1}}. The sexuality of the {\em napuṃsaka} is not clearly defined but is seen as a threat to the chastity of monks\footnote{See Ācārāṅga Sūtra (English translation \cite{jacobi}) p.220: monks are warned that a danger of drunkenness is seduction by a woman or a {\em klība}; p.285: sleeping places frequented by women or {\em paṇḍaka} are to be avoided}.

\cite{zwilling} mention:

\begin{quote}
From these passages we may infer that sexual desire for a man forms at least one aspect of third-sex sexuality. In a set of similes descriptive of the relative intensities of the sexualities of the three sexes, that of the third sex is viewed as most intense of all: a woman's {\em veda} is compared to a dung fire, a man's to a forest fire, but the third sex's is compared to a burning city. Thus third-sex persons are not only sexual persons, but hyperlibidinous ones at that.
\end{quote}

It seems that the word {\em napuṃsaka} has been subject to much debate within the Jain order, resulting over time in changes in meaning and use. In the canon we find many instances with ambiguities where it is not clear what a {\em napuṃsaka} really is and many questions are raised. Questions that have fueled the discussion and refined definitions and sub-categories. The {\em Śvētāmbara} in their later Bhāgavatī Sūtra\footnote{Bhāgavatī Sūtra4.1-2} even define a fourth sex, namely the {\em puruṣanapuṃsaka} (male {\em napuṃsaka}, possibly a {\em napuṃsaka} who on the outside could "pass" as a regular male)\footnote{see \cite{zwilling} for more details.}. 

In general the canon presents the {\em napuṃsaka} as a feminized male and therefore can also be an object of lust for celibate monks. Part of this discussion was also fuelled by the nakedness of the Jain monks and therefore their physical male appearance as well as behavior. As celibate monks same-sex relations and the possibility of same-sex attractiveness were also an issue; the public perception, and the fear thereof, was of utmost importance for the livelihood of the Jain order. We also see a shift in the discussion over time about the abilities for a {\em napuṃsaka}, or at least some sub-categories thereof, to attain enlightenment or to ordain.

While in the canonical literature the the sexuality of a {\em napuṃsaka} went unexamined in any explicit way, it is clear that this sexuality was essentially female in nature. The period of the commentarial literature redefined the sexuality of the {\em napuṃsaka} as being more bisexual in orientation. \cite{zwilling} believe that this new definition is not so much driven by actual observations of the behavior of {\em napuṃsaka} but rather by theoretical discussion, which also created the myth that {\em napuṃsaka} are hyperlibidious. In other words, these ideas were based on prejudices that were alive at that time. It was believed that having both 


\section{Jain monastic ordination}
In the formative years of the Jain order, the rules for ordination were still rather simple. Only the {\em klība}, the {\em paṇḍaka} and ill people were not allowed to ordain. Of the two Jain sects after the schism, the {\em Digambara} maintained nakedness and eligibility to ordain as a monk was quite straightforward; one had to be a man without genital defects and virile, except when he is overly libidinous. For the {\em Śvētāmbara}, who wore a cloth, the matter was far more complex and they devised an intricate system of ordainable categories, whereby the {\em napuṃsaka} was divided in 16 types, 10 of which were not allowed to be ordained, amongst which the original two categories. Most notably the aforementioned {\em puruṣanapuṃsaka} was allowed to ordain, presumably because these could not potentially evoke a monk's lust.


\section{Buddhist views on the third sex}
The Buddhist view as mentioned in the Abhidharma Kośa (IV.14c) approximates that of the Brahmanical view that sex ({\em vyañjana}) is distinguised on the basis of primary and secondary sexual characteristics.

In the early Buddhist Pāli texts, we find remarkably little on the subject, but in later commentaries and Mahāyāna texts we find a number of recognized third sex types that are discussed in more detail, defined also based on their sexual behavior and not only on their external characteristics. As Buddhist monks did not go naked, and in fact identified themselves as different from Jains by the fact that they had a bowl and robe, this issue might have been less important than for the Jain.


(CLAIRE's ESSAY ON THIS)
(THE DISCUSSION BIT IN HERE)


From the statistical analysis in \ref{appendix2} we can see that the sanskrit texts follow the Pali in that no mention is made of {\em ubhayavyañjana} in the sutras. Just like in the pali, we only find the word in the later texts and Vinaya. What is more interesting is that the word is not found in the pre-Buddhist texts or later Brahmanical texts. This raises questions as to the origin of the word, of which not much seems to be known.

Considering that the literal meaning of the word {\em ubhayavyañjana} is "having the characteristics of both sexes"\footnote{\href{https://suttacentral.net/define/ubhatovya%C3%B1janaka}{The definition in the New Concise Pali English Dictionary} defines the term as "having the sexual characteristics of both sexes; hermaphrodite". But the literal meaning of the term says nothing about the characteristics needing to be sexual in origin; {\em ubhaya} meaning "both, of both kinds" and {\em byañjana} or {\em vyañjana} means "sign or mark". As the term "hermaphrodite" is nowadays only used for species that can reproduce both as male and female, I discard this translation.}, we could postulate that the word is a synomym of {\em napuṃsaka} or at least a class of {\em napuṃsaka}. As we have also seen that the Buddhist texts seem to follow the Brahmanical idea of gender-characteristics or {\em liṅga}, this seems to underline this postulation.

Religious groups also formed their own vocabulary like for instance the the Jain did not use the Brahmanical standard term {\em liṅga} for gender-characteristics but introduced their own term {\em veda}. It is not unlikely that the Buddhists also formed their own vocabulary\footnote{In the suttas also we find instances where the Buddhists use different terms for the same things as the Jains. Majjhima Nikāya 56 recounts a discussion between the Buddha and the Jain ascetic Tapassī in which the ascetic says: {\em “Na kho, āvuso gotama, āciṇṇaṃ nigaṇṭhassa nāṭaputtassa ‘kammaṃ, kamman’ti paññapetuṃ; ‘daṇḍaṃ, daṇḍan’ti kho, āvuso gotama, āciṇṇaṃ nigaṇṭhassa nāṭaputtassa paññapetun”ti.} “Reverend Gotama, Nigaṇṭha Nātaputta {\em (i.e. Mahāvīra)} doesn’t usually speak in terms of ‘deeds’ He usually speaks in terms of ‘rods’.” }. and used the term {\em ubhayavyañjana} instead of {\em napuṃsaka}. In Buddhism we also find the term {\em napuṃsaka} in the commentarial and later texts only.


The {\em klība} is notable for it's absense in the Pali canon. Considering that this is defined as a man whose penis is destroyed, it could also have been taken as synomym of {\em paṇḍaka}.