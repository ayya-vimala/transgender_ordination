\section{Vedic, Brahmanic and Jain scriptures}

\subsection{Emergence of the Third Gender in Mythology}
We can trace the emergence of the concept of a third sex back to the late Vedic period (800–600 BCE). In the Vedic myths and legends, we frequently find the theme of a man turning into a woman or of being both a mother and a father. The function of these myths and legends is to confront deep anxieties and fears associated with the complex and problematic issues involving body, gender, sexuality, power, hierarchy and subordination. We see literary representations of these anxieties in all patriarchal societies, expressing the deeply ambivalent attitude towards women and women's sexuality. On the one hand, women are depicted as pure and nurturing as long as they are controlled within the constraints of kinship, but outside such regulated environment they are seen as dangerous and destructive to men. Through such projective devices of men unto women, male-dominated cultures have been able to establish a hegemonic ideology of gender. We see that transsexualism is a recurring theme in these myths and legends, derived from these anxieties and attitudes towards gender\footnote{\cite{goldman} gives an excellent account of the myths that formed the notions of gender and transsexualism in ancient India.}.

These myths have not stayed confined to mere story-telling, but have informed every aspect of life in India. The hijra\footnote{A good study on the {\em hijras} is provided by \cite{nanda}.} are a representation of Śiva in her form of Ardhanārīśwara. They enact out the religious myths and make them come alive. They are viewed as vehicles of the divine power of the Mother Goddess, which transforms their impotence into the power of generativity. These hijras are the contemporary representatives of these myths that have been around for thousands of years.

In practise, hijra are emasculated males. They call themselves 'not a male' or 'neither man nor women' ({\em napuṃsaka}). They dress in women's clothing, jewelry and make-up, sing and perform dances and religious rituals like baby-blessings. The origin myth of the hijra, just like that of most Indian castes, "explains" the origin of the caste, linking the caste to Hindu deities, providing religious sanction for its claimed place in Indian society. The myths validate a positive identity.

Unlike the other castes, one can not be born into the hijra caste as normal but men and boys are admitted into the fraternity from all other castes\footnote{\cite{ibbetson} believes all Hijra to be Muslims, also relating them to the Muslim tradition of keeping eunuchs as harem-guards, but \cite{nanda} points out that they have their own Hindu deities and religious rituals. The hijra caste seems to be something much older and very different from the harem-guards of the Muslim period.} and undergo an initiation rite. A key defining criterion of a hijra is that he is sexually impotent with women. This impotence is something he is born with and according to some accounts is tested before he is admitted into the caste\footnote{\cite{shah} and \cite{bhimbhai} Some 19th-century accounts report that impotence was an essential qualification for admission into the hijra community and that a newcomer initiated into the community was on probation for as long as a year. During this time his impotence was carefully tested, sometimes by making the person sleep four nights 'with a prostitute'. Only after impotence was established would the newcomer be permitted to undergo the emasculation operation and become a full member of the community. \cite{preston} mentions that another 9th-century account of the hijras also reports that "all state that they were incapable of copulation and that becoming [hijras] was on that account only"}.

The view as hijra as 'not a man' begins with their being men who are impotent from birth, and therefore not 'real men'. But they are not considered hijra until the moment they are initiated i.e. emasculated. According to \cite{nanda}, their anatomy is not the only characteristic of a hijra, but also their physiology and their sexual capacities, feelings, preferences and behaviors. Although dressing like a woman is part of a hijra, they are also something quite different from a transvestite; they are the religious embodiment of the deities. Although at first the hijra do not seem to have been prostitutes, over the centuries prostitution has appeared among them.

The term {\em napuṃsaka} was an umbrella term used to denote such men who were impotent, effeminate or dressing in traditional women's clothing\footnote{\cite{zwilling}}. Literally the term means 'not-a-male' i.e. men who did not conform to gender-role expectations. The adoption of the word {\em napuṃsaka} as a grammatical third gender\footnote{{\em Śatapatha Brāhmaṇa (ŚB) 10.5.1.2–3}} in the 6th century BCE seems to have prompted a significant shift in meaning. Because now the {\em napuṃsaka} was interpreted s meaning 'neither male nor female'. This resulted in the previously mentioned 'un-males' to be regarded as persons with ambiguous sex\footnote{\cite{zwilling2000}}. The individuals that the word {\em napuṃsaka} referred to were however all males, just not conforming to gender role expectations. The word {\em napuṃsaka} itself retained it's masculine form in grammar.

The fact that Sanskrit is a gendered language forced people to assign gender to all objects including all living creatures and humans. Gender was seen as a property belonging to objects and objects are gendered by the presence or absence of certain defining characteristics or {\em liṅga}\footnote{The original meaning of {\em liṅga} is 'characteristic mark or sign' (Nirukta 1.17) but later starts to mean 'sexual characteristic'}. The third gender ({\em napuṃsaka}) was basically a class for things that were neither male nor female in nature. This meant that there was an intimate connection between sex and grammatical gender that had far reaching consequences and caused much confusion\footnote{Other languages, like Uralic languages, do not have gender in language. Such gender-less languages exclude many possibilities for reinforcement of gender-related stereotypes as they do not place objects (and thus people) in boxes.}. 

Just after the late Vedic period we see that a set of terms relating to the class of {\em napuṃsaka} has emerged like {\em klība} (sexually defective man\footnote{As pointed out by \cite{zwilling} the nature of the {\em klība} is suggested by the Bṛhadāraṇyaka Upaniṣad 6.1.12 and can be acquired due to the destruction of the penis as in ŚB 1.4.3.19}) and {\em paṇḍaka} ('impotent', or 'sterile'\footnote{Atharva Veda (AV) 8.6.7, 11.16. The etymology of {\em paṇḍaka} is unknown but cf. baṇḍa at AV 7.65.3 is glossed by the commentator as {\em nirvīrya} ('impotent') \cite{zwilling}. Albrecht Wezler has suggested that paṇḍa and paṇḍaka be regarded as ultimately derived from {\em *apa}+ {\em āṇḍā}, thus: “one who has no testicles (anymore),” designating, if used as a substantive, a “eunuch,” etc. \cite{wezler}}). Both of these types were associated with transvestism and dancing. With the word {\em napuṃsaka} having gained a much broader meaning, it seems likely that these new subcategories represent different names for the original meaning of 'un-males' and therefore what we now know as hijra, or at least something very similar, with the {\em paṇḍaka} being a hijra in general, and a {\em klība} one who has undergone the initiation rite.


\subsection{Sex and Gender in the Jain Order}
Just like in Buddhism, the Jain order had a strong interest in controlling the sexuality of it's monastics. Jain monastics live celibate and at the time of it's emergence, the monks were mostly naked ascetics. The prestige and power of the order depended to a large extend on public opinion and therefore on the purity of their behavior, as well as their external appearance. The 'third sex' was therefore subject of a very lengthy debate within the order. 

In addition to these practical considerations, there was a debate within the Jain community as to whether women can attain spiritual liberation because the monks felt it was improper for them to go naked. Eventually it was this dispute that led to the schism between the two major Jain orders\footnote{\cite{dudas}}\footnote{The two main sects of Jainism, the {\em Digambara} and the {\em Śvētāmbara} sect, likely started forming about the 3rd century BCE and the schism was complete by about 5th century CE.}. This controversy hinged on the identification of the signs to designate somebody as a woman, which logically also led to the examination of what is male, and neither male nor female. 

The speculations and discussions that followed focused around the characteristics necessary to identify a person as belonging to one of three groups. The {\em paṇḍaka}, {\em klība} and {\em keśavan} (long-haired male)\footnote{note that apparently long hair was seen as a sign of a woman} were recognized as males, but their gender role nonconformity assimilated them to females, so not 'real males' and therefore still {\em napuṃsaka}. Yet their grammatical gender was still masculine.

This discussion was influenced by the Brahmanical views at that time concerning the essential markers for sex assignment ({\em liṅga}). By the third century BCE two views had developed to define gender.
\begin{enumerate}
 \item The first view went from the premise that gender was defined by what one perceived as a man, woman or neither based on the presence or absence of primary or secondary characteristics\footnote{Mahābhāṣya 4.1.3: Q: "What is it that people see when they decide, this is a woman, this is a man, this is neither woman nor man?", A: "That person who has breasts and long hair is a woman; that person who is hairy all over is a man; that person who is different from either when those characteristics are absent, is {\em napuṃsaka}."}.
 \item The second view is that gender assignment has to do with the ability to procreate or conceive. 
\end{enumerate}

Both these Brahmanical views were rejected by the Jains as being inadequate to determine sex. \cite{dundas} describes how the Jains developed a system to define gender as a combination of sex, sexual behavior, physical characteristics and also the underlying sexuality and feelings. The Jain came up with their own term {\em veda} to describe these characteristics\footnote{This move is rather remarkable because for the Brahmins {\em veda} meant their sacred knowledge and scriptures. But it is not unprecedented because the Jains often used existing words and gave them new meaning. In the Buddhist suttas we also find instances where the Buddhists use different terms for the same things as the Jains. Majjhima Nikāya 56 recounts a discussion between the Buddha and the Jain ascetic Tapassī in which the ascetic says: {\em “Na kho, āvuso gotama, āciṇṇaṃ nigaṇṭhassa nāṭaputtassa ‘kammaṃ, kamman’ti paññapetuṃ; ‘daṇḍaṃ, daṇḍan’ti kho, āvuso gotama, āciṇṇaṃ nigaṇṭhassa nāṭaputtassa paññapetun”ti.} “Reverend Gotama, Nigaṇṭha Nātaputta {\em (i.e. Mahāvīra)} doesn’t usually speak in terms of ‘deeds’ He usually speaks in terms of ‘rods’.” See also \cite{zwilling} note 34}. This conception of sexuality most likely predates the schism between the two major Jain sects in the 5th century BC but was not part of the earliest Jain doctrine. This concept appears frequently in the later canonical Jain texts but is also mentioned once in the early Jain literature where male sexuality is explained as sexual desire for women and visa versa\footnote{See {\em Viyāha 2.5.1}}. The sexuality of the {\em napuṃsaka} is not clearly defined in the earlier texts but is seen as a threat to the chastity of monks\footnote{See Ācārāṅga Sūtra (English translation \cite{jacobi}) p.220: monks are warned that a danger of drunkenness is seduction by a woman or a {\em klība}; p.285: sleeping places frequented by women or {\em paṇḍaka} are to be avoided}.

\cite{zwilling} mention:

\begin{quote}
... we may infer that sexual desire for a man forms at least one aspect of third-sex sexuality. In a set of similes descriptive of the relative intensities of the sexualities of the three sexes, that of the third sex is viewed as most intense of all: a woman's {\em veda} is compared to a dung fire, a man's to a forest fire, but the third sex's is compared to a burning city. Thus third-sex persons are not only sexual persons, but hyperlibidinous ones at that.
\end{quote}

The word {\em napuṃsaka} has been subject to much debate within the Jain order, resulting over time in changes in meaning and use and definition of sub-categories. The word in the canonical texts seems to have referred only to males who were effeminate and transvestite, who are identified by the feminine way they dress, their behavior and sexual object choice. Because they looked female, their sexuality was also assumed as such. Because of this characterization the {\em napuṃsaka} can also be an object of lust for celibate monks. If we compare this with the aforementioned hijra, it seems likely that their feminine behavior, also before their initiation rite, was seen as problematic for ordination as a male monastic. Part of the discussion was also fuelled by the nakedness of the Jain monks and therefore their physical male appearance as well as behavior. As celibate monks same-sex relations and the possibility of same-sex attractiveness were also an issue; the public perception, and the fear thereof, was of utmost importance for the livelihood of the Jain order. 

We also see a shift in the discussion over time about the abilities for a {\em napuṃsaka}, or at least some sub-categories thereof, to attain enlightenment or to ordain. The {\em Śvētāmbara} in their later Bhāgavatī Sūtra\footnote{Bhāgavatī Sūtra4.1–2} even define a fourth sex, namely the {\em puruṣanapuṃsaka} (male {\em napuṃsaka}, possibly a {\em napuṃsaka} who on the outside could "pass" as a regular male)\footnote{see \cite{zwilling} for more details.}. Lacking any of the outside characteristics of a {\em napuṃsaka}, the only characteristic left to define them as such must have been their sexuality (i.e. attraction to men).

The period of the commentarial literature redefined the sexuality of the {\em napuṃsaka} as being more bisexual in orientation. \cite{zwilling} believe that this new definition is not so much driven by actual observations of the behavior of {\em napuṃsaka} but rather by theoretical discussion. This bisexual orientation was not conceived of as a separate orientation, but as possessing the sexuality of \textbf{both} males and females together. This is a change from the canonical literature, where the sexuality of a {\em napuṃsaka} was characterized as female only.

The commentarial period continues to define the male and female {\em napuṃsaka} more clearly. The female {\em napuṃsaka} being the old category as defined in the canon of which the {\em klība} and {\em paṇḍaka} are sub-categories, the male {\em napuṃsaka} being the aforementioned {\em puruṣanapuṃsaka}. The female {\em napuṃsaka} seems to act as a female partner only (i.e. be acted upon), while the male {\em napuṃsaka} acts in both ways. So here male and female sexuality are no longer just defined as the sexual desire to have sex with a female and male resp. but also in terms of the role taken in intercourse as a penetrator or a receptor or both\footnote{Niśītha Sūtra 3507}. The hyperlibidinous nature of the {\em napuṃsaka} was ascribed to the bisexual character of his sexuality.

It is interesting to note that throughout this discussion the {\em napuṃsaka} and it's subcategories were males who are somehow blocked in their exercise of their male sexuality in one way or the other owing to their performance of some unvirtuous act (karma) in a previous life. Females who did not conform gender expectations were not considered in the class of {\em napuṃsaka} or are only very rarely mentioned, without much explanation as to their nature. 

\subsection{Jain Monastic Ordination}
In the formative years of the Jain order, the rules for ordination were still rather simple. Only the {\em klība}, the {\em paṇḍaka} and ill people were not allowed to ordain. Of the two Jain sects after the schism, the {\em Digambara} maintained nakedness and eligibility to ordain as a monk was quite straightforward; one had to be a man without genital defects and virile, except when he is overly libidinous. 

For the {\em Śvētāmbara}, who wore a cloth, the matter was far more complex and they devised an intricate system of ordainable categories, whereby the {\em napuṃsaka} was divided in sixteen types\footnote{See Bhagavatī 5166–67}. Over time, the ban against ordination of {\em napuṃsaka} was relaxed more, first based on practical grounds like a known and well-behaved candidate, later an exception was made for those who were able to control their sexuality. One of the main grounds why certain {\em napuṃsaka} were denied ordination was their perceived hyperlibidinous-ness, which would render them incapable of keeping their celibate vows and made them unfit to live in either the monks or the nuns communities. Only ten of the sixteen were not allowed to be ordained because they were regarded as uncontrollable in their passions. Amongst these were the original two categories of {\em klība} and {\em paṇḍaka}. The aforementioned {\em puruṣanapuṃsaka} was allowed to ordain, presumably because these could not potentially evoke a monk's lust. Since outside appearance was no longer a clear guide to who is {\em napuṃsaka}, the candidate for ordination had to be questioned. 

By the 17th century CE, this rule on ordination had been nearly abolished. So we have seen a radical shift from total nonacceptance to nearly total acceptance of {\em napuṃsaka} in the Jain order over time\footnote{\cite{zwilling} references {\em Yuktiprabodha} in footnote 80}.



