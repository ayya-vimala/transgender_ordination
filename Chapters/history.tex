\section{Vedic, Brahmanic and Jain scriptures}

Various authors already noticed the similarities of Vinaya terminology between the Buddhist and Jain orders\footnote{See \cite{maes2016} page 9 footnotes 26–28, also \cite{sujato2009} and \cite{zwilling}.}. In the Buddhist suttas we find many examples of discussions between the two groups and after the Buddha passed away such discussions would certainly have had an impact on the Buddhist Vinaya. Both groups will also have used vocabulary that was already in existence at that time. It is therefore important to first have a look at what gender meant in the larger context of the society in pre-Buddhist India and how this understanding developed within the Jain order.

\subsection{Emergence of the Third Gender}
We can trace the emergence of the concept of a third sex back to the late Vedic period (800–600 BCE). In the Vedic myths and legends, we frequently find the theme of a man turning into a woman or of being both a mother and a father. The function of these myths and legends is to confront deep anxieties and fears associated with the complex and problematic issues involving body, gender, sexuality, power, hierarchy and subordination. We see literary representations of these anxieties in all patriarchal societies, expressing the deeply ambivalent attitude towards women and women's sexuality. On the one hand, women are depicted as pure and nurturing as long as they are controlled within the constraints of kinship, but outside such regulated environment they are seen as dangerous and destructive to men. Through such projective devices of men unto women, male-dominated cultures have been able to establish a hegemonic ideology of gender\footnote{See \cite{sujato2011} for an extensive work on the role of the male/female relationship portrayed in mythology in Buddhism and more specifically with regards to women's ordination.}. We see that 'transgenderism' is a recurring theme in these myths and legends, derived from these anxieties and attitudes towards gender\footnote{\cite{goldman} gives an excellent account of the myths that formed the notions of gender in ancient India.}.

The term {\em napuṃsaka} was originally an umbrella term used to denote such men who were impotent, with a female gender-expression or dressing in traditional women's clothing\footnote{\cite{zwilling} page 362.}. Literally the term means 'not-a-male' i.e. men who did not conform to gender-role expectations. 

It is accepted by many scholars that the {\em hijras} of India are the contemporary representatives of these {\em napuṃsakas} and they even refer to themselves as 'not-a-male' or 'neither male nor female'\footnote{See \cite{zwilling} page 363, \cite{goldman} page 388 and \cite{wendy}. A good study on the {\em hijras} is provided by Serena \cite{nanda}.}. Although the {\em hijras} have been subject to influences from the Muslim period in the 12th to 16th century as well as from Christian and more modern western ideas, by studying them we get a glimpse of what the lives of the {\em napuṃsakas} must have been like in Vedic times. 

The {\em hijras} are a representation of the god Śiva in their androgynous form of Ardhanārīśwara from Vedic mythology. They enact the religious myths and make them come alive in the rituals, songs and dances they perform. They are viewed as vehicles of the divine power of the Mother Goddess, which transforms their impotence into the power of generativity. According to sources from the beginning of he 20th century, potential candidates for initiation as a {\em hijra} (which includes emasculation) had to be born impotent\footnote{The 19th and early 20th century sources are important because at that time the influence of globalization and the influx of western ideas on the lives of the {\em hijra} was less prevalent as it is today. See \cite{ibbetson} pages 332, \cite{shah} page 1325 and \cite{bhimbhai}. Some 19th-century accounts report that impotence was an essential qualification for admission into the hijra community and that a newcomer initiated into the community was on probation for as long as a year. During this time his impotence was carefully tested, sometimes by making the person sleep four nights 'with a prostitute'. Only after impotence was established would the newcomer be permitted to undergo the emasculation operation and become a full member of the community. \cite{preston}.}. Their emasculated body is not the only characteristic of a {\em hijra}, but also their physiology and their sexual capacities, feelings, preferences and behaviors. Although dressing like a woman is part of a {\em hijra}, they are also something quite different from a transgender person; they are the religious embodiment of the deities\footnote{See \cite{nanda}.}.

The adoption of the word {\em napuṃsaka} as a grammatical third gender\footnote{{\em Śatapatha Brāhmaṇa (ŚB) 10.5.1.2–3}.} in the 6th century BCE seems to have prompted a significant shift in meaning. Because now the {\em napuṃsaka} was interpreted as meaning 'neither male nor female'. This resulted in the previously mentioned 'un-males' to be regarded as persons with ambiguous gender\footnote{See \cite{zwilling2000} page 362.}. The individuals that the word {\em napuṃsaka} originally referred to were however all males, just not conforming to gender role expectations. The word {\em napuṃsaka} itself retained it's masculine form in grammar.

The fact that Sanskrit is a gendered language forced people to assign gender to all objects including all living beings and humans. Gender was seen as a property belonging to objects and objects are gendered by the presence or absence of certain defining characteristics or {\em liṅga}\footnote{The original meaning of {\em liṅga} is 'characteristic mark or sign' (Nirukta 1.17) but later starts to mean 'sexual characteristic'. See also chapter \ref{linga} for a detailed description of {\em liṅga}.}. The third gender ({\em napuṃsaka}) was basically a class for things that were neither male nor female in nature. This meant that there was an intimate connection between sex and grammatical gender that had far reaching consequences and caused much confusion\footnote{Other languages, like the Uralic languages, do not have gender in language. Such gender-less languages exclude many possibilities for reinforcement of gender-related stereotypes as they do not place objects (and thus people) in boxes.}. 

Just after the late Vedic period we see that a set of terms relating to the class of {\em napuṃsaka} has emerged like {\em klība} (sexually defective man\footnote{As pointed out by \cite{zwilling} page 363, the nature of the {\em klība} is suggested by the Bṛhadāraṇyaka Upaniṣad 6.1.12 and can be acquired due to the destruction of the penis as in ŚB 1.4.3.19}.) and {\em paṇḍaka} ('impotent', or 'sterile'\footnote{Atharva Veda (AV) 8.6.7, 11.16. The etymology of {\em paṇḍaka} is unknown but cf. baṇḍa at AV 7.65.3 is glossed by the commentator as {\em nirvīrya} ('impotent'). Albrecht \cite{wezler} has suggested that paṇḍa and paṇḍaka be regarded as ultimately derived from {\em *apa}+ {\em āṇḍā}, thus: “one who has no testicles (anymore),” i.e. a eunuch. Allan \cite{bomhard} points out that there is a range of translations and interpretations that can apply to the {\em paṇḍaka}. He believes that the word can be a loan-word from the Dravidian {\em peṇṭan, peṇṭakan, peṇṭakam}, which can mean both hermaphrodite and eunuch.}). Both of these types were associated with cross-dressing and dancing, as we have also seen in the above description of the contemporary {\em hijras}. With the word {\em napuṃsaka} having gained a much broader meaning, it seems likely that these new sub-categories represent different names for the original meaning of 'un-males'.

\subsection{Sex and Gender in the Jain Order}
Just like in Buddhism, the Jain order had a strong interest in controlling the sexuality of it's monastics. Jain monastics live celibate and at the time of its emergence, the monks were mostly naked ascetics. The prestige and power of the order depended to a large extend on public opinion and therefore on the purity of their behavior, as well as their external appearance. The 'third sex' was therefore subject of a very lengthy debate within the order. 

In addition to these practical considerations, there was a debate within the Jain community as to whether women can attain spiritual liberation because the monks felt it was improper for them to go naked. Eventually it was this dispute that led to the schism between the two major Jain orders\footnote{See Paul \cite{dudas} for an extensive account of the schism. The two main sects of Jainism, the {\em Digambara} and the {\em Śvētāmbara} sect, likely started forming about the 3rd century BCE and the schism was complete by about 5th century CE.}. This controversy hinged on the identification of the signs to designate somebody as a woman, which logically also led to the examination of what is male, and 'neither male nor female'. 

The speculations and discussions that followed focused around the characteristics necessary to identify a person as belonging to one of three groups. The {\em paṇḍaka}, {\em klība} and {\em keśavan} (long-haired male)\footnote{Apparently long hair was seen as a sign of a woman.} were recognized as males, but their gender role nonconformity assimilated them to females, so not 'real males' and therefore still {\em napuṃsaka}. Yet their grammatical gender was masculine.

This discussion was influenced by the Brahmanical views at that time concerning the essential markers for sex assignment ({\em liṅga}). By the third century BCE two views had developed to define gender\footnote{I will discuss the meaning of this term in more detail in chapter \ref{linga}.}.
\begin{enumerate}
 \item The first view went from the premise that gender was defined by what one perceived as a man, woman or neither based on the presence or absence of primary or secondary characteristics\footnote{Mahābhāṣya 4.1.3: Q: "What is it that people see when they decide, this is a woman, this is a man, this is neither woman nor man?", A: "That person who has breasts and long hair is a woman; that person who is hairy all over is a man; that person who is different from either when those characteristics are absent, is {\em napuṃsaka}."}.
 \item The second view is that gender assignment has to do with the ability to procreate or conceive. 
\end{enumerate}

Both these Brahmanical views were rejected by the Jains as being inadequate to determine sex. Paul \cite{dundas} describes how the Jains developed a system to define gender as a combination of sex, behavior, physical characteristics and also the underlying sexuality and feelings. The Jain came up with their own term {\em veda} to describe these characteristics\footnote{This move is rather remarkable because for the Brahmins {\em veda} meant their sacred knowledge and scriptures. But it is not unprecedented because the Jains often used existing words and gave them new meaning. In the Buddhist suttas we also find instances where the Buddhists use different terms for the same things as the Jains. F.i. Majjhima Nikāya 56 recounts a discussion between the Buddha and the Jain ascetic Tapassī in which the ascetic mentions that their leader Nigaṇṭha Nātaputta {\em (i.e. Mahāvīra)} doesn’t speak in terms of ‘deeds’, like the Buddha, but uses the term ‘rods’.” See also \cite{zwilling} note 34.}. This conception of sexuality most likely predates the schism between the two major Jain sects but was not part of the earliest Jain doctrine. This concept appears frequently in the later canonical Jain texts but is also mentioned once in the early Jain literature where male sexuality is explained as sexual desire for women and visa versa\footnote{See {\em Viyāha 2.5.1}.}. The sexuality of the {\em napuṃsaka} is not clearly defined in the earlier texts but is seen as a threat to the chastity of monks\footnote{See Ācārāṅga Sūtra (English translation by \cite{jacobi}) p.220: monks are warned that a danger of drunkenness is seduction by a woman or a {\em klība}; p.285: sleeping places frequented by women or {\em paṇḍaka} are to be avoided.}.

Zwilling and Sweet\footnote{See \cite{zwilling} page 368.} mention:

\begin{quote}
... we may infer that sexual desire for a man forms at least one aspect of third-sex sexuality. In a set of similes descriptive of the relative intensities of the sexualities of the three sexes, that of the third sex is viewed as most intense of all: a woman's {\em veda} is compared to a dung fire, a man's to a forest fire, but the third sex's is compared to a burning city. Thus third-sex persons are not only sexual persons, but hyperlibidinous ones at that.
\end{quote}

The word {\em napuṃsaka} has been subject to much debate within the Jain order, resulting over time in changes in meaning and use and the definition of sub-categories. The word in the canonical texts seems to have referred only to males who were cross-dressing and had a female gender-expression, who are identified by the way they dress, their behavior and sexual object choice. Because they looked like a woman, their sexuality was also assumed as such. Because of this characterization the {\em napuṃsaka} can also be an object of lust for celibate monks and would have been seen as problematic for ordination as a male monastic. Part of the discussion was also fuelled by the nakedness of the Jain monks and therefore their physical male appearance as well as behavior. As celibate monks same-sex relations and the possibility of same-sex attractiveness were also an issue; the public perception, and the fear thereof, was of utmost importance for the livelihood of the Jain order. 

We also see a shift in the discussion over time about the abilities for a {\em napuṃsaka}, or at least some sub-categories thereof, to attain enlightenment or to ordain. The {\em Śvētāmbara} in their later Bhāgavatī Sūtra\footnote{Bhāgavatī Sūtra 4.1–2.} even define a fourth sex, namely the {\em puruṣanapuṃsaka} (male {\em napuṃsaka}, possibly a {\em napuṃsaka} who appeared the same as other men)\footnote{see \cite{zwilling} for more details.}. Lacking any of the outside characteristics of a {\em napuṃsaka}, the only characteristic left to define them as such must have been their sexuality (i.e. attraction to men).

The period of the commentarial literature redefined the sexuality of the {\em napuṃsaka} as being more bisexual in orientation. Leonard Zwilling and Michael Sweet\footnote{See \cite{zwilling} pages 371–374.} believe that this new definition is not so much driven by actual observations of the behavior of {\em napuṃsaka} but rather by theoretical discussion. This bisexual orientation was not conceived of as a separate orientation, but as possessing the sexuality of \textbf{both} males and females together. This is a change from the canonical literature, where the sexuality of a {\em napuṃsaka} was characterized as female only.

The commentarial period continues to (re)define the male and female {\em napuṃsaka}. The female {\em napuṃsaka} being the old category as defined in the canon of which the {\em klība} and {\em paṇḍaka} are sub-categories, the male {\em napuṃsaka} being the aforementioned {\em puruṣanapuṃsaka}. The female {\em napuṃsaka} seems to act as a female partner only (i.e. be acted upon), while the male {\em napuṃsaka} acts in both ways. So here male and female sexuality are no longer just defined as the sexual desire to have sex with a female and male resp. but also in terms of the role taken in intercourse as a penetrator or a receptor or both\footnote{Niśītha Sūtra 3507.}. The hyperlibidinous nature of the {\em napuṃsaka} was ascribed to the bisexual character of his sexuality.

It is interesting to note that throughout this discussion the {\em napuṃsaka} and it's sub-categories always refer to males who are somehow blocked in their exercise of their male sexuality in one way or the other owing to their performance of some unvirtuous act (karma) in a previous life. Females who did not conform gender expectations were not considered in the class of {\em napuṃsaka} or are only very rarely mentioned, without much explanation as to their nature. 

\subsection{Jain Monastic Ordination}
In the formative years of the Jain order, the rules for ordination were still rather simple. Only the {\em klība}, the {\em paṇḍaka} and ill people were not allowed to ordain. Of the two Jain sects after the schism, the {\em Digambara} maintained nakedness and eligibility to ordain as a monk was quite straightforward; one had to be a man without genital defects and virile, except when he is overly libidinous. 

For the {\em Śvētāmbara}, who wore a cloth, the matter was far more complex and they devised an intricate system of ordainable categories, whereby the {\em napuṃsaka} was divided in sixteen types\footnote{See Bhagavatī 5166–67.}. Over time, the ban against ordination of {\em napuṃsaka} was relaxed more, first based on practical grounds like a known and well-behaved candidate, later an exception was made for those who were able to control their sexuality. One of the main grounds why certain {\em napuṃsaka} were denied ordination was their perceived hyperlibidinous-ness, which would render them incapable of keeping their celibate vows and made them unfit to live in either the monks or the nuns communities. Only ten of the sixteen were not allowed to be ordained because they were regarded as uncontrollable in their passions. Amongst these were the original two categories of {\em klība} and {\em paṇḍaka}. The aforementioned {\em puruṣanapuṃsaka} was allowed to ordain, presumably because these could not potentially evoke a monk's lust. Since outside appearance was no longer a clear guide to who is {\em napuṃsaka}, the candidate for ordination had to be questioned. 

By the 17th century CE, this rule on ordination had been nearly abolished. So we have seen a radical shift from total nonacceptance to nearly total acceptance of {\em napuṃsaka} in the Jain order over time\footnote{\cite{zwilling} references {\em Yuktiprabodha} in footnote 80.}.


