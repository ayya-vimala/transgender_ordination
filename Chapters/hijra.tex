\section{Hijra}

Before we continue, I want to explore the cultural and social meanings of sex and gender in ancient India. Although we do not have very detailed accounts on this topic, from the Buddhist scriptures we know a lot about how people lived in India 2500 years ago. This was a world that was not too different from parts of India we know today, or more accurately, the India we know from colonial records and studies.

This is just a summary of the article on hijra.

Hijra are an alternative gender, neither men or women. The origin myth of the Hijra, just like that of most Indian castes, "explains" the origin of the caste, linking the caste to Hindu deities, providing religious sanction for its claimed place in Indian society. The myths validate the a positive identity. 

The word Hijra comes from Urdu can either mean "eunuch" or "hermaphrodite". Both of these terms, as used in India, connote impotence i.e. an inability to function in the male sexual role. The word Hijra primarily implies a physical defecit impairing the male sexual function (Opler, 1960)

OPLER, M.
1960 The hijara (hermaphrodites) of India and Indian national character: A rejoinder. American Anthropologist, 62, 505–511.

In both cases the irregularity of the male genitalia is central to the definition. ... Although historically in Norther India a linguistic distinction was made between "born hijras" (intersex) and "made hijras" (eunuchs) (Ibbetson et all, 1911), the term hijra as it is currently used collapses both of these categories.

IBBETSON, D. C. J., MACLAGEN, M. E., & ROSE, H. A.
1911 A glossary of the tribes and castes of the Punjab and Nortb-West Frontier Province (vol. 2, pp. 331–333). Lahore: Civil and Military Gazette Press.
https://archive.org/details/glossaryoftribes03rose/page/n689/mode/2up?q=hijra
volume 2 page 332 mentions alternative words for Hijra and the Khoja is one of them
Also employed as harem-guards by the beginning of the 20th century.
Ibbetson describes various forms of similar behavior, the Hijra being eunuchs (kliba?) but also working as harem-guards and ritual performers. But they are not prostitutes.
There are other forms that are not castrated and work as singers/dancers and/orprostitutes, they are said to be impotent.

Since 1911 the term hijra became an umbrella term for all of these.

In the 19th century, an initiate had to be impotent to classify to become a hijra. 
So impotence is a pre-requisite to become hijra.
Although dressing like a woman is part of a hijra, it is something quite different from a transvestite.

We go into the house of all, and never has a eunuch looked upon a woman with a bad eye; we are like bullocks [castrated male cattle].

As indicated by this quote (Ibbetson et al., 1911:331), the view of hijras as an "in-between" gender begins with their being men who are impotent, therefore not men. But being impotent is
only a necessary and not sufficient condition for being a hijra. Hijras are men who are impotent for one reason or another and only become hijras by having their genitals cut off

The hijra view of themselves as "not men" as it occurred in my conversations with them focused primarily on their anatomy, but also implicated their physiology and their sexual capacities, feelings, and preferences. These definitions incorporated both the ascribed status of "being born this way" and the achieved status of renouncing sexual desire and sexual activity.

We hijras are born as boys, but then we'get spoiled' and have sexual desires only for men."