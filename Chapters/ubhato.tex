\section{Ubhatob­yañ­janaka}

For the {\em ubhatob­yañ­janaka} we have less material to go on as for the {\em paṇḍaka}. It is only briefly mentioned in the Chinese Vinayas as those with two roots/faculties (二根) who are not allowed to ordain, but without any further explanation. The Therāvada Vinaya merely states that this person "acted and was acted upon". 

All the Vinayas agree that the {\em ubhatob­yañ­janaka}/二根 is one of the four sex/gender types next male, female, {\em paṇḍaka}/黃門. Considering that the male and female were seen as both having just one root/faculty, and the {\em paṇḍaka} in the meaning of 'eunuch' has none\footnote{Note that when the {\em paṇḍaka} appears in the texts in the list of these four sex/gender types, it is in the Chinese Vinayas always described with the characters 黃門 ('eneuch') and never as 種不能男 ('impotent')}, the two-facultied person fills a gap. This could indicate a philosophical position using the {\em catuskoti}\footnote{Dr. M. Vermeulen, to be published book on this subject}.

The commentarial literature is slightly more forthcoming but this does not give us much more clarity as to our understanding of the {\em ubhatob­yañ­janaka}.



-----------------------------------------------

The Buddhist view as mentioned in the Abhidharma Kośa (IV.14c) (4–5th century CE) approximates that of the Brahmanical view that sex ({\em vyañjana}) is distinguised on the basis of primary and secondary sexual characteristics.

It is clear that {\em ubhatob­yañ­janaka} is translated as 二根 i.e. having two roots/faculties but without explanation in the Vinaya. 

二根者 or simply 二根 (2 roots/faculties) is translated as hermaphrodite in the dictionary, but no description, nor any stories are given in any of the schools. The passage for {\em ubhatob­yañ­janaka} as we see in the Theravāda Vinaya is not there. It is simply mentioned in passing that these are also not allowed to ordain. The word {\em ubhatob­yañ­janaka} (i.e. one with both signs) seems to be consistent with 二根.

With regards to ubhatovyañjanaka the scriptures are more consistent: The Vinaya's only mention they cannot ordain (always in conjuction with the word 黃門 (yellow gate)), the commentaries all say the same thing with minor variations: There are three kinds of 2-facultied people: those who can impregnate and conceive; those who can impregnate but not conceive; and those who cannot impregnate but who can conceive. These three types of people are not allowed to become monks and take the full precepts; if they have already taken the full precepts, they should be expelled.It feels like the  ubhatovyañjanaka is simply something to fill a gap, without quite understanding what it could be or who it could refer to. It seems more theoretical than anything else. And the conjuction with the Yellow Gate would indicate it is a late introduction also.

Dīrghāgama Sutta T24 describes how at first all beings in all the heavens were subject to marriage between men and women. But then they were bestowed with the gift that:
"Henceforth all the heavens above shall be free from all marriages, so that there shall be no distinction between male and female. Bhikkhus, the Asuras, the Heavens of the Four Great Kings, the heaven of the Thirty-three, when they so desire they can have the same faculties (i.e. 二根 as above), and the same aura comes out of them, just like the Dragons, the Garudas, and there will be no difference between them.


sanskrit vinaya often speaks of strīpuruṣapaṇḍakam in the Varṣāvastu

The commentary makes a distinction between male and female {\em ubhatob­yañ­janaka} whereby characteristics of the other sex are hidden. This is a much broader definition and a much broader subset of the term "intersex" as we know it today. The fact that they are predominantly male or female would be a fairly objective basis for deciding on ordination with the Bhikkhus or Bhikkhunis. This distinction is not mentioned in the Vinaya. But the commentary also makes a distinction between {\em pabbajjā} and {\em upasampadā} and does not allow either for ordination.


The Vinaya recognizes 4 main sex/gender types (check PJ1): Male, female, pandaka and ubhatobyanjanaka. 
Jackson mentioned that Bunmi Methangkun (1986) observes that psychological as well as physiological factors are involved in the consitution of the ubhatobyanjanaka. This article is in Thai.
jackson observes (without reference) that in early Buddhist communities men who engage in receptive anal sex are seen as feminized and thought to be hermaphrodites.



This word appears in various chapters of the Vinaya Khandaka, but never in the early Buddhist Suttas, nor in the pātimokkhas. It appears in the Milindapañha, the questions of a Greek king and therefore a much later text, even goes as far as to mention that a {\em paṇḍaka } or a {\em ubhatob­yañ­janaka } cannot attain enlightenment but there does not seem to be a basis for this assertion in any of the Early Buddhist texts. This seems to be added later at the end of a list of those who cannot attain enlightenment which is found elsewhere in the canon.


\cite{goldman} In the story of King Ila as told in the {\em Uttarakanda} of the {\em Rāmāyaṇam}\footnote{Rām 7.78-79}, the king accidentally stumbles upon the Mother Goddess in intimate embrace with Śiva, who turn him into a woman. Now Ilā, she turns to the Goddess for mercy to restore her manhood but is only granted half her wish; namely that she has to change sex each month. With the change of sex also comes a change in sexual desire. As a woman she falls in love, becomes pregnant and gives birth, reverting back and forth between male and female. 

From the statistical analysis in \ref{appendix2} we can see that the sanskrit texts follow the Pāli in that no mention is made of {\em ubhayavyañjana} in the sutras. Just like in the pali, we only find the word in the later texts and Vinaya. What is more interesting is that the word is not found in the pre-Buddhist texts or later Brahmanical texts. This raises questions as to the origin of the word, of which not much seems to be known.

Considering that the literal meaning of the word {\em ubhayavyañjana} is "having the characteristics of both sexes"\footnote{\href{https://suttacentral.net/define/ubhatovya%C3%B1janaka}{The definition in the New Concise Pāli English Dictionary} defines the term as "having the sexual characteristics of both sexes; hermaphrodite". But the literal meaning of the term says nothing about the characteristics needing to be sexual in origin; {\em ubhato} meaning "in both ways, on both sides" and {\em byañjana} or {\em vyañjana} means "sign or mark". As the term "hermaphrodite" is nowadays only used for species that can reproduce both as male and female, I discard this translation.}, we could postulate that the word is a synomym of {\em napuṃsaka} or at least a class of {\em napuṃsaka}. As we have also seen that the Buddhist texts seem to follow the Brahmanical idea of gender-characteristics or {\em liṅga}, this seems to underline this postulation.


The Dharmaguptaka Bhikkhunī Khandhaka also mentions that women with no breasts should not be given ordination.


Traditionally, the word {\em ubhatob­yañ­janaka } was translated as "hermaphrodite" as is mentioned in the Pāli Text Society dictionary. But the word "hermaphrodite" has shifted in meaning over the last decades with a greater understanding of the complex variations of natural bodies as well as the understanding that human beings are never able to be fully hermaphrodite i.e. can never reproduce both as male and female but are always dominant in one. Ajahn Brahmali uses the word "intersex" as as translation.


Just as with the word {\em paṇḍaka }, it seems that this word has shifted in meaning over time to mean a more lustful individual (\cite{vimala}).