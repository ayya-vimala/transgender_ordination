\section{Definitions}
\subsection{Modern definitions}
\subsubsection{Intersex}
The definition of the term "intersex" according to the \cite{un2015} is as follows:

\begin{quote}
Intersex people are born with sex characteristics (including genitals, gonads and chromosome patterns) that do not fit typical binary notions of male or female bodies.

Intersex is an umbrella term used to describe a wide range of natural bodily variations. In some cases, intersex traits are visible at birth while in others, they are not apparent until puberty. Some chromosomal intersex variations may not be physically apparent at all.
\end{quote}

Intersex can be divided into 4 categories according to the \cite{nlm}:

\begin{tabular}{ l l }
46, XX intersex & female internal organs and chromosomes \\
& external genitals appear male \\
46, XY intersex & male internal organs and chromosomes \\
& external genitals appear female or ambiguous \\
True gonadal intersex & both ovarian and testicular tissue \\
& external genitals ambiguous or \\
& appear female or male \\
Complex or undetermined intersex & chromosomes discrepancies only \\
\end{tabular}


\subsubsection{Hermaphrodite}
A hermaphrodite is an organism that has both male and female reproductive organs. Until the mid-20th century, "hermaphrodite" was used synonymously with "intersex". The distinctions "male pseudohermaphrodite", "female pseudohermaphrodite" and especially "true hermaphrodite" are terms no longer used, which reflected histology (microscopic appearance) of the gonads. Medical terminology has shifted not only due to concerns about language, but also a shift to understandings based on genetics.

Currently, hermaphroditism is not to be confused with intersex, as the former refers only to a specific phenotypical presentation of sex organs and the latter to a more complex combination of phenotypical and genotypical presentation. Using hermaphrodite to refer to intersex individuals is considered to be stigmatizing and misleading (\cite{is2013}). Hermaphrodite is used for animal and plant species in which the possession of both ovaries and testes is either serial or concurrent, and for living organisms without such gonads but present binary form of reproduction, which is part of the typical life history of those species; intersex has come to be used when this is not the case.


\subsubsection{Transgender}
Transgender people have a gender identity or gender expression that differs from the sex that they are assigned at birth (\cite{altilio}). Some transgender people who desire medical assistance to transition from one sex to another identify as transsexual (\cite{polly}). Transgender, often shortened as trans, is also an umbrella term. In addition to including people whose gender identity is the opposite of their assigned sex (trans men and trans women), it may include people who are not exclusively masculine or feminine (people who are non-binary or genderqueer, including bigender, pangender, genderfluid, or agender). Other definitions of transgender also include people who belong to a third gender, or else conceptualize transgender people as a third gender.

The term transgender is also distinguished from intersex. 

The opposite of transgender is cisgender, which describes persons whose gender identity or expression matches their assigned sex.

Many transgender people experience gender dysphoria, and some seek medical treatments such as hormone replacement therapy, sex reassignment surgery, or psychotherapy. Not all transgender people desire these treatments, and some cannot undergo them for financial or medical reasons. (\cite{maizes})


\subsection{Scriptural definitions}

\subsubsection{Paṇḍaka}

Pali word: {\em paṇḍaka} \\
Pali dictionary: \href{https://suttacentral.net/define/pa%E1%B9%87%E1%B8%8Daka}{see SuttaCentral} \\
Sanskrit word: {\em paṇḍaka} \\
Tibetan word: {\em ma ning} or {\em ’dod ’gro} \\
Chinese word: {\em 非男非女} or {\em 半擇} \\
ITLR dictionary: \href{http://www.itlr.net/hwid:281142}{itlr.net} \\

\medskip

The term {\em paṇḍaka } and it's female form {\em itthipaṇḍaka } have been discussed at length in the last years. The strictest readings of the word as used in the Buddhist world simply use this term very loosly to mean all LGBTIQA+, but this is a much too simple perspective, which also is very hurtful for many people. At the time when the Vinaya was laid down, the word most likely meant "eunuch" (\cite{vimala}). However, the meaning has shifted over time and the commentaries describe an individual who has a much higher libido than others.

{\em Itthipaṇḍaka } occurs far less in the canon and it's meaning is not quite sure because a female eunuch seems like something that cannot exist The word only appears in the later additions to the Vinaya.

The word is not found in any of the early Buddhist Suttas, nor does it appear in the pātimokkhas, the lists of rules for monastics. Next to the pāli Vinaya, it appears twice in the Aṅguttara Nikāya, but both of these only have parallels to the Vinaya or later texts.


\subsubsection{Ubhatob­yañ­janaka}

Pali word: {\em ubhatob­yañ­janaka} or {\em ubhatovyañ­janaka} \\
Pali dictionary: \href{https://suttacentral.net/define/ubhatovya%C3%B1janaka}{see SuttaCentral} \\
Sanskrit word: {\em ubhayavyañjana} \\
Tibetan word: {\em mtshan gnyis pa} \\
ITLR dictionary: \href{http://www.itlr.net/hwid:62844}{itlr.net} \\

\medskip

Traditionally, the word {\em ubhatob­yañ­janaka } was translated as "hermaphrodite" as is mentioned in the Pali Text Society dictionary. But the word "hermaphrodite" has shifted in meaning over the last decades with a greater understanding of the complex variations of natural bodies as well as the understanding that human beings are never able to be fully hermaphrodite i.e. can never reproduce both as male and female but are always dominant in one. Ajahn Brahmali uses the word "intersex" as as translation.

It would probably more accurate to refer to this as a more narowly defined subset of intersex rather than all intersex people. I would propose a subset of true gonadal intersex but will come back to this later in this paper.

Just as with the word {\em paṇḍaka }, it seems that this word has shifted in meaning over time to mean a more lustful individual (\cite{vimala}).

This word appears in various chapters of the Vinaya Khandaka, but never in the early Buddhist Suttas, nor in the pātimokkhas. It appears in the Milindapañha, the questions of a Greek king and therefore a much later text, even goes as far as to mention that a {\em paṇḍaka } or a {\em ubhatob­yañ­janaka } cannot attain enlightenment but there does not seem to be a basis for this assertion in any of the Early Buddhist texts. This seems to be added later at the end of a list of those who cannot attain enlightenment which is found elsewhere in the canon.


\subsubsection{Other references}
There are various other words mentioned in the ordination procedures for Bhikkhuni as described in Bhikkhunikkhandhaka that might be interesting in this context. These do not excluse from ordination and have been translated by Ajahn Brahmali as follows: \\

\begin{tabular}{ l l }
 {\em animittā } & woman who lacks genitals \\
 {\em nimittamattā } & woman with incomplete genitals \\ 
 {\em vepurisikā } & woman who is manlike \\
\end{tabular} \\

The word {\em animittā } literally means "signless" and appears a number of times in the canon (excluding commentaries) but mostly in a different meaning, namely as in {\em Animitto (ceto)samādhi }, which is translated by Bhikkhu Sujato as "signless immersion", a term used in the context of meditation. In the context of not having genitals, it only appears in the canon in the Bhikkhunikkhandhaka and as a form of abuse for women in the Bhikkhu Saṃ­ghā­di­sesa­ 3, never on it's own but always in the same sequence of words of which the above are a few. This points towards a later development.

The other words mentioned here also only appear in these two places in the canon and always exclusively refer to women. The words {\em (itthi)paṇḍaka } and {\em ubhatob­yañ­janaka } also appear in the same sequence.

The reason I mention these words here is that although they do not exclude a candidate from ordination, these categories would be seen as various forms of intersex in our modern understanding. Yet they are denoted separately in the context of Bhikkhuni ordination. 


\subsubsection{Napuṃsaka}

Pali word: {\em napuṃsaka} \\
Pali dictionary: \href{https://suttacentral.net/define/napu%E1%B9%83saka}{see SuttaCentral} \\
Sanskrit word: {\em napuṃsaka} \\
Sanskrit dictionary: \href{https://www.wisdomlib.org/definition/napumsaka}{see WisdomLib} \\

\medskip


The term {\em napuṃsaka} does not appear in the Pali Suttas, Vinaya or Abhidhamma but it appears extensively in the later commentaries, especially the Anya. 

\subsubsection{Tṛtīyāprakṛti}

{\em tṛtīyāprakṛti}


\subsection{A note on translation and interpretation}
The main responsibility of a translator is to translate words as accurately as possible according to the meaning at the time the text was laid down. In practise this can be challenging as meanings of words have shifted over time, both in the ancient past as they have in our times. Words used 100 years ago might not be used any more in the same way they are today. It is clear from the above definitions that no modern word can clearly capture the understanding of the words in the canon entirely and it remain a question of interpretation. It is therefore important that we carefully study the socio-cultural conditions under which these words have formed and take these conditions into consideration when making decisions about the lives of others, most notably future candidates for ordination.


----------------------------------------------------------------
In cases such as this, we should always err on the side of compassion. If there is no clear rule in the Vinaya, are we obliged to follow the commentarial interpretation? Of course there is always the danger that an ordination is not accepted in the wider Sangha, but with the Bhikkhuni ordinations we have seen that acceptence gradually grows. It is up to the individual communities, with agreement of all Sangha members, to ordain a person or not and to judge their suitability, even if the interpretation of other communities is different. 



-----------------------------------------------------------------



Another route to interpreting the texts is, as always, comparative study. It may turn out that the prohibition against ubhatobyañjanakas is a relatively late development that happened after the time of the Buddha. If so, we would have good grounds for disregarding this prohibition.

There are further complications, such as getting agreement from all Sangha members to ordain such a person. It is possible not everyone would feel at ease with it, for a number of possible reasons.
