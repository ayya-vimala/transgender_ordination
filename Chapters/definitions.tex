\section{Glossary of Definitions}
\label{appendix3}

\subsection{Definitions of Pali and Sanskrit Words}
In this section I refer to the various dictionary definitions of the words relevant to the subject matter and provide links to these dictionaries. Click on a website link to open the definition in your browser.

\begin{multicols}{2}
\subsubsection*{Napuṁsaka}
Pali word: \textit{napuṁsaka}\\
Pali dictionary: \href{https://suttacentral.net/define/napu%E1%B9%83saka}{see SuttaCentral.net}\\
Sanskrit word: \textit{napuṁsaka}\\
Sanskrit dictionary: \href{https://www.wisdomlib.org/definition/napumsaka}{see WisdomLib.org}
\medskip

\subsubsection*{Paṇḍaka}
Pali word: \textit{paṇḍaka}\\
Pali dictionary: \href{https://suttacentral.net/define/pa%E1%B9%87%E1%B8%8Daka}{see SuttaCentral.net}\\
Sanskrit word: \textit{paṇḍaka}\\
Tibetan word: \textit{ma ning} or \textit{'dod 'gro}\\
Chinese word: \textit{不能男} or \textit{黃門}\\
ITLR dictionary: \href{http://www.itlr.net/hwid:281142}{see ITLR.net}
\medskip

\subsubsection*{Ubhatob­yañ­janaka}
Pali word: \textit{ubhatob­yañ­janaka} or 

\textit{ubhatovyañ­janaka}\\
Pali dictionary: \href{https://suttacentral.net/define/ubhatovya%C3%B1janaka}{see SuttaCentral.net}\\
Sanskrit word: \textit{ubhayavyañjana}\\
Tibetan word: \textit{mtshan gnyis pa}\\
Chinese word: \textit{二根} or \textit{二形}\\
ITLR dictionary: \href{http://www.itlr.net/hwid:62844}{see ITLR.net}
\medskip

\subsubsection*{Ṣaṇḍha}
Sanskrit word: \textit{ṣaṇḍha}\\
Tibetan word: \textit{za ma} or \textit{nyug rum}\\
ITLR dictionary: \href{https://www.itlr.net/hwid:281142?md=view&m2=op1&usid=281142}{see ITLR.net}
\vfill\null
\columnbreak

\subsubsection*{Vepurisikā}
Pali word: \textit{vepurisikā}\\
Pali dictionary: \href{https://suttacentral.net/define/vepurisik%C4%81}{see SuttaCentral.net}

\subsubsection*{Liṅga}
Pali word: \textit{liṅga}\\
Pali dictionary: \href{https://suttacentral.net/define/li%E1%B9%85ga}{see SuttaCentral.net}\\
Sanskrit word: \textit{liṅga}\\
Chinese word: \textit{根} or \textit{形}

\subsubsection*{Byañ­jana}
Pali word: \textit{b­yañ­jana} or \textit{vyañjana}\\
Pali dictionary: \href{https://suttacentral.net/define/li%E1%B9%85ga}{see SuttaCentral.net}\\
Sanskrit word: \textit{vyañjana}\\
ITLR dictionary: \href{https://www.itlr.net/hwid:281142?md=view&m2=op1&usid=281142}{see ITLR.net}

\subsubsection*{Nimitta}
Pali word: \textit{nimitta}\\
Pali dictionary: \href{https://suttacentral.net/define/nimitta}{see SuttaCentral.net}\\
Sanskrit word: \textit{nimitta}\\
Tibetan word: \textit{mtshan ma}\\
ITLR dictionary: \href{https://www.itlr.net/hwid:281142?md=view&m2=op1&usid=281142}{see ITLR.net}
\end{multicols}

\subsection{Modern Definitions}
In this section I list a few terms relevant to the subject matter because there are many misunderstandings with regards to these terms and their meanings. For other terms, I refer to \href{https://www.hrc.org/resources/glossary-of-terms}{the website of the Human Rights Campaign}.

\subsubsection*{Intersex}
\label{intersex}

The definition of the term `intersex' according to the \href{https://unfe.org/system/unfe-65-Intersex_Factsheet_ENGLISH.pdf}{UN Office of the High Commissioner for Human Rights} is as follows:

\begin{quote}
Intersex people are born with sex characteristics (including genitals, gonads and chromosome patterns) that do not fit typical binary notions of male or female bodies.

Intersex is an umbrella term used to describe a wide range of natural bodily variations. In some cases, intersex traits are visible at birth while in others, they are not apparent until puberty. Some chromosomal intersex variations may not be physically apparent at all.
\end{quote}

Intersex can be divided into four categories according to the \href{https://medlineplus.gov/ency/article/001669.htm}{US National Library of Medicine}:

\begin{tabular}{ l l }
46, XX intersex & female internal organs and chromosomes\\
& external genitals appear male\\
46, XY intersex & male internal organs and chromosomes\\
& external genitals appear female or ambiguous\\
True gonadal intersex & both ovarian and testicular tissue\\
& external genitals ambiguous or\\
& appear female or male\\
Complex or undetermined intersex & chromosomes discrepancies only\\
\end{tabular}

\subsubsection*{Hermaphrodite}
\label{hermaphrodite}

A hermaphrodite is an organism that has both male and female reproductive organs. Until the mid-20\textsuperscript{th} Century, `hermaphrodite' was used synonymously with `intersex'. The distinctions `male pseudohermaphrodite', `female pseudohermaphrodite' and especially `true hermaphrodite' are terms that are no longer used, which reflected histology (microscopic appearance) of the gonads. Medical terminology has shifted not only due to concerns about language, but also a shift to understandings based on genetics.

Currently, hermaphroditism is not to be confused with intersex, as the former refers only to a specific phenotypical presentation of sex organs and the latter to a more complex combination of phenotypical and genotypical presentation. Using hermaphrodite to refer to intersex individuals is considered to be stigmatizing and misleading.\footnote{See \href{https://web.archive.org/web/20130701061246/http://www.isna.org/faq/hermaphrodite}{Intersex Society of North America}.} Hermaphrodite is used for animal and plant species in which the possession of both ovaries and testes is either serial or concurrent, and for living organisms without such gonads but which present binary form of reproduction, which is part of the typical life history of those species. Intersex has come to be used when this is not the case.

\subsubsection*{Transgender}

Transgender people have a gender identity or gender expression that differs from the sex that they were assigned at birth.\footnote{\cite{altilio} page 380.} Some transgender people who desire medical assistance to transition from one sex to another identify as transsexual.\footnote{\cite{polly}.} Transgender, often shortened as trans, is also an umbrella term. In addition to including people whose gender identity is the opposite of their assigned sex (trans men and trans women), it may include people who are not exclusively masculine or feminine (people who are non-binary or genderqueer, including bigender, pangender, genderfluid, or agender). Other definitions of transgender also include people who belong to a third gender, or else conceptualize transgender people as a third gender.

The term transgender is also distinguished from intersex. 

The opposite of transgender is cisgender, which describes persons whose gender identity or expression matches their assigned sex.

Many transgender people experience gender dysphoria, and some seek medical treatments such as Hormone Replacement Therapy, sex reassignment surgery, or psychotherapy. Not all transgender people desire these treatments, and some cannot undergo them for financial or medical reasons.\footnote{For more information on these issues, see \cite{maizes}.}
