\section{Analysis of the terms in the context of ordination}

The rule against ordination of both the {\em paṇḍaka } and the {\em ubhatob­yañ­janaka } are laid down in Khandaka 1 and are clearly against the full ordination of these two types of individuals, the {\em upasampadā}. Yet it is interesting that in both cases the person to whom the rule applies is said to have been given the {\em pabbajjā}. This really only makes sense if we understand {\em pabbajjā} here to be equivalent to {\em upasampadā}. In fact this equivalence between {\em pabbajjā} and {\em upasampadā} is what we find throughout the earliest Vinaya, and indeed the suttas \footnote{The {\em sāmaṇeras/īs} are barely mentioned in the suttas. Instead we find the figure of the {\em samaṇuddesa}, "one designated as a {\em samaṇa}", who seems to have had a looser affiliation with the Sangha, that is, no proper ordination. The commentaries glosses them as {\em sāmaṇeras}, but this might be an oversimplification. More likely they were a kind of precursor to the more formal status of novice. It seems likely that such people merely put on robes, and then lived in with loose connection to a particular community of ascetics, in which case their sex would have been a non-issue. I would argue it is natural to see novices proper in the same way. But the {\em samaṇuddesa} remains obscure.}. In any case, the rules itself are clearly limited to {\em upasampadā}.

\subsection{Paṇḍaka}

\begin{quote}
{\em Tena kho pana samayena aññataro paṇḍako bhikkhūsu \textbf{pabbajito} hoti. So dahare dahare bhikkhū upasaṅkamitvā evaṃ vadeti—“etha, maṃ āyasmanto dūsethā”ti. Bhikkhū apasādenti—“nassa, paṇḍaka, vinassa, paṇḍaka, ko tayā attho”ti. So bhikkhūhi apasādito mahante mahante moḷigalle sāmaṇere upasaṅkamitvā evaṃ vadeti—“etha, maṃ āvuso dūsethā”ti. Sāmaṇerā apasādenti—“nassa, paṇḍaka, vinassa, paṇḍaka, ko tayā attho”ti. So sāmaṇerehi apasādito hatthibhaṇḍe assabhaṇḍe upasaṅkamitvā evaṃ vadeti—“etha, maṃ āvuso dūsethā”ti. Hatthibhaṇḍā assabhaṇḍā dūsesuṃ. Te ujjhāyanti khiyyanti vipācenti—“paṇḍakā ime samaṇā sakyaputtiyā. Yepi imesaṃ na paṇḍakā, tepi ime paṇḍake dūsenti. Evaṃ ime sabbeva abrahmacārino”ti. Assosuṃ kho bhikkhū tesaṃ hatthibhaṇḍānaṃ assabhaṇḍānaṃ ujjhāyantānaṃ khiyyantānaṃ vipācentānaṃ. Atha kho te bhikkhū bhagavato etamatthaṃ ārocesuṃ. “Paṇḍako, bhikkhave, anupasampanno na \textbf{upasampādetabbo}, upasampanno nāsetabbo”ti. (Mahakkhandhaka, PTS 1.86)}
\end{quote}

\medskip

\begin{quote}
At one time a certain {\em paṇḍaka} had gone forth as a monk. He approached the young monks and said, “Venerables, come and have sex with me.”
The monks dismissed him, “Go away, {\em paṇḍaka}. Who needs you?”
\end{quote}
\begin{quote}
He went to the big and fat novices, said the same thing, and got the same response.
He then went to the elephant keepers and horse keepers, and once again he said the same thing. And they had sex with him. They complained and criticized them, “These Sakyan ascetics are {\em paṇḍaka}s. And those who are not have sex with them. None of them is celibate.”
\end{quote}
\begin{quote}
The monks heard their complaints. They told the Buddha and he said, “A {\em paṇḍaka} should not be given the full ordination. If it has been given, he should be expelled.”
\end{quote}



\subsection{Ubhatob­yañ­janaka}

\begin{quote}
{\em Tena kho pana samayena aññataro ubhatobyañjanako bhikkhūsu \textbf{pabbajito} hoti. So karotipi kārāpetipi. Bhagavato etamatthaṃ ārocesuṃ. Ubhatobyañjanako, bhikkhave, anupasampanno na upasampādetabbo, \textbf{upasampanno} nāsetabboti. (Mahakkhandhaka, PTS 1.89)}
\end{quote}

\medskip

\begin{quote}
At one time an {\em ubhatob­yañ­janaka} had gone forth as a monk. He had sex and made others have it.
\end{quote}
\begin{quote}
They told the Buddha and he said, “An {\em ubhatob­yañ­janaka} should not be given the full ordination. If it has been given, he should be expelled.”
\end{quote}

The commentary {\em (Samantapādādikā, vol. 3, para. 116)} mentions the following about {\em ubhatob­yañ­janaka}:

\begin{quote}
Ubhatobyañjanako ... so duvidho hoti – itthiubhatobyañjanako, purisaubhatobyañjanakoti. Tattha itthiubhatobyañjanakassa itthinimittaṃ pākaṭaṃ hoti, purisanimittaṃ paṭicchannaṃ. Purisaubhatobyañjanakassa purisanimittaṃ pākaṭaṃ, itthinimittaṃ paṭicchannaṃ. ... Imassa pana duvidhassāpi ubhatobyañjanakassa neva pabbajjā atthi, na upasampadāti
\end{quote}

\medskip

\begin{quote}
The {\em ubhatob­yañ­janaka} is of two sorts: the female {\em ubhatob­yañ­janaka} and the male {\em ubhatob­yañ­janaka}. For the female {\em ubhatob­yañ­janaka} the female characteristics are revealed, whereas the male characteristics are concealed. For the male {\em ubhatob­yañ­janaka} the male characteristics are revealed, whereas the female characteristics are concealed. ... 
\end{quote}
\begin{quote}
For either of these there is neither a going forth ({\em pabbajjā}) nor a full ordination ({\em upasampadā}).
\end{quote}

The commentary makes a distinction between male and female {\em ubhatob­yañ­janaka} whereby characteristics of the other sex are hidden. This is a much broader definition and a much broader subset of the term "intersex" as we know it today. The fact that they are predominantly male or female would be a fairly objective basis for deciding on ordination with the Bhikkhus or Bhikkhunis. This distinction is not mentioned in the Vinaya. But the commentary also makes a distinction between {\em pabbajjā} and {\em upasampadā} and does not allow either for ordination.





-------------------------------------------
One thing that struck me in Khandaka 1 is that both passages for paṇḍaka and ubhatobyañjanako deal with monks (already ordained!) having sex and therefore breaking Parajika 1. But instead of being rightfully expelled on those grounds for their bad behavior, they are expelled for who they ARE, together will a whole lot of other monks (at that time and in the future), who might be examplarery practitioners. The only other example of that in KD1 is animals but the origin story of that also seems rather more mythological than anything that actually happened. All the other examples of people not to be ordained are those who have such heavy defilements that they cannot either practise properly or would be a burden to the community because they might fall into the same bad habits again. So those people are not accepted because of what they have DONE. Having a certain body can hardly be regarded as a defilement, unless you look at it in the light of the idea of "past bad kamma". However, the Buddha never punished anybody for the way they were born.
This difference only makes sense if you read the translations according to a later interpretation of the words paṇḍaka and ubhatobyañjanako, namely people who have extremely strong sexual desire as I pointed out in my earlier article on pandakas. In fact, a eunuch has far less sexual desire as he no longer produces testosterone and there is no indication that an intersex person would have more sexual desire than other people. 



  "atk-vin02a20:323": "Ubhatobyañjanakavatthukathā",
  "atk-vin02a20:324": "116.Ubhatobyañjanako bhikkhaveti itthinimittuppādanakammato ca purisanimittuppādanakammato ca ubhato byañjanamassa atthīti ubhatobyañjanako. Karotīti purisanimittena itthīsu methunavītikkamaṁ karoti. Kārāpetīti paraṁ samādapetvā attano itthinimitte kārāpeti, so duvidho hoti – itthiubhatobyañjanako, purisaubhatobyañjanakoti.",
  "atk-vin02a20:325": "Tattha itthiubhatobyañjanakassa itthinimittaṁ pākaṭaṁ hoti, purisanimittaṁ paṭicchannaṁ. Purisaubhatobyañjanakassa purisanimittaṁ pākaṭaṁ, itthinimittaṁ paṭicchannaṁ. Itthiubhatobyañjanakassa itthīsu purisattaṁ karontassa itthinimittaṁ paṭicchannaṁ hoti, purisanimittaṁ pākaṭaṁ hoti. Purisaubhatobyañjanakassa purisānaṁ itthibhāvaṁ upagacchantassa purisanimittaṁ paṭicchannaṁ hoti, itthinimittaṁ pākaṭaṁ hoti. Itthiubhatobyañjanako sayañca gabbhaṁ gaṇhāti, parañca gaṇhāpeti. Purisaubhatobyañjanako pana sayaṁ na gaṇhāti, paraṁ gaṇhāpetīti, idametesaṁ nānākaraṇaṁ. Kurundiyaṁ pana vuttaṁ – ‘‘yadi paṭisandhiyaṁ purisaliṅgaṁ pavatte itthiliṅgaṁ nibbattati, yadi paṭisandhiyaṁ itthiliṅgaṁ pavatte purisaliṅgaṁ nibbattatī’’ti . Tattha vicāraṇakkamo vitthārato aṭṭhasāliniyā dhammasaṅgahaṭṭhakathāya veditabbo. Imassa pana duvidhassāpi ubhatobyañjanakassa neva pabbajjā atthi, na upasampadāti idamidha veditabbaṁ.",
  "atk-vin02a20:326": "Ubhatobyajjanakavatthukathā niṭṭhitā.",

Samantapasadika
Ubhatobyañjanako bhikkhaveti itthinimittuppādanakammato ca purisanimittuppādanakammato ca ubhato byañjanamassa atthīti ubhatobyañjanako.Karotīti purisanimittena itthīsu methunavītikkamaṃ karoti. Kārāpetīti paraṃ samādapetvā attano itthinimitte kārāpeti, so duvidho hoti – itthiubhatobyañjanako, purisaubhatobyañjanakoti.Tattha itthiubhatobyañjanakassa itthinimittaṃ pākaṭaṃ hoti, purisanimittaṃ paṭicchannaṃ. Purisaubhatobyañjanakassa purisanimittaṃ pākaṭaṃ, itthinimittaṃ paṭicchannaṃ. Itthiubhatobyañjanakassa itthīsu purisattaṃ karontassa itthinimittaṃ paṭicchannaṃ hoti, purisanimittaṃ pākaṭaṃ hoti. Purisaubhatobyañjanakassa purisānaṃ itthibhāvaṃ upagacchantassa purisanimittaṃ paṭicchannaṃ hoti, itthinimittaṃ pākaṭaṃ hoti. Itthiubhatobyañjanako sayañca gabbhaṃ gaṇhāti, parañca gaṇhāpeti. Purisaubhatobyañjanako pana sayaṃ na gaṇhāti, paraṃ gaṇhāpetīti, idametesaṃ nānākaraṇaṃ. 

Because of kamma giving rise to female characteristics and kamma giving rise to male characteristics, there is for them the characteristics of both. With the male characteristic they act to transgress through sexual intercourse with women. Having encouraged another, they cause action in their own female characteristic. 

[My comment: This seems to refer to a true hermaphrodite, assuming that such people even exist. The positive thing about this interpretation is that I am guessing very few people would be barred from ordaining.] 

They are twofold: the female ubhatobyañjanaka and the male ubhatobyañjanaka. In regard to this, the female characteristic of the female ubhatobyañjanaka is apparent, but the male characteristic is hidden. The male characteristic of the male ubhatobyañjanaka is apparent, but the female characteristic is hidden. 

While the female ubhatobyañjanaka is acting with manliness among women, the female characteristic is hidden, whereas the male characteristic is apparent. 
When the male ubhatobyañjanaka enters the state of a woman for the sake of men, the male characteristic is hidden, whereas the female characteristic is apparent. 
The female ubhatobyañjanaka becomes pregnant and causes others to become pregnant. The male ubhatobyañjanaka does not become pregnant, but causes others to become pregnant. This is the difference between them."



  "atk-vin02a20:277": "Paṇḍakavatthukathā",
  "atk-vin02a20:278": "109.Dahare dahareti taruṇe taruṇe. Moḷigalleti thūlasarīre. Hatthibhaṇḍe assabhaṇḍeti hatthigopake ca assagopake ca.",
  "atk-vin02a20:279": "Paṇḍakobhikkhaveti ettha āsittapaṇḍako usūyapaṇḍako opakkamikapaṇḍako pakkhapaṇḍako napuṁsakapaṇḍakoti pañca paṇḍakā. Tattha yassa paresaṁ aṅgajātaṁ mukhena gahetvā asucinā āsittassa pariḷāho vūpasammati, ayaṁ āsittapaṇḍako. Yassa paresaṁ ajjhācāraṁ passato usūyāya uppannāya pariḷāho vūpasammati, ayaṁ usūyapaṇḍako. Yassa upakkamena bījāni apanītāni, ayaṁ opakkamikapaṇḍako. Ekacco pana akusalavipākānubhāvena kāḷapakkhe paṇḍako hoti, juṇhapakkhe panassa pariḷāho vūpasammati, ayaṁ pakkhapaṇḍako. Yo pana paṭisandhiyaṁyeva abhāvako uppanno, ayaṁ napuṁsakapaṇḍakoti. Tesu āsittapaṇḍakassa ca usūyapaṇḍakassa ca pabbajjā na vāritā, itaresaṁ tiṇṇaṁ vāritā. Tesupi pakkhapaṇḍakassa yasmiṁ pakkhe paṇḍako hoti, tasmiṁyevassa pakkhe pabbajjā vāritāti kurundiyaṁ vuttaṁ. Yassa cettha pabbajjā vāritā, taṁ sandhāya idaṁ vuttaṁ – ‘‘anupasampanno na upasampādetabbo upasampanno nāsetabbo’’ti. Sopi liṅganāsaneneva nāsetabbo. Ito paraṁ ‘‘nāsetabbo’’ti vuttesupi eseva nayo.",
  "atk-vin02a20:280": "Paṇḍavatthukathā niṭṭhitā.",

  https://web.archive.org/web/20170404051712/http://www.australianhumanitiesreview.org/archive/issue1-feb-mar-96/jackson/references.html translates and explain these pandaka terms