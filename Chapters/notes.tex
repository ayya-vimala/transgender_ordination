-------------------------------------------


Check this for Hijra .... 
So for instance the {\em paṇḍaka} is 種不能男 (i.e. a male incapable of producing seed) but this 'impotence' seems to also be taken more broadly in a social sense; some men like those born of a concubine were not allowed to marry and also seen as pandaka. That is very much in line with how I currently understand how the Hijra of India work. Of course these things meant different things in a Chinese culture, who did not have Hijra. So a second word was made up for Pandaka, which is 黃門, which means 'eunuch' according to the dictionary, but I found it did not get that meaning until the Eastern Han dynasty. Before that it seems to be more in its original meaning related to "妒黃門" ("dùhuángmén","man impotent due to lack of control of emissions". It has to do something with the Indian medical system of how male and female sexuality are associated with the taste for certain types of food. So the expression of female sexuality in the desire for a man is like bile, which causes craving for something sweet. Now the Chinese character for bile is 黃熱. So an impotent man 黃門 is seen as having female sexuality i.e. desire for other males and therefore not suitable as a monk. These men were seen as hyperlibidous, which is in line with the description that the Jains give to those of the third sex. Of course all this is very theoretical and this kind of thinking is not mentioned in the early Jain scriptures but appears in the Jain commentarial texts around 5th century CE. 


The {\em Śāriputraparipṛcchā} attributes the schism of the {\em Mahāsaṅghika} school with the other schools at around 150 BCE to an attempt to expand the Vinaya by the other schools\footnote{See \cite{sujato2012}}. We would therefore expect that this school would have the le



二根者 or simply 二根 (2 roots/faculties) is translated as hermaphrodite in the dictionary, but no description, nor any stories are given in any of the schools. The passage for {\em ubhatob­yañ­janaka} as we see in the Theravāda Vinaya is not there. It is simply mentioned in passing that these are also not allowed to ordain. The word {\em ubhatob­yañ­janaka} (i.e. one with both signs) seems to be consistent with 二根.



---------------------------------------------------------------


Interest in the origins of the Mahāsaṅghika school lies in the fact that their Vinaya recension appears in several ways to represent an older redaction overall. (Redaction is a form of editing in which multiple sources of texts are combined (redacted) and altered slightly to make a single document. Often this is a method of collecting a series of writings on a similar theme and creating a definitive and coherent work.)



Samantapāsādikā is a translation of Sinhala commentaries into Theravāda by Bhikkhu Buddhaghosa in the 5th century CE. It was based on the Mahāpaccariya and the Kurundī Atthakathā. This passage comes literally out of the Mahāvagga-aṭṭhakathā - mahākhandhakaṃ.

https://www.bps.lk/olib/wh/wh113_Goonesekere_Buddhist-Commentarial-Literature.html#A77897789hakath257
 The commentaries that Mahinda is said to have brought to Ceylon, along with the canon, probably consisted of the expositions as laid down at the Third Council which had just been concluded. But this is disputed. In any case, these commentaries are fairly old.



"妒黃門":["dùhuángmén","man impotent due to lack of control of emissions"],
"牛黃":["niúhuáng","bovine bile"],
"犍黃門":["jiān huángmén","palace eunuch"],
"黃熱":["huángrè","bile"],


Dīrghāgama Sutta T24 describes how at first all beings in all the heavens were subject to marriage between men and women. But then they were bestowed with the gift that:
"Henceforth all the heavens above shall be free from all marriages, so that there shall be no distinction between male and female. Bhikkhus, the Asuras, the Heavens of the Four Great Kings, the heaven of the Thirty-three, when they so desire they can have the same faculties (i.e. 二根 as above), and the same aura comes out of them, just like the Dragons, the Garudas, and there will be no difference between them.


乳房 = breasts



sanskrit vinaya often speaks of strīpuruṣapaṇḍakam in the Varṣāvastu


It is clear that {\em ubhatob­yañ­janaka} is translated as 二根 i.e. having two roots/faculties but without explanation in the Vinaya. 


Moreover, while all the Vinaya's agree in the sections on the m/patricide, killing an arahant, harming a Buddha and animals, the sections on Pandaka and "2 roots/faculties" differ. Some Vinaya's list 5 or 6 kinds that all cannot ordain (not all consistently the same 5), some are like the Pali and don't mention the types of Pandaka at all, just mention that pandaka cannot ordain. 

With regards to ubhatovyañjanaka the scriptures are more consistent: The Vinaya's only mention they cannot ordain (always in conjuction with the word 黃門 (yellow gate)), the commentaries all say the same thing with minor variations: There are three kinds of 2-facultied people: those who can impregnate and conceive; those who can impregnate but not conceive; and those who cannot impregnate but who can conceive. These three types of people are not allowed to become monks and take the full precepts; if they have already taken the full precepts, they should be expelled.It feels like the  ubhatovyañjanaka is simply something to fill a gap, without quite understanding what it could be or who it could refer to. It seems more theoretical than anything else. And the conjuction with the Yellow Gate would indicate it is a late introduction also.



The Dharmaguptaka Bhikkhunī Khandhaka also mentions that women with no breasts should not be given ordination.



\subsection{Scriptural definitions}

\subsubsection{Paṇḍaka}

Pali word: {\em paṇḍaka} \\
Pali dictionary: \href{https://suttacentral.net/define/pa%E1%B9%87%E1%B8%8Daka}{see SuttaCentral} \\
Sanskrit word: {\em paṇḍaka} \\
Tibetan word: {\em ma ning} or {\em ’dod ’gro} \\
Chinese word: {\em 非男非女} or {\em 半擇} (first one lit means "neither male nor female".\\
ITLR dictionary: \href{http://www.itlr.net/hwid:281142}{itlr.net} \\

\medskip

The term {\em paṇḍaka } and it's female form {\em itthipaṇḍaka } have been discussed at length in the last years. The strictest readings of the word as used in the Buddhist world simply use this term very loosly to mean all LGBTIQA+, but this is a much too simple perspective, which also is very hurtful for many people. At the time when the Vinaya was laid down, the word most likely meant "eunuch" (\cite{vimala}). However, the meaning has shifted over time and the commentaries describe an individual who has a much higher libido than others.

{\em Itthipaṇḍaka } occurs far less in the canon and it's meaning is not quite sure because a female eunuch seems like something that cannot exist The word only appears in the later additions to the Vinaya.

The word is not found in any of the early Buddhist Suttas, nor does it appear in the pātimokkhas, the lists of rules for monastics. Next to the pāli Vinaya, it appears twice in the Aṅguttara Nikāya, but both of these only have parallels to the Vinaya or later texts.


\subsubsection{Ubhatob­yañ­janaka}

Pali word: {\em ubhatob­yañ­janaka} or {\em ubhatovyañ­janaka} \\
Pali dictionary: \href{https://suttacentral.net/define/ubhatovya%C3%B1janaka}{see SuttaCentral} \\
Sanskrit word: {\em ubhayavyañjana} \\
Tibetan word: {\em mtshan gnyis pa} \\
ITLR dictionary: \href{http://www.itlr.net/hwid:62844}{itlr.net} \\

\medskip

Traditionally, the word {\em ubhatob­yañ­janaka } was translated as "hermaphrodite" as is mentioned in the Pali Text Society dictionary. But the word "hermaphrodite" has shifted in meaning over the last decades with a greater understanding of the complex variations of natural bodies as well as the understanding that human beings are never able to be fully hermaphrodite i.e. can never reproduce both as male and female but are always dominant in one. Ajahn Brahmali uses the word "intersex" as as translation.

It would probably more accurate to refer to this as a more narowly defined subset of intersex rather than all intersex people. I would propose a subset of true gonadal intersex but will come back to this later in this paper.

Just as with the word {\em paṇḍaka }, it seems that this word has shifted in meaning over time to mean a more lustful individual (\cite{vimala}).

This word appears in various chapters of the Vinaya Khandaka, but never in the early Buddhist Suttas, nor in the pātimokkhas. It appears in the Milindapañha, the questions of a Greek king and therefore a much later text, even goes as far as to mention that a {\em paṇḍaka } or a {\em ubhatob­yañ­janaka } cannot attain enlightenment but there does not seem to be a basis for this assertion in any of the Early Buddhist texts. This seems to be added later at the end of a list of those who cannot attain enlightenment which is found elsewhere in the canon.


\subsubsection{Other references}
There are various other words mentioned in the ordination procedures for Bhikkhuni as described in Bhikkhunikkhandhaka that might be interesting in this context. These do not excluse from ordination and have been translated by Ajahn Brahmali as follows: \\

\begin{tabular}{ l l }
 {\em animittā } & woman who lacks genitals \\
 {\em nimittamattā } & woman with incomplete genitals \\ 
 {\em vepurisikā } & woman who is manlike \\
\end{tabular} \\

The word {\em animittā } literally means "signless" and appears a number of times in the canon (excluding commentaries) but mostly in a different meaning, namely as in {\em Animitto (ceto)samādhi }, which is translated by Bhikkhu Sujato as "signless immersion", a term used in the context of meditation. In the context of not having genitals, it only appears in the canon in the Bhikkhunikkhandhaka and as a form of abuse for women in the Bhikkhu Saṃ­ghā­di­sesa­ 3, never on it's own but always in the same sequence of words of which the above are a few. This points towards a later development.

The other words mentioned here also only appear in these two places in the canon and always exclusively refer to women. The words {\em (itthi)paṇḍaka } and {\em ubhatob­yañ­janaka } also appear in the same sequence.

The reason I mention these words here is that although they do not exclude a candidate from ordination, these categories would be seen as various forms of intersex in our modern understanding. Yet they are denoted separately in the context of Bhikkhuni ordination. 


\subsubsection{Napuṃsaka}

Pali word: {\em napuṃsaka} \\
Pali dictionary: \href{https://suttacentral.net/define/napu%E1%B9%83saka}{see SuttaCentral} \\
Sanskrit word: {\em napuṃsaka} \\
Sanskrit dictionary: \href{https://www.wisdomlib.org/definition/napumsaka}{see WisdomLib} \\

\medskip


The term {\em napuṃsaka} does not appear in the Pali Suttas, Vinaya or Abhidhamma but it appears extensively in the later commentaries, especially the Anya. 

\subsubsection{Tṛtīyāprakṛti}

{\em tṛtīyāprakṛti}


\subsection{A note on translation and interpretation}
The main responsibility of a translator is to translate words as accurately as possible according to the meaning at the time the text was laid down. In practise this can be challenging as meanings of words have shifted over time, both in the ancient past as they have in our times. Words used 100 years ago might not be used any more in the same way they are today. It is clear from the above definitions that no modern word can clearly capture the understanding of the words in the canon entirely and it remain a question of interpretation. It is therefore important that we carefully study the socio-cultural conditions under which these words have formed and take these conditions into consideration when making decisions about the lives of others, most notably future candidates for ordination.


----------------------------------------------------------------
In cases such as this, we should always err on the side of compassion. If there is no clear rule in the Vinaya, are we obliged to follow the commentarial interpretation? Of course there is always the danger that an ordination is not accepted in the wider Sangha, but with the Bhikkhuni ordinations we have seen that acceptence gradually grows. It is up to the individual communities, with agreement of all Sangha members, to ordain a person or not and to judge their suitability, even if the interpretation of other communities is different. 



-----------------------------------------------------------------




There are further complications, such as getting agreement from all Sangha members to ordain such a person. It is possible not everyone would feel at ease with it, for a number of possible reasons.
