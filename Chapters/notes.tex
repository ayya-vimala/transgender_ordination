Similar discussion re women held among the Buddhists ... more references

remove em from vinaya schools in all texts

add proper references with page numbers

regex n-dash in between numbers

----------------------------------------------------------------
\footnote{Note that in this work I have deviated from some of the earlier points I made in \cite{vimala} with regards to the Vedic concept of the third sex. I have now rejected certain sources on the basis that I found them unreliable upon closer inspection and I hope to have rectified this with more thorough research.}


This word appears in various chapters of the Vinaya Khandaka, but never in the early Buddhist Suttas, nor in the pātimokkhas. It appears in the Milindapañha, the questions of a Greek king and therefore a much later text, even goes as far as to mention that a {\em paṇḍaka } or a {\em ubhatob­yañ­janaka } cannot attain enlightenment but there does not seem to be a basis for this assertion in any of the Early Buddhist texts. This seems to be added later at the end of a list of those who cannot attain enlightenment which is found elsewhere in the canon.

Interest in the origins of the Mahāsaṅghika school lies in the fact that their Vinaya recension appears in several ways to represent an older redaction overall. (Redaction is a form of editing in which multiple sources of texts are combined (redacted) and altered slightly to make a single document. Often this is a method of collecting a series of writings on a similar theme and creating a definitive and coherent work.)

testicles 及卵
"妒黃門":["dùhuángmén","man impotent due to lack of control of emissions"],
"犍黃門":["jiān huángmén","palace eunuch"],
乳房 = breasts

無乳 = no milk / no breasts

The Dharmaguptaka Bhikkhunī Khandhaka also mentions that women with no breasts should not be given ordination.

\subsection{A note on translation and interpretation}
The main responsibility of a translator is to translate words as accurately as possible according to the meaning at the time the text was laid down. In practise this can be challenging as meanings of words have shifted over time, both in the ancient past as they have in our times. Words used 100 years ago might not be used any more in the same way they are today. It is clear from the above definitions that no modern word can clearly capture the understanding of the words in the canon entirely and it remain a question of interpretation. It is therefore important that we carefully study the socio-cultural conditions under which these words have formed and take these conditions into consideration when making decisions about the lives of others, most notably future candidates for ordination.


By the start of the common era, a term literally meaning "third sex" was introduced, possibly in the schools of traditional medicine, next to male and female ({\em tṛtīyāprakṛti}) and like the other two determined at conception by biological causes. \footnote{It is interesting to note that this third sex was seen as an equilibrium between male and female; it develops when the father's seed and the mother's blood are in equilibrium and the third sex also develops in the middle of the womb rather than on the right or left as with males or females. See footnote 16 in \cite{zwilling} for a detailed description.} In the 4th century CE the term had become synonymous to {\em napuṃsaka}.


\subsection{Paṇḍaka}

\begin{quote}
{\em Tena kho pana samayena aññataro paṇḍako bhikkhūsu \textbf{pabbajito} hoti. So dahare dahare bhikkhū upasaṅkamitvā evaṃ vadeti—“etha, maṃ āyasmanto dūsethā”ti. Bhikkhū apasādenti—“nassa, paṇḍaka, vinassa, paṇḍaka, ko tayā attho”ti. So bhikkhūhi apasādito mahante mahante moḷigalle sāmaṇere upasaṅkamitvā evaṃ vadeti—“etha, maṃ āvuso dūsethā”ti. Sāmaṇerā apasādenti—“nassa, paṇḍaka, vinassa, paṇḍaka, ko tayā attho”ti. So sāmaṇerehi apasādito hatthibhaṇḍe assabhaṇḍe upasaṅkamitvā evaṃ vadeti—“etha, maṃ āvuso dūsethā”ti. Hatthibhaṇḍā assabhaṇḍā dūsesuṃ. Te ujjhāyanti khiyyanti vipācenti—“paṇḍakā ime samaṇā sakyaputtiyā. Yepi imesaṃ na paṇḍakā, tepi ime paṇḍake dūsenti. Evaṃ ime sabbeva abrahmacārino”ti. Assosuṃ kho bhikkhū tesaṃ hatthibhaṇḍānaṃ assabhaṇḍānaṃ ujjhāyantānaṃ khiyyantānaṃ vipācentānaṃ. Atha kho te bhikkhū bhagavato etamatthaṃ ārocesuṃ. “Paṇḍako, bhikkhave, anupasampanno na \textbf{upasampādetabbo}, upasampanno nāsetabbo”ti. (Mahakkhandhaka, PTS 1.86)}
\end{quote}

\medskip

\begin{quote}
At one time a certain {\em paṇḍaka} had gone forth as a monk. He approached the young monks and said, “Venerables, come and have sex with me.”
The monks dismissed him, “Go away, {\em paṇḍaka}. Who needs you?”
\end{quote}
\begin{quote}
He went to the big and fat novices, said the same thing, and got the same response.
He then went to the elephant keepers and horse keepers, and once again he said the same thing. And they had sex with him. They complained and criticized them, “These Sakyan ascetics are {\em paṇḍaka}s. And those who are not have sex with them. None of them is celibate.”
\end{quote}
\begin{quote}
The monks heard their complaints. They told the Buddha and he said, “A {\em paṇḍaka} should not be given the full ordination. If it has been given, he should be expelled.”
\end{quote}



\subsection{Ubhatob­yañ­janaka}

\begin{quote}
{\em Tena kho pana samayena aññataro ubhatobyañjanako bhikkhūsu \textbf{pabbajito} hoti. So karotipi kārāpetipi. Bhagavato etamatthaṃ ārocesuṃ. Ubhatobyañjanako, bhikkhave, anupasampanno na upasampādetabbo, \textbf{upasampanno} nāsetabboti. (Mahakkhandhaka, PTS 1.89)}
\end{quote}

\medskip

\begin{quote}
At one time an {\em ubhatob­yañ­janaka} had gone forth as a monk. He had sex and made others have it.
\end{quote}
\begin{quote}
They told the Buddha and he said, “An {\em ubhatob­yañ­janaka} should not be given the full ordination. If it has been given, he should be expelled.”
\end{quote}

The commentary {\em (Samantapādādikā, vol. 3, para. 116)} mentions the following about {\em ubhatob­yañ­janaka}:

\begin{quote}
Ubhatobyañjanako ... so duvidho hoti – itthiubhatobyañjanako, purisaubhatobyañjanakoti. Tattha itthiubhatobyañjanakassa itthinimittaṃ pākaṭaṃ hoti, purisanimittaṃ paṭicchannaṃ. Purisaubhatobyañjanakassa purisanimittaṃ pākaṭaṃ, itthinimittaṃ paṭicchannaṃ. ... Imassa pana duvidhassāpi ubhatobyañjanakassa neva pabbajjā atthi, na upasampadāti
\end{quote}

\medskip

\begin{quote}
The {\em ubhatob­yañ­janaka} is of two sorts: the female {\em ubhatob­yañ­janaka} and the male {\em ubhatob­yañ­janaka}. For the female {\em ubhatob­yañ­janaka} the female characteristics are revealed, whereas the male characteristics are concealed. For the male {\em ubhatob­yañ­janaka} the male characteristics are revealed, whereas the female characteristics are concealed. ... 
\end{quote}
\begin{quote}
For either of these there is neither a going forth ({\em pabbajjā}) nor a full ordination ({\em upasampadā}).
\end{quote}

The commentary makes a distinction between male and female {\em ubhatob­yañ­janaka} whereby characteristics of the other sex are hidden. This is a much broader definition and a much broader subset of the term "intersex" as we know it today. The fact that they are predominantly male or female would be a fairly objective basis for deciding on ordination with the Bhikkhus or Bhikkhunis. This distinction is not mentioned in the Vinaya. But the commentary also makes a distinction between {\em pabbajjā} and {\em upasampadā} and does not allow either for ordination.




  "atk-vin02a20:323": "Ubhatobyañjanakavatthukathā",
  "atk-vin02a20:324": "116.Ubhatobyañjanako bhikkhaveti itthinimittuppādanakammato ca purisanimittuppādanakammato ca ubhato byañjanamassa atthīti ubhatobyañjanako. Karotīti purisanimittena itthīsu methunavītikkamaṁ karoti. Kārāpetīti paraṁ samādapetvā attano itthinimitte kārāpeti, so duvidho hoti – itthiubhatobyañjanako, purisaubhatobyañjanakoti.",
  "atk-vin02a20:325": "Tattha itthiubhatobyañjanakassa itthinimittaṁ pākaṭaṁ hoti, purisanimittaṁ paṭicchannaṁ. Purisaubhatobyañjanakassa purisanimittaṁ pākaṭaṁ, itthinimittaṁ paṭicchannaṁ. Itthiubhatobyañjanakassa itthīsu purisattaṁ karontassa itthinimittaṁ paṭicchannaṁ hoti, purisanimittaṁ pākaṭaṁ hoti. Purisaubhatobyañjanakassa purisānaṁ itthibhāvaṁ upagacchantassa purisanimittaṁ paṭicchannaṁ hoti, itthinimittaṁ pākaṭaṁ hoti. Itthiubhatobyañjanako sayañca gabbhaṁ gaṇhāti, parañca gaṇhāpeti. Purisaubhatobyañjanako pana sayaṁ na gaṇhāti, paraṁ gaṇhāpetīti, idametesaṁ nānākaraṇaṁ. Kurundiyaṁ pana vuttaṁ – ‘‘yadi paṭisandhiyaṁ purisaliṅgaṁ pavatte itthiliṅgaṁ nibbattati, yadi paṭisandhiyaṁ itthiliṅgaṁ pavatte purisaliṅgaṁ nibbattatī’’ti . Tattha vicāraṇakkamo vitthārato aṭṭhasāliniyā dhammasaṅgahaṭṭhakathāya veditabbo. Imassa pana duvidhassāpi ubhatobyañjanakassa neva pabbajjā atthi, na upasampadāti idamidha veditabbaṁ.",
  "atk-vin02a20:326": "Ubhatobyajjanakavatthukathā niṭṭhitā.",

Samantapasadika
Ubhatobyañjanako bhikkhaveti itthinimittuppādanakammato ca purisanimittuppādanakammato ca ubhato byañjanamassa atthīti ubhatobyañjanako.Karotīti purisanimittena itthīsu methunavītikkamaṃ karoti. Kārāpetīti paraṃ samādapetvā attano itthinimitte kārāpeti, so duvidho hoti – itthiubhatobyañjanako, purisaubhatobyañjanakoti.Tattha itthiubhatobyañjanakassa itthinimittaṃ pākaṭaṃ hoti, purisanimittaṃ paṭicchannaṃ. Purisaubhatobyañjanakassa purisanimittaṃ pākaṭaṃ, itthinimittaṃ paṭicchannaṃ. Itthiubhatobyañjanakassa itthīsu purisattaṃ karontassa itthinimittaṃ paṭicchannaṃ hoti, purisanimittaṃ pākaṭaṃ hoti. Purisaubhatobyañjanakassa purisānaṃ itthibhāvaṃ upagacchantassa purisanimittaṃ paṭicchannaṃ hoti, itthinimittaṃ pākaṭaṃ hoti. Itthiubhatobyañjanako sayañca gabbhaṃ gaṇhāti, parañca gaṇhāpeti. Purisaubhatobyañjanako pana sayaṃ na gaṇhāti, paraṃ gaṇhāpetīti, idametesaṃ nānākaraṇaṃ. 

Because of kamma giving rise to female characteristics and kamma giving rise to male characteristics, there is for them the characteristics of both. With the male characteristic they act to transgress through sexual intercourse with women. Having encouraged another, they cause action in their own female characteristic. 

[My comment: This seems to refer to a true hermaphrodite, assuming that such people even exist. The positive thing about this interpretation is that I am guessing very few people would be barred from ordaining.] 

They are twofold: the female ubhatobyañjanaka and the male ubhatobyañjanaka. In regard to this, the female characteristic of the female ubhatobyañjanaka is apparent, but the male characteristic is hidden. The male characteristic of the male ubhatobyañjanaka is apparent, but the female characteristic is hidden. 

While the female ubhatobyañjanaka is acting with manliness among women, the female characteristic is hidden, whereas the male characteristic is apparent. 
When the male ubhatobyañjanaka enters the state of a woman for the sake of men, the male characteristic is hidden, whereas the female characteristic is apparent. 
The female ubhatobyañjanaka becomes pregnant and causes others to become pregnant. The male ubhatobyañjanaka does not become pregnant, but causes others to become pregnant. This is the difference between them."



  "atk-vin02a20:277": "Paṇḍakavatthukathā",
  "atk-vin02a20:278": "109.Dahare dahareti taruṇe taruṇe. Moḷigalleti thūlasarīre. Hatthibhaṇḍe assabhaṇḍeti hatthigopake ca assagopake ca.",
  "atk-vin02a20:279": "Paṇḍakobhikkhaveti ettha āsittapaṇḍako usūyapaṇḍako opakkamikapaṇḍako pakkhapaṇḍako napuṁsakapaṇḍakoti pañca paṇḍakā. Tattha yassa paresaṁ aṅgajātaṁ mukhena gahetvā asucinā āsittassa pariḷāho vūpasammati, ayaṁ āsittapaṇḍako. Yassa paresaṁ ajjhācāraṁ passato usūyāya uppannāya pariḷāho vūpasammati, ayaṁ usūyapaṇḍako. Yassa upakkamena bījāni apanītāni, ayaṁ opakkamikapaṇḍako. Ekacco pana akusalavipākānubhāvena kāḷapakkhe paṇḍako hoti, juṇhapakkhe panassa pariḷāho vūpasammati, ayaṁ pakkhapaṇḍako. Yo pana paṭisandhiyaṁyeva abhāvako uppanno, ayaṁ napuṁsakapaṇḍakoti. Tesu āsittapaṇḍakassa ca usūyapaṇḍakassa ca pabbajjā na vāritā, itaresaṁ tiṇṇaṁ vāritā. Tesupi pakkhapaṇḍakassa yasmiṁ pakkhe paṇḍako hoti, tasmiṁyevassa pakkhe pabbajjā vāritāti kurundiyaṁ vuttaṁ. Yassa cettha pabbajjā vāritā, taṁ sandhāya idaṁ vuttaṁ – ‘‘anupasampanno na upasampādetabbo upasampanno nāsetabbo’’ti. Sopi liṅganāsaneneva nāsetabbo. Ito paraṁ ‘‘nāsetabbo’’ti vuttesupi eseva nayo.",
  "atk-vin02a20:280": "Paṇḍavatthukathā niṭṭhitā.",

--------------------------------------------------------------
Religion plays a large role in socializing individuals to their assigned gender-roles. Buddhism is certainly no exception to that, but we have to make a difference here between the Buddha’s teachings, and Buddhist culture, which has developed based on geographical and interpretational differences since the time of the Buddha.

In order to understand the Buddhist’ point of view we have to go back in history. As Ayya Sujato pointed out in his article on the Buddha’s genitals 16, the Buddhas himself appears to be more non-binary: he has gone beyond the notion of gender. But how were things in the Buddha’s time? This was a very different society than ours and the only things we know about it have come to us through the background stories of the Suttas and the Vinaya. But what that tells us is that in social relationships, this society was not so very different from India today. In any case, it was most likely a hetero-patriarchical society where forced marriage was the norm.

It was in this society that the Buddha had to teach. Even if he himself felt different, he would not directly challenge existing structures in society but would teach people to contemplate and look inside of themselves. His way of teaching was very subtle, never lecturing but always guiding people to find answers inside. And what he taught was to be compassionate to all beings, regardless of caste and gender.

In his Teachings he makes very clear that these distinctions between humans are irrelevant:

“Neither in neck, nor shoulders found,
not in belly or the back,
neither in buttocks nor the breast,
not in groin or sexual parts.

Neither in hands nor in the feet,
not in fingers or the nails,
neither in knees nor in the thighs,
not in their “colour”, not in sound,
here is no distinctive mark
as in the many other sorts of birth.

In human bodies as they are,
such differences cannot be found:
the only human differences
are those in names alone.”

Suttanipāta 3.9 4

The Teachings
There are various ways to look at the teachings. One way which I find very helpful here is the teachings on the five Khandhas: form, feeling, perception, choices and consciousness. When teaching about the five Khandhas, the Buddha teaches to contemplate them as Anicca (impermanence), Dukkha (suffering) and Anatta (not-self). He does not say that there is not a self, but that if you identify with a self, you actually identify with these five Khandhas; if you see them in accordance with reality, they don’t operate as a “self”, something you can cling to.

His approach is to encourage investigation. How do you know these things? Go through each aspect of experience and see if it is permanent of impermanent. “Dukkha” is a bit more subtle and sometimes confusing because the term “dukkha” is here used as a characteristic of the five Khandhas and is not used in the same sense as the feeling of “dukkha”, which is part of the second Khandha. What is meant here is something like seeing the imperfection of things, seeing that it is not fit to provide lasting satisfaction. Anatta is seeing that these experiences cannot be controlled and are therefore not a self.

One of the most important things about the five Khandhas is not the particular definition of each each but the interrelationship between the first four (form, feeling, perception, choices) with the fifth (consciousness). The interaction, the responsiveness and the resonance between the inner subjective awareness and the sentient body, between the inner sense of awareness and the external signs. We don’t simply see objects as annica, dukkha, Anatta but the very nature and structure of the interaction itself is interdependent and constantly spinning around, constantly changing. Deep insight is not about knowing what external objects are but how the mind is involved with these objects.

In relation to gender, we can see that our gender-identity is a perception. Whether we identify as man, woman or something else, we all have this perception. We also have an idea about how other people perceive us and this is where our assigned gender-role comes into play. This gender-role is a socially conditioned phenemenon, but it also has an internal part, namely how we perceive this in relation to ourselves; the interaction between what is inside and what is outside. We have no control over how these things are and we cannot will ourselves to have a different gender-identity. It is Anatta. It is as it is and we cannot change it. Our gender-identity is not a choice.

But when contemplating, there is a shift from Sañña (perception) to Pañña (wisdom). By observing things, slowly our defilements disappear and we will start to see things in a different light. The inner qualities of the mind change while wisdom grows. This way, we can let go of our thinking about how we “should be” according to some perceived social standard outside of us and learn to accept ourselves, and our gender, as we are, while keeping in mind that it is Anatta. This way we change our relationship to the outside world and we change our perception of ourselves and others.